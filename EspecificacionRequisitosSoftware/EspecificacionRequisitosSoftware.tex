% Para documento texto corto
%\documentclass[paper=a4,oneside,fontsize=12pt]{scrartcl}
\documentclass[paper=letter,oneside,fontsize=12pt, parskip=full]{scrartcl}

% Establecer dimensiones de los margenes
%\ usepackage[inner=1.5cm,outer=3cm,top=2cm,bottom=4cm,
%	bindingoffset=5mm]{geometry}
\usepackage[left=3cm,right=3cm,top=3cm,bottom=3cm,
bindingoffset=0cm]{geometry}

% Permite ingresar caracteres acentuados y especiales 
% sin necesidad de emplear comando
% utf8 codificacion de entrada Unicode (mas simbolos que ASCII)
\usepackage[utf8]{inputenc}

% T1 encoding for European, English, American text
\usepackage[T1]{fontenc}
% Fuente escalable
\usepackage{lmodern}

% Carga babel, idioma ingles
\usepackage[english,spanish]{babel}

% Mejor jsutificacion, tipografia alta calidad.
\usepackage{microtype}

% Agrega comandos extra al comando tabular
% \toprule, \midrule, \bottomrule
\usepackage{booktabs}
% Tablas con ancho establecido por usuario
\usepackage{tabularx}
% Para posicionamiento preciso de tablas dentro del texto
\usepackage{float}

% Encabezados personalizados
\usepackage{fancyhdr}
\usepackage{graphicx}

% Permite obtener el numero de la ultima pagina
\usepackage{lastpage}

% Cabeceras
\pagestyle{fancy}
% Borra cabecera y pie actuales
\fancyhead{}
% Cintillo cabecera
%\chead{
%	\includegraphics[width=150mm]{Imagenes/Cabecera.png}
%}
\fancyhead[L]{\includegraphics[width=150mm]{Imagenes/Cabecera.png}}
\fancyfoot[C]{
	\begin{tabular}{|m{2.0cm}|m{10.0cm}|m{2.0cm}|}
		\hline
		\centering Versión 0.1	& 
		\centering \includegraphics[height=0.8cm]{Imagenes/Pie.png} & 
		{\centering  \thepage / \pageref{LastPage}} \\			
		\hline 
	\end{tabular}
}

% Numeracion de paginas
% numeros arabigos
\pagenumbering{arabic}

% Comando que define el nombre la aplicacion
\newcommand{\AppName}{\textsc{CenditLab}\ }

% Comando que define el nombre del sistema de medición de figura de ruido
\newcommand{\smr}{sistema de medición de ruido}
\newcommand{\SMR}{SMR}

% Comando que define el nombre del Cendit
\newcommand{\cendit}{Cendit}

\begin{document}
	
	\begin{titlepage}
				
		\begin{center}	
			
			\vspace{10cm}
			
			\begin{Large}			
				Especificación de Requisitos de Software
			\end{Large}
		
			\begin{Huge}
				\textsc{\AppName}
			\end{Huge}
			
		\end{center}
		
	\end{titlepage}

	\tableofcontents
	
	\section{Introducción}
	Esta sección brinda un panorama general del contenido del presente documento de especificaciones de requisitos de software. Se da al final una lista de abreviaturas.
	
	\subsection{Propósito}
	El propósito de este documento es dar una descripción detallada de los requerimientos para el software \AppName. Ilustrará el propósito y sentará de manera formal y completa la base inicial para el proceso de desarrollo. Explicará la restricciones del sistema, interfaces e interacciones con aplicaciones externas. 
	
	Este documento se dirige principalmente a los futuros usuarios de la software \AppName para fines de aprobación y para desarrolladores, que en el futuro se encarguen de mantener o ampliar la aplicación.
	
	\subsection{Alcance}
	
	El software \AppName es una aplicación de escritorio que brinda a los usuarios del \emph{\smr} un entorno de trabajo o laboratorio virtual 
	
	El software \AppName es una aplicación para PC de escritorio que sirve como interfaz de software para el \emph{\smr} presente en el \cendit. Permite realizar muchas de las tareas propias de los equipos de un \smr dentro de un entorno de trabajo en una aplicación centralizada, o laboratorio virtual. 
	
	\AppName permite a los usuarios realizar mediciones de manera automatizada y remota, configuración los instrumentos instrumentos del \smr, capturar y visualizar de datos además de generar reportes en formatos digitales, como pdf o html.
	
	\AppName utiliza la capacidad de transferencia de datos a través de buses de comunicaciones que disponen los instrumentos del \smr, como GPIB, USB o LAN. Para ello el PC donde se ejecute la aplicación debe disponer de los aditamentos de hardware apropiado que permitan acceder a estos buses, asi como también el respectivo soporte de software en forma de librerías o controladores de dispositivo, que permitan a un PC el acceso a estos buses.
	
	\subsection{Definiciones, acronimos y abreviaturas}
	
	\begin{table}[h!]
		\begin{tabular}{|p{4cm}|p{10cm}|} 
			\hline
			Termino & Definición \\
			\hline
		\end{tabular}	
	\end{table}

	\subsection{Referencias}
	% 	\bibliographystyle{acm}	
	% 	\bibliographystyle{abbrv}	
	% 	\bibliographystyle{apalike}
	
	%\bibliographystyle{ieeetr}
	%\bibliography{Bibliografia}
	
	\begingroup
		\renewcommand{\section}[2]{}
		\begin{thebibliography}{9}
			\bibitem{iso-29148} 
			ISO/IEC/IEEE 29148. 
			\textit{Systems and software engineering - Lyfe cycle processes - Requirements enginering}. 
			ISO/IEC/IEEE, 2011.
			
			\bibitem{iso-42010} 
			ISO/IEC/IEEE 42010. 
			\textit{Systems and software engineering - Architecture description}. 
			ISO/IEC/IEEE, 2011.		
			
			\bibitem{softwareengineering-narang} 
			Rajesh Narang. 
			\textit{Software Engineering Principles and Practices}. 
			Mc Graw Hill, New Delhi, India, 2015.			
			
			\bibitem{softwareengineering-sommerville} 
			Ian Sommerville. 
			\textit{Software Engineering}. 
			Addison-Wesley, Boston, Massachusetts, 2011.					
			
			\bibitem{latexcompanion} 
			Michel Goossens, Frank Mittelbach, and Alexander Samarin. 
			\textit{The \LaTeX\ Companion}. 
			Addison-Wesley, Boston, Massachusetts, 2011.		
			
			\bibitem{einstein} 
			Albert Einstein. 
			\textit{Zur Elektrodynamik bewegter K{\"o}rper}. (German) 
			[\textit{On the electrodynamics of moving bodies}]. 
			Annalen der Physik, 322(10):891–921, 1905.
			
			\bibitem{knuthwebsite} 
			Knuth: Computers and Typesetting,
			\\\texttt{http://www-cs-faculty.stanford.edu/\~{}uno/abcde.html}
		\end{thebibliography}
	\endgroup

	
	\subsection{Visión general}
	
	El resto de este documento incluye tres capítulos y un apéndice. El segundo provee un panorama de la funcionalidad del sistema y su interacción con el \smr. Este capitulo además introduce los diferentes tipos de \emph{stakeholders} y su interacción con el sistema. Más adelante, el capitulo menciona las restricciones del sistema y las suposiciones de la aplicación.
	
	El tercer capitulo describe la especificación de requerimientos al detalle y describe las diferentes interfaces del sistema. Se emplean diferentes técnicas de especificación con el objeto de mostrar los requerimientos de manera precisa para distintas audiencias.
	
	El cuarto capitulo trata con la priorización de los requerimientos. Este incluye una motivación. Expone los motivos para los métodos de prioridad escogidos y el por que de las alternativas rechazadas.  
	
	El apéndice al final de este documento incluye los resultados de la priorización de requerimientos y el plan de entrega basado en las prioridades.
	
	\section{Descripción general}
	
	Esta sección brindará una perspectiva general de todo el sistema. El sistema será explicado en su contexto para mostrar como el sistema interactúa con otros sistemas y su funcionalidad básica. Describirá además que tipo de participantes <stakeholders> usarán el sistema y que funcionalidad estará disponible para cada tipo de ellos. Al final, se presentan las restricciones y suposiciones para el sistema.
	
	\subsection{Perspectiva de producto}
	
	\AppName será un aplicación de escritorio que servirá como interfaz de usuario de software para el \smr. El software del sistema consiste de tres partes: el soporte para interfaz gráfica, la capa de abstracción de comunicaciones y la automatizador de mediciones.	
	
	\subsection{Funciones de producto}
	
	\subsection{Características de los usuarios}
	
	La aplicación \AppName está dirigida a usuarios con formación técnica en el tarea de radio frecuencia, comunicaciones, antenas, electrónica y afines. En estos campos se  consideran Ingenieros y Técnicos Universitarios como \emph{usuarios técnicos} en este documento.
	
	\subsection{Restricciones}
	
	\subsection{Suposiciones y dependencias}	
	
	\section{Requerimientos específicos}
	
	\subsection{Interfaces de usuario}
	
	\subsubsection{Interfaces de hardware}
	
	\subsubsection{Interfaces de software}
	
	\subsubsection{Interfaces de comunicaciones}
	
		% -----------------------------------------------------
		% Comando para generar formato con requisitos funcionales
		\newcommand{\funcreq}[7]{
			\subsubsection{Requerimiento funcional {#1}}

			\begin{table}[H]			
			%\begin{table}[!htbp]
				\begin{tabularx}{\textwidth}{rX}			
					\textbf{ID:} 	&	{#2}		\\
					Titulo:			&  	{#3}. 		\\
					Descripción:	&	{#4}.		\\
					Justificación: 	&	{#5}.		\\
					Dependencias:	& 	{#6}.		\\
					Usuario:		&	{#7}.		\\
				\end{tabularx}				
			\end{table}
		}
		% -----------------------------------------------------		
		
		% -----------------------------------------------------
		% Comando para generar formato con requisitos de desempeño
		\newcommand{\perfreq}[6]{
			\subsubsection{{Requerimiento de desempeño #1}}
			
			\begin{table}[H]%[!htbp]
				\begin{tabular}{rl}			
					\textbf{ID:} 	&	{#2}		\\
					Titulo:			&  	{#3}. 		\\
					Descripción:	&	{#4}.		\\
					Justificación: 	&	{#5}.		\\
					Dependencias:	& 	{#6}.		\\
				\end{tabular}				
			\end{table}
		}
		% -----------------------------------------------------				
	
	\subsection{Requerimientos funcionales}	
	
	A continuación los requerimientos funcionales
		
	\funcreq{1}
		{FR1}
			{Programar tareas de medición}
			{Los usuarios pueden programar los instrumentos y generar tareas de medición para sus posterior ejecución, de forma automatizada y remota}
			{Programar y automatizar las mediciones}
			{Ninguna}
			{Técnico}
	\funcreq{2}
		{FR2}
			{Explorador de instrumentos conectados a los buses}
			{El usuario podrá visualizar los instrumentos conectados a los buses, al cual el PC tiene acceso}
			{Con el fin de mostrar un listado de instrumentos en línea}
			{Ninguna}
			{Técnico}
	\funcreq{3}
		{FR3}
			{Configurar los instrumentos de medición}
			{Al usuario se le presentara una interfaz gráfica en donde podrá establecer la configuración de cada instrumento en linea}
			{Los instrumentos requieren una configuración previa antes de iniciar la medición}
			{FR2}
			{Técnico}
	\funcreq{4}
		{FR4}
			{Capacidad de programar el proceso medición}
			{El usuario puede elegir un conjunto de instrumentos y establecer una secuencia de comandos para estos, que podrá almacenar en disco y ejecutar en el futuro}
			{Con el objeto de automatizar los procesos de medición}
			{FR2, FR3}
			{Técnico}			
	\funcreq{5}
		{FR5}
			{Capacidad para almacenar y cargar datos de calibración de instrumentos}
			{El usuario puede guardar almacenar los datos de calibración y cargarlos nuevamente antes de cada proceso de medición}
			{Con el objeto de automatizar los procesos de medición}
			{FR3}
			{Técnico}
	\funcreq{6}
		{FR5}
			{Capacidad de programar la secuencia de pasos en el proceso de medición}
			{El usuario podrá programar la secuencia de pasos a realizar en un proceso de medición dado}
			{Con el objecto de automatizar el proceso de medición}
			{FR2, FR3}	
			{Técnico}		
	\funcreq{6}
		{FR6}
			{Capacidad programar la presentación gráfica reporte con resultados de medición}
			{El usuario podrá escoger los datos resultado de la medición y configurar su presentación y disposición en el documento de salida}
			{Con el objeto de configurar el documento que presenta los resultados de medición}
			{FR3,FR4}
			{Técnico}
	\funcreq{7}
		{FR7}
			{Interfaz gráfica orientada a diagrama de bloques}
			{El usuario puede programar las tareas de medición por medio de la creación de diagramas de bloques, que representan funcionalidades diversas}
			{Con el objeto de automatizar el proceso de medición}	
			{}		
			{Técnico}			
			
	\subsection{Requerimientos de desempeño}	
			
	\subsection{Restricciones de diseño}
	
	\subsection{Atributos del sistema de software}
	
	\perfreq{1}
		{QR1}
			{Diseño de interfaz gráfica limpio, descongestionado y ordenado}
			{La interfaz gráfica deberá presentar un diseño limpio y estructurado, con funcionalidad común agrupada en pantallas y el anidamiento de menús no sobrepasará de 2 niveles. Evitar la congestión de la interfaz gráfica.}
			{Con el objeto de facilitar la navegación por las pantallas de la aplicación}
			{Ninguna}
		
	\perfreq{2}
		{QR2}
			{La aplicación para PC deberá ser portable entre los SO Windows y Linux}
			{La aplicación debe ejecutarse de manera uniforme en los sistemas operativos Windows y Linux}
			{Facilitar al usuario el uso de la aplicación en ambos sistemas operativos.}
			{Ninguna}		
	
	
	
\end{document}