% Para documento texto corto
\documentclass[paper=letter,oneside,fontsize=12pt, parskip=full]{article}
%\documentclass[paper=letter,oneside,fontsize=11pt, parskip=full]{scrartcl}
%\documentclass{amsart}
%\documentclass[paper=letter,oneside,fontsize=12pt]{scrartcl}

% Establece dimensiones de los margenes
% \usepackage[inner=1.5cm,outer=3cm,top=2cm,bottom=4cm,
% bindingoffset=5mm]{geometry}
\usepackage[left=3cm,right=3cm,top=3cm,bottom=3cm,
bindingoffset=0cm, footskip=0.5cm, headheight=2cm]{geometry}

% Carga babel, idioma ingles
\usepackage[english, spanish]{babel}

% Elimina sangrias y aumenta espacio entre parrafos
\usepackage{parskip}

% Permite ingresar caracteres acentuados y especiales 
% sin necesidad de emplear comando
% utf8 codificacion de entrada Unicode (mas simbolos que ASCII)
\usepackage[utf8]{inputenc}

% T1 encoding for European, English, American text
\usepackage[T1]{fontenc}
% Fuente escalable
\usepackage{lmodern}

% Agrega comandos extra al comando tabular
% \toprule, \midrule, \bottomrule
\usepackage{booktabs}
% Unir filas
\usepackage{multirow}
% Tablas con ancho establecido por usuario
\usepackage{tabularx}
% Para posicionamiento preciso de tablas dentro del texto

\usepackage{float}

% Establece espacio entre lineas
\usepackage[onehalfspacing]{setspace}

% Tabla de tres secciones
\usepackage[flushleft]{threeparttable}

%Para resizebox
\usepackage{graphicx}

%Uso de colores
\usepackage{xcolor}
% Permite controlar colores de tablas
\usepackage{xcolor, colortbl}

% Elimina numeros de pagina del docuento
\pagenumbering{gobble}


% Definicion colores tabla cronograma
\definecolor{colorta}{rgb}{0.3569,0.608,0.8353}
\definecolor{colortb}{rgb}{0.4392,0.678,0.2784}
\definecolor{colortc}{rgb}{1.0000,0.361,0.0000}
\definecolor{colortd}{rgb}{0.1804,0.455,0.7098}
\definecolor{colorte}{rgb}{0.9294,0.490,0.1922}
\definecolor{colortf}{rgb}{0.2667,0.3290,0.4157}
\definecolor{colortg}{rgb}{0.3290,0.2667,0.4157}
\definecolor{colortj}{rgb}{0.2667,0.4157,0.329}

\newcommand{\TA}{\cellcolor{colorta}}
\newcommand{\TB}{\cellcolor{colortb}}
\newcommand{\TC}{\cellcolor{colortc}}
\newcommand{\TD}{\cellcolor{colortd}}
\newcommand{\TE}{\cellcolor{colorte}}
\newcommand{\TF}{\cellcolor{colortf}}
\newcommand{\TG}{\cellcolor{colortg}}
\newcommand{\TJ}{\cellcolor{colortj}}

\begin{document}

	\begin{flushright}		
		\large
		Caracas, octubre de 2017. \\	
	\end{flushright}

	
	\begin{onehalfspace}
		\large
		Prof. Ebert Brea. \\
		Jefe del Departamento de Electrónica, Computación y Control. \\
		Escuela de Ingeniería Eléctrica. \\
		Universidad Central de Venezuela.	
	\end{onehalfspace}

	Presente.- 
	
	Tengo el agrado de dirigirme a Usted, en la oportunidad de hacerle llegar un cordial saludo y a la vez solicitarle la extensión del lapso de ejecución de mi Trabajo Especial de Grado (TEG) y el replanteamiento del alcance y de los objetivos del TEG.
	 
	Soy José Aquilino Arias Bustamante, C.I. V-14.666.744 y me encuentro realizando mi TEG dentro de la fundación CENDIT. Aprobado por Consejo de Escuela el 21 de febrero del año en curso, mi TEG comenzó el 6 de marzo dentro de la fundación.
	
	El lapso asignado de 28 semanas, aprobado por Consejo de Escuela para la ejecución del TEG, se agotó sin que haya sido posible culminar el proyecto. Varias han sido las circunstancias surgidas durante el curso del TEG que han impedido aprovechar el tiempo asignado al mismo, entre las cuales menciono las siguientes:
	
	\begin{itemize}
		\item Las protestas acontecidas en las inmediaciones de la fundación CENDIT (base aérea la Carlota) afectó en gran medida la ejecución del proyecto, durante el periodo que va desde comienzos de abril hasta finales de julio (16 semanas).
	
		\item Retardo en la contratación por parte de la fundación CENDIT. Aunque en la planilla de inscripción se indique como fecha de inicio del TEG el 31 de enero de 2017, el CENDIT tramitó mi ingreso como pasante hasta el 6 de marzo del mismo año. 
		
		\item La situación económica actual plantea una seria dificultad en la adquisición de los materiales e insumos básicos que se requieren para sustentar un proyecto como este, que obliga a invertir una buena cantidad de tiempo en la búsqueda de alternativas,
	
		\item Por último, el proyecto asignado por la fundación CENDIT presenta un buen grado de complejidad, la cual no se pudo identificar durante la redacción del anteproyecto, sino hasta haberme involucrado de lleno en el TEG. Éste implica el diseño e implementación de software, hardware y firmware en conjunto con las tareas documentación asociada a cada una de estas fases.
	\end{itemize}
	
	A pesar de los contratiempos citados arriba, se han logrado avances en el proyecto, en especial en las tareas de las fases 1 y 2, dedicadas principalmente a la documentación inicial y al diseño de software y de hardware. 
	
	La solicitud de la extensión del lapso de TEG se realiza con una revisión de su alcance y de sus objetivos, con miras a simplificar el proyecto. El proceso de revisión de y reformulación de objetivos así como la reducción del alcance del proyecto se ha hecho bajo la supervisión de Tutor y del Prof. Guía, quienes suscriben esta misiva. 
	
	Seguidamente presento los cambios que se realizan en objetivos y alcance del proyecto. El título y el objetivo general permanecen sin cambios que, a modo de recordatorio, se indican a continuación,	
	
	\emph{Título del proyecto}

	\begin{minipage}{0.95\textwidth}
		\centering
		DISEÑO DE UN EQUIPO ELECTRÓNICO CONTROLADOR DE INTERRUPTORES Y ATENUADORES EMPLEADO EN LA MEDICIÓN DE LA FIGURA DE RUIDO EN DISPOSITIVOS DE RADIO FRECUENCIA
	\end{minipage}

	\emph{Objetivo general}
	
	\hfill%	
	\begin{minipage}{0.95\textwidth}
		Diseñar un equipo electrónico que permita emular las características funcionales de un controlador electrónico de interruptores y atenuadores.
	\end{minipage}
	
	Inicialmente, los objetivos específicos propuestos en el anteproyecto de TEG eran los siguientes,
	
	\emph{Objetivos específicos, propuestos en el anteproyecto}
	
	\begin{enumerate}
		\item Elaborar un informe técnico a partir de un estudio del funcionamiento de los
		dispositivos presentes en el sistema de medición de figura de ruido de la fundación
		CENDIT.
		
		\item Diseñar un equipo electrónico que permita replicar las características funcionales de
		una unidad de control para atenuadores e interruptores de la serie 11713 de KeySight.
		
		\item Implementar el diseño como dispositivo físico.
		
		\item Integrar el diseño físico en el banco de medición de figura de ruido presente en el
		laboratorio del CENDIT.
		
		\item Generar un manual de usuario para el dispositivo diseñado.
	\end{enumerate}	
	
	Los nuevos objetivos específicos propuestos para el TEG se enumeran seguidamente,
	
	\emph{Objetivos específicos, nueva propuesta}
		
	\begin{enumerate}
		\item Realizar una investigación documental sobre caracterización de dispositivos de radio frecuencia y su medición de figura de ruido.
		
		\item Recopilar la documentación y software asociado al SMFR.
		
		\item Codificar una librería de software para intercambio de datos entre PC y el SMFR.
		
		\item Diseñar y codificar el firmware para dispositivo.
		
		\item Diseño y construir las tarjetas electrónicas PCB para cada uno de los módulos del equipo: expansor de puertos, fuente de alimentación y tarjeta madre.
		
		\item Desarrollar aplicación de software para gestión de la medición de la figura de ruido con el SMFR.
		
		\item Generar manuales de usuario para el equipo y para la aplicación.
	\end{enumerate}		

	\emph{Cambios en el alcance}
	
	El proyecto asignado en la fundación CENDIT presenta un buen grado de complejidad, es bastante exigente en esfuerzo de investigación, diseño, implementación y documentación, así como en recursos materiales; presenta un nivel muy elevado para un proyecto de TEG. Por esta razón y bajo la supervisión y aprobación tanto del Tutor como del Prof Guía, se reduce el alcance del proyecto como se indica seguidamente,
	
	\emph{Cambios en el diseño de hardware}

	En el diseño original para el dispositivo electrónico, se reducen sus  interfaces de comunicaciones, de tres que se habían propuesto inicialmente (GPIB, USB, LAN) a una interfaz (USB). De esta forma, se diseñará un dispositivo con todo el soporte requerido para la interfaz USB (Class Device Communication, full speed).			

	La interfaz eléctrica del dispositivo, que consiste en las señales de control que el dispositivo genera para comandar la unidad de atenuadores y aisladores estarán, constituidas por dos grupos de 16 señales cada uno. 
	
	El panel frontal consistirá de un teclado matricial, de no más de 16 teclas. 
	
	En el diseño y construcción de las tarjetas de circuitos impresos (PCB) se contará con la asesoría y ayuda del personal encargado del departamento de electrónica de la fundación CENDIT, quienes cuentan con experiencia y el equipo necesario para la elaboración de prototipos en PCB.
	
	\emph{Cambios en el diseño de software}
	
	La aplicación a desarrollar, conocida como Software para Gestión de la Medición de Figura de Ruido (SGMFR) consistirá, a grandes rasgos, de lo siguiente.
	
	\begin{enumerate}
		\item Instalador para la aplicación
		\item Soporte para establecer comunicación de datos con los dispositivos del sistema de medición de figura de ruido.
		\item Interfaz de usuario gráfica.
		\item Asistencia al usuario en las etapas de medición: configuración, ejecución y generación de reportes.
		\item Generación de reportes de resultados de medición, en formato pdf.
	\end{enumerate}

	\emph{Cambios en el diseño de firmware}
	
	El firmware que se desarrollará para el microcontrolador central brindará soporte a las comunicaciones por medio de las interfaz USB. Se encargará de gestionar la interacción del usuario con el panel frontal. 	
	
	Para la ejecución de los nuevos objetivos, se propone el siguiente cronograma de actividades,
	
	\emph{Nuevo cronograma de trabajo}

	\begin{center}
		\begin{threeparttable}[!h]	
			\arrayrulecolor{gray}
			\setlength{\extrarowheight}{4pt}		
		
				\begin{tabular}{|c|c|l|l|l|l|l|l|l|l|l|l|l|l|}
					\hline 		
					\textbf{Item} &	
					\begin{tabular}{c}
						{\raggedright \textbf{Semanas}} \\
						\textbf{Tareas}
					\end{tabular}
					 & 1 & 2 & 3 & 4 & 5 & 6 & 7 & 8 & 9 & 10 & 11 & 12 \\
					\hline
					1 & Expansor de puertos Viking & \TA & \TA & \TA & \TA & & & & & & & & \\
					\hline				
					3 & Firmware del dispositivo & & & & & & \TC & \TC & \TC & \TC & & & \\ 
					\hline
					4 & Tarjeta madre & & & & & & & & \TD & \TD & \TD & \TD & \\
					\hline
					5 & Tarjeta de alimentación DC  &  & \TE & \TE & \TE & \TE & & & & & & & \\
					\hline
					6 & Aplicación SGMFR  & & & & & \TG & \TG & \TG & \TG & \TG & \TG & \TG & \TG \\
					\hline
					7 & Libro TEG & & & & & & & \TJ & \TJ & \TJ & \TJ & \TJ & \TJ \\
					\hline
				\end{tabular}	
				%\begin{tablenotes}
					%\small
					%\item Fecha de inicio: 6 de Marzo de 2017. 
					%\item Jornada de 8 horas diarias, lunes a viernes 8:00 AM a 12:00 M y 1:30 %PM a 4:30 PM.
				%\end{tablenotes}
			\caption{Cronograma de actividades para la extensión del lapso de TEG}			
			\label{Tab:CronogramaActividadesInicial}
		\end{threeparttable}
	\end{center}

	Como anexo, en las siguientes tablas se indican las actividades realizadas hasta la fecha. En éstas se indica la fase a la cual pertenece, la actividad y el porcentaje de culminación de cada actividad.

	\begin{table}[h!]
		\begin{tabularx}{\textwidth}{p{0.15\textwidth}p{0.6\textwidth}c}
			\toprule
			{Fase} & 
			{Actividad} & 
			{\% culminación} \\
			\midrule
			\multirow{4}{0.15\textwidth}{Fase 1 \newline \small semanas 1 a 5} &
			Documentación sobre caracterización y medición de figura de ruido en dispositivos de radio frecuencia. & 100 \% \\
			& Documentación sobre equipos que integran el sistema de medición de figura de ruido (SMFR). & 100 \% \\
			& Elaboración de informe técnico descriptivo del SMFR. & 80 \% \\
			& Investigación y recopilación del software asociado al SMFR. & 100 \% \\
			\bottomrule		
		\end{tabularx}
		\caption{Actividades ejecutadas durante la fase 1}
	\end{table}

	\begin{table}[h!]
		\begin{tabularx}{\textwidth}{p{0.15\textwidth}p{0.6\textwidth}c}
			\toprule
			{Fase} & 
			{Actividad} & 
			{\% culminación} \\
			\midrule
			\multirow{5}{0.15\textwidth}{Fase 2 \newline \small semanas 6 a 17} &
			Investigación y selección de componentes electrónicos y mecánicos. & 100 \% \\
			& Elaboración de concepto de diseño de hardware para el prototipo. & 100 \% \\
			& Elaboración de esquemáticos. & 75 \% \\
			& Routeo de tarjetas de circuito impreso. & 25 \% \\	
			& Elaboración de una tarjeta prototipo, con fines de pruebas de firmware. & 100 \% \\
			\bottomrule			
		\end{tabularx}
		\caption{Actividades ejecutadas durante la fase 2 (relativas al hardware)}
	\end{table}

	\begin{table}[h!]
		\begin{tabularx}{\textwidth}{p{0.15\textwidth}p{0.6\textwidth}c}
			\toprule
			{Fase} & 
			{Actividad} & 
			{\% culminación} \\
			\midrule
			\multirow{4}{0.15\textwidth}{Fase 2 \newline \small semanas 6 a 17} &
			Investigación de librería de software para comunicaciones con el SMFR & 100 \% \\
			& Diseño y codificación de librería de comunicaciones alternativa & 80 \% \\
			& Diseño de aplicación para gestión del SMFR & 50 \% \\
			& Codificación de ésta aplicación & 50 \% \\	
			\bottomrule	
		\end{tabularx}
		\caption{Actividades ejecutadas durante la fase 2 (relativas al software)}
	\end{table}	

	\newpage	

	Sin más a que hacer referencia y agradecido por su atención, se despide de Ud.	
	
	\begin{flushright}
		Atentamente, 			
	
		\vspace{2cm}
		%\vfill
		\begin{singlespace}
			\large
			Jose Arias \\
			{
				\small
				C.I. 14.666.744 \\
				correo@josearias.com.ve \\			
			}	
		\end{singlespace}
	
	\end{flushright}


	\vfill
	%\vspace{1.9cm}

	%\hfill
	\begin{minipage}{0.2\textwidth}
		\centering 
		\large
		MSc. Pedro Ruiz \\
		\small
		Tutor
	\end{minipage}%
	\hfill 
	\begin{minipage}{0.4\textwidth}
		\centering \large
		MSc. Alejandro G. González E.	\\	
		\small
		Prof. Guía
	\end{minipage}
	\\
	%\hfill 
	\vfill

\end{document}