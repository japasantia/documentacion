% Para documento texto corto
\documentclass[paper=letter,oneside,fontsize=10pt]{article}
%\documentclass[paper=letter,oneside,fontsize=11pt, parskip=full]{scrartcl}
%\documentclass{amsart}
%\documentclass[paper=letter,oneside,fontsize=12pt]{scrartcl}

% Establece dimensiones de los margenes
% \usepackage[inner=1.5cm,outer=3cm,top=2cm,bottom=4cm,
% bindingoffset=5mm]{geometry}
\usepackage[left=3cm,right=3cm,top=3cm,bottom=3cm,
bindingoffset=0cm, footskip=0.0cm, headheight=2cm]{geometry}

% Para espacio entre lineas y parrafos
\usepackage{setspace}
\renewcommand{\baselinestretch}{1.1} 
\setlength{\parskip}{0.0em}

% Mejor jsutificacion, tipografia alta calidad.
\usepackage{microtype}

% Carga babel, idioma ingles
\usepackage[english, spanish]{babel}

% Elimina sangrias y aumenta espacio entre parrafos
\usepackage{parskip}

% Permite ingresar caracteres acentuados y especiales 
% sin necesidad de emplear comando
% utf8 codificacion de entrada Unicode (mas simbolos que ASCII)
\usepackage[utf8]{inputenc}

% T1 encoding for European, English, American text
\usepackage[T1]{fontenc}
% Fuente escalable
\usepackage{lmodern}

% Agrega comandos extra al comando tabular
% \toprule, \midrule, \bottomrule
\usepackage{booktabs}
% Tablas con ancho establecido por usuario
\usepackage{tabularx}
% Para posicionamiento preciso de tablas dentro del texto

\usepackage[flushleft]{threeparttable}

% Establece espacio entre lineas
%\usepackage[onehalfspacing]{setspace}

%Para resizebox
\usepackage{graphicx}

% Elimina numeros de pagina del docuento
\pagenumbering{gobble}

%Uso de colores
\usepackage{xcolor}
% Permite controlar colores de tablas
\usepackage{xcolor, colortbl}

% Definicion colores tabla cronograma
\definecolor{colorta}{rgb}{0.3569,0.608,0.8353}
\definecolor{colortb}{rgb}{0.4392,0.678,0.2784}
\definecolor{colortc}{rgb}{1.0000,0.361,0.0000}
\definecolor{colortd}{rgb}{0.1804,0.455,0.7098}
\definecolor{colorte}{rgb}{0.9294,0.490,0.1922}
\definecolor{colortf}{rgb}{0.2667,0.3290,0.4157}
\definecolor{colortg}{rgb}{0.3290,0.2667,0.4157}
\definecolor{colortj}{rgb}{0.2667,0.4157,0.329}

\newcommand{\TA}{\cellcolor{colorta}}
\newcommand{\TB}{\cellcolor{colortb}}
\newcommand{\TC}{\cellcolor{colortc}}
\newcommand{\TD}{\cellcolor{colortd}}
\newcommand{\TE}{\cellcolor{colorte}}
\newcommand{\TF}{\cellcolor{colortf}}
\newcommand{\TG}{\cellcolor{colortg}}
\newcommand{\TJ}{\cellcolor{colortj}}

\definecolor{colorfa}{rgb}{0.3569,0.608,0.8353}
\definecolor{colorfb}{rgb}{0.4392,0.678,0.2784}
\definecolor{colorfc}{rgb}{1.0000,0.361,0.0000}
\definecolor{colorsem}{rgb}{0.1804,0.455,0.7098}
\definecolor{colorfd}{rgb}{0.9294,0.490,0.1922}
\definecolor{colorfe}{rgb}{0.2667,0.329,0.4157}

\newcommand{\fa}{\cellcolor{colorfa}}
\newcommand{\fb}{\cellcolor{colorfb}}
\newcommand{\fc}{\cellcolor{colorfc}}
\newcommand{\sem}{\cellcolor{colorsem}}
\newcommand{\fd}{\cellcolor{colorfd}}
\newcommand{\fe}{\cellcolor{colorfe}}

% Elimina numeros de pagina del docuento
\pagenumbering{gobble}

\begin{document}
	
	\section{Anteproyecto inicial}
	
	A continuación se presentan el título, objetivo general, objetivos específicos y el cronograma de actividades presentes en el anteproyecto inicial.
	
	\subsection{Título}
	
	\begin{minipage}{0.95\textwidth}
		\centering
		DISEÑO DE UN EQUIPO ELECTRÓNICO CONTROLADOR DE INTERRUPTORES Y ATENUADORES EMPLEADO EN LA MEDICIÓN DE LA FIGURA DE RUIDO EN DISPOSITIVOS DE RADIO FRECUENCIA
	\end{minipage}

	\subsection{Objetivo general}

	\begin{minipage}{0.95\textwidth}
		Diseñar un equipo electrónico que permita emular las características funcionales de un controlador electrónico de interruptores y atenuadores.
	\end{minipage}

	\subsection{Objetivos específicos}

	\begin{enumerate}
		\item Elaborar un informe técnico a partir de un estudio del funcionamiento de los dispositivos presentes en el sistema de medición de figura de ruido (SMFR) de la fundación CENDIT.
		
		\item Diseñar un equipo electrónico que permita replicar las características funcionales de	una unidad de control para atenuadores e interruptores de la serie 11713 de KeySight.
		
		\item Implementar el diseño como dispositivo físico.
		
		\item Integrar el diseño físico en el banco de medición de figura de ruido presente en el laboratorio del CENDIT.
		
		\item Generar un manual de usuario para el dispositivo diseñado.
	\end{enumerate}	

	\subsection{Cronograma de actividades}
	
	\begin{threeparttable}[!h]
		\centering
		\arrayrulecolor{gray}
		\setlength{\extrarowheight}{4pt}		
		\resizebox{\textwidth}{!}{
			\begin{tabular}{|c|l|l|l|l|l|l|l|l|l|l|l|l|l|l|l|l|l|l|l|l|l|l|l|l|l|l|l|l|}
				\hline 			
				\textbf{Semanas} & 1 & 2 & 3 & 4 & 5 & 6 & 7 & 8 & 9 & 10 & 11 & 12 & 13 & 14 & 15 & 16 & 17 & 18 & 19 & 20 & 21 & 22 & 23 & 24 & 25 & 26 & 27 & 28 \\
				\hline
				\textbf{Preparación e investigación}
				& \fa & \fa & \fa & \fa & \fa & \fa & & & & & & & & & & & & & & & & & & & & & & \\			
				\hline			
				\textbf{Diseño de hardware} & & & & & & & \fb & \fb & \fb & \fb & \fb & \fb & \fb & \fb & \fb & \fb & \fb & & & & & & & & & & & \\
				\hline
				\textbf{Implementación del hardware} & & & & & & & & & & & & & & & & & & \fc & \fc & \fc & \fc & \fc & & & & & & \\	
				\hline		
				\textbf{Seminario} & & & & & & & & & & & & & & \sem & & & & & & & & & & & & & & \\
				\hline
				\textbf{Preparación de manuales} & & & & & & & & & & & & & & & & & & & & & & & \fd & \fd & \fd & \fd &  & \\
				\hline
				\textbf{Preparación de documentos } & & & & & & & & & & & & & & & & & & & & & & & & & & & \fe & \fe \\
				\hline	
			\end{tabular}
		}
		\begin{tablenotes}
			\item Fecha de inicio: 6 de Marzo de 2017. 
			\item Jornada de 8 horas diarias, lunes a viernes, de 8:00 AM a 12:00 M y de 1:30 PM a 4:30 PM.
		\end{tablenotes}
		\caption{Cronograma de actividades inicial}			
		\label{Tab:CronogramaActividadesInicial}
	\end{threeparttable}

	\newpage
	
	\section{Propuesta de cambio}
	
	Seguidamente se indican los cambios que se proponen realizar para el proyecto. El título y el objetivo general se preservan sin cambios.
	
	\subsection{Título}
	
	\begin{minipage}{0.95\textwidth}
		\centering
		DISEÑO DE UN EQUIPO ELECTRÓNICO CONTROLADOR DE INTERRUPTORES Y ATENUADORES EMPLEADO EN LA MEDICIÓN DE LA FIGURA DE RUIDO EN DISPOSITIVOS DE RADIO FRECUENCIA
	\end{minipage}
	
	\subsection{Objetivo general}
	
	\begin{minipage}{0.95\textwidth}
		Diseñar un equipo electrónico que permita emular las características funcionales de un controlador electrónico de interruptores y atenuadores.
	\end{minipage}
	
	\subsection{Objetivos específicos}	
	
	\begin{enumerate}
		\item Realizar una investigación documental sobre caracterización de dispositivos de radio frecuencia y la medición de figura de ruido en éstos.
		
		\item Recopilar la documentación y software asociado al sistema de medición de figura de ruido (SFMR).
		%\item Recopilar la documentación y software asociado al sistema de medición de figura de ruido (SMFR).		
		
		\item Codificar una librería de software para intercambio de datos entre PC y el SMFR.
		
		\item Diseñar y codificar el firmware para dispositivo.
		
		\item Diseñar y construir las tarjetas electrónicas PCB para cada uno de los módulos del equipo: expansor de puertos, fuente de alimentación y tarjeta madre.
		
		\item Desarrollar una aplicación de software para gestión de la medición de figura de ruido con el SMFR.
		
		\item Generar manuales de usuario para el equipo y para la aplicación.
	\end{enumerate}	
	
	\subsection{Cronograma de actividades}	
	
		\begin{center}	
			\begin{table}[h!]
				\arrayrulecolor{gray}
				\setlength{\extrarowheight}{4pt}	
				\resizebox{\textwidth}{!}{				
				\begin{tabular}{|c|c|l|l|l|l|l|l|l|l|l|l|l|l|l|l|l|}
					\hline 		
					\textbf{Item} &	
					\begin{tabular}{c}
						{\raggedright \textbf{Semanas}} \\
						\textbf{Tareas}
					\end{tabular}
					& 1 & 2 & 3 & 4 & 5 & 6 & 7 & 8 & 9 & 10 & 11 & 12 & 13 & 14 & 15\\
					\hline
					1 & Expansor de puertos Viking & \TA & \TA & \TA & \TA & & & & & & & & & & & \\
					\hline				
					2 & Firmware del dispositivo & & & & & & \TC & \TC & \TC & \TC & & & & & & \\ 
					\hline
					3 & Tarjeta madre & & & & & & & & \TD & \TD & \TD & \TD &  & & & \\
					\hline
					4 & Tarjeta de alimentación DC  &  & \TE & \TE & \TE & \TE & & & & & & & & & & \\
					\hline
					5 & Desarrollo de la aplicación SGMFR  & & & & & & &  & \TG & \TG & \TG & \TG & \TG & \TG & \TG & \TG \\
					\hline
					6 & Libro de TEG & & & & & & & & &  & \TJ & \TJ & \TJ & \TJ & \TJ & \TJ \\
					\hline
				\end{tabular}	
				}	
				\caption{Cronograma de actividades pendientes por ejecutar  en el TEG}			
				\label{Tab:CronogramaActividadesInicial}
			\end{table}			
		\end{center}	
	
\end{document}