% This file was converted to LaTeX by Writer2LaTeX ver. 1.4
% see http://writer2latex.sourceforge.net for more info
\documentclass[paper=letter,oneside,fontsize=10pt,parskip=full]{article}
\usepackage[left=3cm,right=3cm,top=3cm,bottom=2cm,
bindingoffset=5mm]{geometry}
\usepackage[utf8]{inputenc}
\usepackage[T1]{fontenc}
\usepackage[english,spanish]{babel}
\usepackage{amsmath}
\usepackage{amssymb,amsfonts,textcomp}
\usepackage{array}
\usepackage{supertabular}
\usepackage{hhline}
\usepackage[pdftex]{graphicx}
\makeatletter
\newcommand\arraybslash{\let\\\@arraycr}
\makeatother
\setlength\tabcolsep{1mm}
\renewcommand\arraystretch{1.3}
\newcounter{Table}
\renewcommand\theTable{\arabic{Table}}
\newcounter{Drawing}
\renewcommand\theDrawing{\arabic{Drawing}}
\title{}
\author{}
\date{2017-04-30}
\begin{document}
\clearpage\subsubsection[\-1 Objeto]{\-1 Objeto}
\subsubsection{2 Alcance}
\subsubsection[3 Informe descriptivo del banco para pruebas con fuente de ruido]{3 Informe descriptivo del banco para
pruebas con fuente de ruido}
\subsubsection[3.1 Introducción]{3.1 Introducción}
\subsection{3.2 Estado del banco para pruebas con fuente de ruido en el CENDIT}
\subsection{}
\subsection{4 Visión general – esquema principal – esquema conceptual conexión}
\subsection{4 Descripción general}
\subsection{4.1 Hardware}
\subsection{4.2 Sofware}
\subsection{}
\subsection{4 Descripción general}
\subsection{4.1 Idea general de medición de ruido factor Y}
\subsection{4.1 Interfaces físicas}
\subsection{4.1.1 Buses GPIB, LAN, USB}
\subsection{4.2 Interfaces de software}
\subsection{4.2.1 GPIB, LAN, USB}
\subsection{4.3 Idea general de instrumentación remota}
\subsection{}
\subsection{}
\subsection{3.1 Descripción de equipos}
\subsection{Caracteristicas técnicas, especificaciones y limite más importantes.}
\subsection{3.1 Descripción hardware}
\subsection{3.1.2 Analizador de figura de ruido N8975A}
\subsection{3.1.3 Controlador de interruptores y atenuadores}
\subsection{3.1.4 Conjunto de pruebas con fuente de ruido N2002}
\subsection{3.1.5 Fuentes de ruido inteligentes serie N4000}
\subsection{3.1.6 Puente de red LAN a bus GPIB E5810}
\subsection[3.1.7 Adaptador bus USB a bus GPIB]{3.1.7 Adaptador bus USB a bus GPIB}
\subsection{}
\subsection{4.1 Descripción de software}
\subsection{5 Ruido eléctrico}
\subsection{}
\subsection{5.1 Ruido}
\subsection{5.2 Figura de ruido}
\subsection{5.3 Medición de figura de ruido: Factor Y}
\subsection{5.3 Incertidumbre en cálculos de ruido}
\subsection{5.4 Ganancia en redes de dos puertos}
\subsection{5.4.1 Ganancia de potencia}
\subsection{5.4.2 Ganancia de potencia de transductor}
\subsection{5.4.3 Ganancia de potencia disponible}
\subsection{5.5 Cálculos de ruido en redes con impedancias desacopladas}
\subsection{}
\subsection{6 Medición de figura de ruido con el banco para pruebas con fuentes de ruido Agilent}
\subsection{6.1 Método manual (acceso local)}
\subsection{6.2 Método automatizado (acceso remoto)}
\subsection{6.3 Automatización de medición de figura de ruido}
\subsection{}
\subsection{7 Referencia bibliográficas}
\section{}
\clearpage\section{Informe descriptivo del banco para pruebas con fuente de ruido}
\subsection{Introducción}
La presente obra es un informe descriptivo del banco para medición de figura de ruido presente en la Fundación CENDIT.
Este sistema esta constituido casi en su totalidad por equipos de Agilent Technologies, es descrito en una nota de
aplicación como un sistema destinado a pruebas con fuentes de ruido, entre ellas, su calibración.

\subsection{Objetivo }
Por medio de una descripción general del sistema y una descripción de carácter técnico de cada uno de sus componentes,
se pretende lograr un visión general acerca del uso y situación actual del banco de medición de figura de ruido dentro
del CENDIT.

\subsection{Descripción general}
En un resumen técnico de la empresa Agilent Technologies [1] se propone los equipos mostrados en la figura 1 como un
sistema para realizar pruebas con fuentes de ruido, entre ellas su calibración, para aplicaciones de radio frecuencia
(RF) y microondas (UW). De acuerdo a esta nota técnica, el sistema planteado estaría compuesto por tres equipos
fundamentales de Agilent Technologies, asociados con equipos periféricos de interconexión y apoyo.

\begin{itemize}
\item Un analizador de figura de ruido, modelo N8975A.
\item Un controlador de interruptores y atenuadores, modelo 11713A.
\item Un banco de atenuadores y aisladores, modelo N2002A.


\bigskip
\end{itemize}


\begin{figure}
\centering
\begin{minipage}{16.009cm}


\includegraphics{Sistemamedicinderuido-img001.png}Figura 1

Banco para pruebas con fuentes de ruido, propuesto por Agilent Technologies [1]
\end{minipage}
\end{figure}
El sistema incluye además fuentes de ruido, cables de interconexión y adaptadores coaxiales, algunos de los cuales se
aprecian en la figura 1.

Si la caracterización de fuentes de ruido bien puede realizarse con un analizador de figura de ruido, como el equipo
N8975A de la figura 1, la empresa Agilent sugiere emplear dos dispositivos adicionales, si se desea disponer de la
instrumentación adecuada para calibración de fuentes de ruido con elevada exactitud y trazabilidad. 

De acuerdo a Agilent, un factor que degrada la exactitud y aumenta la incertidumbre en las mediciones de parámetros de
ruido, es la interacción que existe a la salida de la NS y la entrada de señal del NFA, debida a ligeros desacoples de
impedancia. Se emplearían entonces un banco de aisladores y atenuadores, el dispositivo Agilent N2002, interpuesto en
el camino de señal, entre la salida de la fuente de ruido y la entrada del NFA, con el objeto de disminuir la
interacción entre la NS y NFA para asi lograr medidas de elevada exactitud y baja incertidumbre.

Como dispositivo N2002 no presenta interfaz de usuario ni posee “inteligencia” interna que permita comandarlo, requiere
de un dispositivo controlador externo, un equipo de la serie 11713 de Agilent. Por medio de la interfaz que dispone el
11713, el usuario realiza la selección de la atenuación requerida en el banco de atenuadores N2002A.

Las mediciones de potencia de ruido en dispositivos RF y UW exige fuentes de ruido calibradas si se desea una elevada
precisión y exactitud, el error en dichas mediciones esta fuertemente ligado a la incertidumbre de la fuente de ruido
empleada. Por ello es de vital importancia su calibración, entendida esta como la verificación que se realiza en los
parámetros de la fuente de ruido para asegurar que la misma se encuentra dentro de los valores nominales establecidos
por la calibración de fabricante. 

Por medio del sistema propuesto por Agilent, se puede calibrar las fuentes de ruido “en casa”, sin depender de
laboratorios de calibración externos.

Uno de los pasos en el proceso de calibración de fuentes de ruido es la medición de la razón de ruido en exceso (ENR,
Excess Noise Ratio) de la fuente en cuestión. Si bien este paso de la calibración puede realizarse empleando únicamente
un analizador de figura de ruido, la motivación de Agilent al presentar este sistema es que utilizando los dispositivos
N2002A y 11713A en conjunto con el analizador de figura de ruido se pueden lograr mediciones de ruido con mayor
exactitud. 

Sin embargo, el sistema no esta limitado únicamente a realizar calibración de fuentes de ruido, permite además realizar
cualquier tipo de medición relacionada con ruido en RF y UW, como mediciones de potencia de ruido, \ figura de ruido,
temperatura equivalente de ruido, razón de ruido en exceso y mediciones de ganancia de potencia.

\subsection{3.2 Estado del banco para pruebas con fuente de ruido en el CENDIT}
El banco para pruebas con fuentes de ruido dentro del CENDIT se encuentra incompleto, ya falta uno de los tres equipos
integrantes de este sistema. El CENDIT dispone del analizador de figura de ruido N8975A y del banco de atenuadores y
aisladores N2002, pero hasta la fecha no ha logrado la adquisición del 11713A, el modelo antiguo producido por Agilent
para la unidad controlador de interruptores y atenuadores, o de los modelos más recientes para este equipo, el 11713B o
11713C, actualmente fabricados por Keysight Technologies.

Dificultades en la procura de equipos desde el exterior, han impedido al CENDIT adquirir un cualquier equipo de la serie
11713. El CENDIT se ha propuesto diseñar y desarrollar una replica de este equipo, funcionalmente equivalente a los
equipos de la serie 11713, en sus instalaciones. Ha delegado esta tarea como un tema de tesis en el pasante Br. Jose
Arias, autor del presente informe.

A continuación se da un inventario de equipos que cuenta el CENDIT para implementar el banco para pruebas con fuentes de
ruido.

\subsection{3.2.1 Equipos}
\begin{center}
\tablefirsthead{}
\tablehead{}
\tabletail{}
\tablelasttail{}
\begin{supertabular}{|m{2.34cm}|m{6.573cm}|m{2.34cm}|m{1.917cm}|}
\hline
\centering Modelo &
\centering Descripción &
\centering Fabricante &
\centering\arraybslash Cantidad\\\hline
\centering N8975A &
\centering Analizador de figura de ruido &
\centering Agilent &
\centering\arraybslash 1\\\hline
\centering N2002 &
\centering Equipo para pruebas con fuente de ruido &
\centering Agilent &
\centering\arraybslash 1\\\hline
\centering E5810 &
\centering Puente red LAN a bus GPIB / RS232 &
\centering Agilent &
\centering\arraybslash 1\\\hline
\centering 82357B &
\centering Adaptador bus USB a bus GPIB  &
\centering Agilent &
\centering\arraybslash 2\\\hline
\end{supertabular}
\end{center}
\subsection{3.2.2 Fuentes de ruido}
\begin{flushleft}
\tablefirsthead{}
\tablehead{}
\tabletail{}
\tablelasttail{}
\begin{supertabular}{|m{2.34cm}|m{6.997cm}|m{2.34cm}|m{2.34cm}|}
\hline
\centering Modelo &
\centering Descripción &
\centering Fabricante &
\centering\arraybslash Cantidad\\\hline
\centering N4000A &
\centering Fuente de ruido SNS ENR nominal de 6 dB &
\centering Agilent &
\centering\arraybslash 1\\\hline
\centering N4001A &
\centering Fuente de ruido SNS ENR nominal de 15 dB &
\centering Agilent &
\centering\arraybslash 1\\\hline
\centering N4002A &
\centering Fuente de ruido SNS ENR nominal de 14 dB &
\centering Agilent &
\centering\arraybslash 1\\\hline
\end{supertabular}
\end{flushleft}

\bigskip

\subsection{Cables}
\begin{flushleft}
\tablefirsthead{}
\tablehead{}
\tabletail{}
\tablelasttail{}
\begin{supertabular}{|m{2.34cm}|m{6.997cm}|m{2.34cm}|m{2.34cm}|}
\hline
\centering Modelo &
\centering Descripción &
\centering Fabricante &
\centering\arraybslash Cantidad\\\hline
\centering 11730A &
Cable para fuente de ruido SNS &
\centering Agilent &
\centering\arraybslash 3\\\hline
\centering 10833A &
Cable conexión bus GPIB, 1 metro &
\centering Agilent &
\centering\arraybslash 2\\\hline
\centering 10833B &
Cable conexión bus GPIB, 2 metros &
\centering Agilent &
\centering\arraybslash 2\\\hline
\centering 11500E &
Cable coaxial de 3.5 mm – (m-m) &
~
 &
\centering\arraybslash 1\\\hline
\end{supertabular}
\end{flushleft}
\subsection[Adaptadores coaxiales]{Adaptadores coaxiales}
\begin{flushleft}
\tablefirsthead{}
\tablehead{}
\tabletail{}
\tablelasttail{}
\begin{supertabular}{|m{2.34cm}|m{6.997cm}|m{2.34cm}|m{2.329cm}|}
\hline
\centering Modelo &
\centering Descripción &
\centering Fabricante &
\centering\arraybslash Cantidad\\\hline
\centering 1250-1744 &
\centering Adaptador coaxial Tipo N (m) a 3.5 mm (f)  &
\centering Agilent &
\centering\arraybslash 1\\\hline
\centering 1250-1745 &
\centering Adaptador coaxial Tipo N (f) a 3.5 mm (f)  &
\centering Agilent &
\centering\arraybslash 1\\\hline
\centering 1250-1750 &
\centering Adaptador coaxial Tipo N (f) a 3.5 mm (m)  &
\centering Agilent &
\centering\arraybslash 1\\\hline
\centering 83059A &
\centering Adaptador coaxial de 3.5 mm (m) a (m) &
\centering Agilent &
\centering\arraybslash 1\\\hline
\centering 83059B &
\centering Adaptador coaxial de 3.5 mm (f) a (f) &
\centering Agilent &
\centering\arraybslash 1\\\hline
\end{supertabular}
\end{flushleft}
\subsection{Accesorios}
\begin{flushleft}
\tablefirsthead{}
\tablehead{}
\tabletail{}
\tablelasttail{}
\begin{supertabular}{|m{2.34cm}|m{6.997cm}|m{2.34cm}|m{2.34cm}|}
\hline
\centering Modelo &
\centering Descripción &
\centering Fabricante &
\centering\arraybslash Cantidad\\\hline
\centering E5810-100 &
\centering Kit para montaje en rack del E5810 &
\centering Agilent &
\centering\arraybslash 1\\\hline
\end{supertabular}
\end{flushleft}

\bigskip

\section{}
\clearpage\section{Sistema de medición de figura de ruido}
En la figura 2 se muestra un diagrama conceptual para el sistema de medición de figura de ruido, objetivo de
implementación dentro del CENDIT, cumplirá dos tareas principales serán la medición de figura de ruido en dispositivos
de RF y UW y \ la calibración de fuentes de ruido.

Se aprecia en la figura 2 que el sistema es el resultado de la integración de equipo de hardware, interconectado por un
conjunto de vías de señal y buses para transmisión de datos más una capa de aplicaciones de software. El sistema
permitirá la medición de parámetros de relativos al ruido como lo son la potencia de ruido, temperatura efectiva y
figura de ruido, su presentación gráfica, almacenamiento y distribución en red.

Es un sistema la resultante de la integración de equipo hardware y aplicaciones de software, la tarea esencial es la
medición de la figura de ruido en dispositivos \ Permite la medición de parámetros de ruido, su presentación, análisis
almacenamiento y distribución en red. No solo de forma local sino de manera remota, empleando sus capacidades de
conexión a buses

El núcleo del sistema lo representa el analizador de figura de ruido (NFA) N8975 y las fuentes de ruido inteligentes
(Smart Noise Source, SNS) de la serie N4000A, ambos productos de Agilent. Las capacidades del NFA definen la
funcionalidad de todo sistema: es el encargado de la medición, presentación, almacenamiento y distribución de datos
relativos a la figura de ruido. 

Las fuentes de ruido inteligentes constituyen una fuente de señal de referencia de \ RF y de UW. Estas generan una señal
de ruido blanco, con niveles de potencia conocidos. 

Empleando el NFA en conjunto con una SNS es posible medir la figura de ruido en un dispositivo o verificar la
calibración de una fuente de ruido. Agilent propone anteponer a la entrada del NFA un banco de atenuadores y
aisladores, el dispositivo N2002A en la figura 2, como un mecanismo para aumentar la exactitud y reducir la
incertidumbre en las mediciones.

El N2002A es un dispositivo que carece interfaz de usuario, es necesario emplear un equipo de la serie 11713 –
actualmente producidos por Keysight Technologies –, conocido como unidad controladora de interruptores y atenuadores
(figura 2). 

El equipo 11713B (figura 2) presenta una interfaz sobre la cual el usuario selecciona el rango de frecuencias sobre el
cual el N2002A debe dejar pasar en el camino de señal. El 11713B traduce las pulsaciones del usuario en los botones
frontales a señales de control apropiadas que permiten establecer las frecuencias de paso en el N2002A. Estas señales
se transmite por medio de un cables especiales conectados en los puertos ubicados en los paneles traseros de ambos
equipos.


\bigskip



\begin{figure}
\centering
\begin{minipage}{18.516cm}
[Warning: Draw object ignored]Figura 2

Sistema para medición de figura de ruido
\end{minipage}
\end{figure}
\subsection[Analizador de figura de ruido N8975A (NFA Series Noise Figure Analyzer)]{Analizador de figura de ruido
N8975A (NFA Series Noise Figure Analyzer)}
Representa el equipo fundamental en el sistema de la figura 2, es esencialmente un dispositivo que mide de potencia de
ruido en RF y UW generada por elementos pasivos o activos. A partir de la medida de potencia, este equipo puede
calcular el valor de la figura de ruido de elemento, así como también su temperatura efectiva, ganancia y presentar el
resultado en pantalla. 

El NFA mide la potencia total de ruido que genera un dispositivo bajo prueba (DUT por sus siglas en inglés) cuando en su
entrada se inyecta una señal de ruido de referencia. Esta señal es producida por una fuente de ruido estándar, sus
características de ruido deben ser conocidas con elevada exactitud. En el sistema CENDIT, se emplean las fuentes de
ruido de la serie Agilent N4000A. \ La fuentes de ruido N4000A son conocidas como fuentes de ruido inteligentes, ya que
estas almacenan en una memoria no volátil interna, una tabla que caracteriza su potencia de ruido en función de la
frecuencia, conocida como tabla de ENR. Además, cuentan con un sensor de temperatura integrado.

A través de un cable que conecta el NFA con el puerto de control de la NS, el NFA puede cargar las tablas de ENR de las
fuentes de ruido. De la misma forma, puede leer la temperatura de la fuente de ruido.

[Warning: Draw object ignored]

El N8975 emplea la técnica del factor Y para la medición de ruido sobre un \ DUT. Esta técnica consiste básicamente en
inyectar en la entrada del DUT dos niveles distintos de potencia de ruido, el primero \ \ sensiblemente mayor que el
segundo, para luego medir la potencia de ruido que a la salida del DUT existen para cada uno de estos niveles.
\ Calculado un cociente entre el valor mayor y el valor menor de la medición, esto es el factor Y, y con la tabla de
ENR de la fuente de ruido se puede conseguir la figura del ruido del DUT.

Aparte de la figura de ruido, puede realizar las siguientes mediciones:

\begin{itemize}
\item Potencia de ruido, fría PCOLD y caliente PHOT
\item Temperatura equivalente de ruido.
\item Factor Y.
\item Ganancia.
\end{itemize}
El NFA cuenta con las siguientes interfaces

\begin{itemize}
\item Una interfaz de usuario. 
\item Una interfaz de señal (RF y UW)
\item Una interfaz de datos.
\end{itemize}
La interfaz de usuario se divide a su vez en dos interfaces: interfaz local e interfaz remota. La interfaz local permite
al usuario operar el NFA en sitio, la constituye el panel frontal del equipo el cual cuenta con una pantalla LCD y un
conjunto de botones agrupados en bloques de acuerdo a su funcionalidad y un teclado numérico.

El monitor LCD es a color, de 17 cm, en el se presentan los datos ya se en formato de gráfico, tabla o modo de medidor.
Puede presentar dos gráficas simultaneas, en estas se pueden agregar cuatro marcadores o cursores y dos lineas limites.

\ Puerto paralelo, 25 pines D-sub, dedicado a impresora. Salida VGA conector 15 pines, mini D-sub, hembra.

El NFA dispone de la capacidad para ser operado de forma remota. En su panel trasero cuenta con un conector IEEE-488
para bus GPIB y un conector RS-232 D-sub de 9 pines. El conector IEEE-4888 permite conectar el equipo a un bus GPIB y
establecer una red de instrumentos de medición. Es posible además conectar el puerto GPIN del NFA a un computador, por
medio de una tarjeta como las del tipo PCI-GPIB y cable de bus GPIB empleando o un adaptador USB-GPIB. A través de bus
GPIB y empleando aplicaciones de software se le envían comandos al equipo que permiten controlarlo y obtener datos de
mediciones.

Interfaz GPIB conector IEEE-488, bus serial RS-232, 9 pines D-sub, macho.

La interfaz de señal señal que dispone el equipo en su panel frontal consiste en 

\begin{itemize}
\item Un conector de entrada para señal RF / \ UF. \ 
\item Un conector de salida para el control, alimentación y datos de una fuente inteligente de ruido (SNS) de la serie
N4000 de Agilent 
\item Un puerto para alimentación y conmutación de una fuente de ruido tradicional serie 346 de Agilent. 
\end{itemize}
EL conector de entrada para señal de RF / UW es de tipo APC 3.5 (m), con \ impedancia nominal de 50 Ω, ESD sensible. La
potencia máxima admisible en la señal de entrada es de 10 dBm. La máxima protección de entrada ± 20V, +15dBm pico (o
promedio) en RF. El rango de frecuencia admisible por ele equipo es de 10 a 26.5 Ghx.

En el conector SNS se conecta una fuente de ruido SNS, a través de un cable multihilo cilíndrico. Por medio de este
puerto el equipo enviá y recibe los datos almacenados de ENR y temperatura medida por la NS además de la señal de
alimentación que permite conmutar la potencia de ruido generada por la SNS, \ las señales de alimentación y datos 


\bigskip

Interfaz eléctrica

Rango de frecuencia: 10 a 26,5 GHz

Ancho de banda de medición: 4 MHz, 2 MHz, 1 MHz, 400 kHz, 200 kHz, 100 kHz.

El desempeño depende del ENR de la fuente empleada-


\bigskip

\begin{center}
\tablefirsthead{}
\tablehead{}
\tabletail{}
\tablelasttail{}
\begin{supertabular}{|m{3.0809999cm}|m{3.284cm}|m{1.8429999cm}|m{1.8429999cm}|m{1.737cm}|}
\hline
\multicolumn{5}{|m{12.588cm}|}{\centering Figura de ruido N975A [AGI00]}\\\hline
\multicolumn{2}{|m{6.565cm}|}{\centering Rango de ENR para la fuente de ruido} &
\centering 4 – 7 dB &
\centering 12 -17 dB &
\centering\arraybslash 20 – 22 dB\\\hline
\centering 10 MHz a 3.0 GHz &
\centering Rango de medición &
\centering 0 a 20 dB &
\centering 0 – 30 dB &
\centering\arraybslash 0 a 35 dB\\\hline
 &
\centering Incertidumbre  &
\centering ± {\textless} 0.05 dB &
\centering ± {\textless} 0.05 dB &
\centering\arraybslash ± {\textless} 0.1 dB\\\hhline{~----}
\centering Mayor a 3.0 GHz &
\centering Rango de medición &
\centering 0 a 20 dB &
\centering 0 a 30 dB &
\centering\arraybslash 0 a 35 dB\\\hline
 &
\centering Incertidumbre &
\centering ± {\textless} 0.15 dB &
\centering ± {\textless} 0.15 dB &
\centering\arraybslash ± {\textless} 0.2 dB\\\hhline{~----}
\end{supertabular}
\end{center}

\bigskip

\begin{center}
\tablefirsthead{}
\tablehead{}
\tabletail{}
\tablelasttail{}
\begin{supertabular}{|m{3.01cm}|m{3.284cm}|m{1.384cm}m{1.599cm}m{1.7759999cm}|}
\hline
\multicolumn{5}{|m{11.853cm}|}{\centering Ganancia N8975A [AGI00}\\\hline
\multicolumn{2}{|m{6.4940004cm}|}{\centering Rango de ENR para la fuente de ruido} &
\multicolumn{1}{m{1.384cm}|}{\centering 4 – 7 dB} &
\multicolumn{1}{m{1.599cm}|}{\centering 12 -17 dB} &
\centering\arraybslash 20 – 22 dB\\\hline
\centering 10 MHz a 3.0 GHz &
\centering Rango de medición &
\multicolumn{3}{m{5.159cm}|}{\centering {}-20 a +40 dB}\\\hline
 &
\centering Incertidumbre  &
\multicolumn{3}{m{5.159cm}|}{\centering ± {\textless} 0.17}\\\hhline{~----}
\centering Mayor a 3.0 GHz &
\centering Rango de medición &
\multicolumn{3}{m{5.159cm}|}{\centering {}-20 a +40 dB}\\\hline
 &
\centering Incertidumbre &
\multicolumn{3}{m{5.159cm}|}{\centering ± {\textless} 0.17 dB}\\\hhline{~----}
\end{supertabular}
\end{center}

\bigskip

Ruido generado por el instrumento [AGI00]

\begin{center}
\tablefirsthead{}
\tablehead{}
\tabletail{}
\tablelasttail{}
\begin{supertabular}{|m{3.707cm}|m{4.854cm}|m{5.952cm}|}
\hline
\centering Frecuencia &
\centering Figura de ruido  &
\centering\arraybslash Figura de ruido en torno a 23 ± 3 °C\\\hline
\centering 10 MHz a {\textless} 500 MHz &
\centering {\textless} 4.9 dB + ( 0.0025 * f(MHz) ) &
\centering\arraybslash {\textless} 4.4 dB + ( 0.0025 * f(MHz) )\\\hline
\centering 500 MHz a {\textless} 2.3 GHz &
\centering {\textless} 4.9 dB + ( 0.00135 * f(MHz) ) &
\centering\arraybslash {\textless} 5.9 dB + ( 0.00135 * f(MHz) )\\\hline
\centering 2.3 GHz a 3.0 GHz &
\centering 4.9 dB + ( 0.0015 * f(MHz) ) &
\centering\arraybslash {\textless} 2.9 dB + ( 0.0015 * f(MHz) )\\\hline
\centering {\textgreater} 3.0 GHz a 13.2 GHz &
\centering {\textless} 12.0 dB &
\centering\arraybslash {\textless} 10.5 dB\\\hline
\centering {\textgreater} 13.2 GHz a 26.5 GHz &
\centering {\textless} 16.0 dB &
\centering\arraybslash {\textless} 12.5 dB\\\hline
\end{supertabular}
\end{center}
El NFA puede realizar mediciones en un rango de frecuencia o en una frecuencia particular. Si se trata de un rango o
lista de frecuencias, el dispositivo calcula la edición requerida de forma automáticas para cada valor de frecuencia de
interés. El usuario también puede configurar el NFA para ejecutar la medición en forma manual en cada punto de
frecuencia. 

El NFA puede presentar los resultados en forma de gráficas o tablas, con el valor de la medición en función de la
frecuencia. Puede almacenar y recuperar los datos de mediciones \ en su memoria interna.

Dispone este equipo en su panel trasero de puerto 

La capacidades del N8975A pueden entenderse mejor examinando el panel frontal del equipo. En la figura se muestra el
panel frontal, el cual esta dividido cuatro grandes grupos de acuerdo a la funcionalidad: \ MEASURE, CONTROL, SYSTEM,
DISPLAY.

MEASURE establece los parámetros que controlan la medición coo el rango de frecuencia, el ancho de banda y la cantidad
de puntos de medición.

CONTROL configura parámetros más avanzados de la medición como compensar las perdidas de elementos auxiliares y agregar
lineas limite.

SYSTEM configuración del sistema como la dirección GPIB del NFA, mostrar información de estado y controlar un oscilador
externo.

DISPLAY periten ajustar las características de la presentación de los datos que mide el instrumento, permite escoger que
parámetros mostrar y permite ajustar la escala de los gráficos.


\bigskip

\clearpage\subsection[]{[Warning: Draw object ignored]}
\clearpage\subsection[Interfaz de usuario. Resumen del panel frontal]{Interfaz de usuario. Resumen del panel frontal}
Para realizar mediciones en modo local el usuario debe interactuar con el panel frontal, este le brinda acceso a todas
las capacidades del NFA, describe por si misma todas las capacidades. Las teclas que típicamente tienen un mayor uso
cuando se desarrolla una medición tienen \ mayor ancho y se ubican cerca de la parte derecha del display, están
organizadas \ de tal forma que describen parte de la secuencia a seguir en la medición: (Frequency / Points, Averaging
/ Bandwidth, Calibrate, Scale and Format), 

Las teclas de mayor ancho ubicadas cerca de la parte derecha del display (Frequency / Points, Averaging / Bandwidth,
Calibrate, Scale and Format) son teclas que típicamente tienen mayor uso cuando se desarrolla una medición. Las teclas
de acción Calibrate, Full Screen, Restart, Save Trace and Print) invocan una acción.

El proceso de medición empleando el NFA básicamente el usuario debe seguir tres pasos: primero configuración del equipo,
ejecutar la auto calibración del equipo por último, la medición. El proceso de configuración consiste esencialmente en
cargar \ los datos de ENR de la fuente de ruido a utilizar, establecer los puntos de frecuencia o el rango de
frecuencia donde se desea realizar la medición, calibrar el ancho de banda y el promedio y establecer que parámetro se
presentará en pantalla para después calibrar el equipo.

[Warning: Draw object ignored]

[Warning: Draw object ignored][Warning: Draw object ignored]ENR permite introducir en el NFA \ todo lo relativo a los
datos de la razón de excedente de ruido. Al presionar la tecla ENR del panel MEASURE se despliega en pantalla el menú
de la figura 1. Se aprecia en la figura 1 que el NFA acepta los datos de ENR en formato tabular o en forma de valor
puntual. El formato tabular le indica al equipo los valores de ENR para cada frecuencia de interés. En cambio se puede
ingresar un único valor de ENR para ser utilizado en todo el rango de frecuencias de medición.

[Warning: Draw object ignored]El equipo puede se configurado para aceptar dos tablas distintas de ENR \ \ \ \ \ \ al
establecer la opción Common Table en Off. 

El equipo puede aceptar dos tablas con datos de ENR, una de ellas se empleara exclusivamente en el durante el proceso de
autocalibración (Cal Table) y la otra se usará para el proceso de medición ENR table.

[Warning: Draw object ignored][Warning: Draw object ignored]Cuando el modo de ENR esta establecido en Tabla, a través de
las opciones y se pueden ingresar o editar las tablas de ENR para medición y calibración. Al presionar la tecla
respectiva, en pantalla se despliega el editor de tablas de ENR (figura). 

[Warning: Draw object ignored]Cuando el modo ENR se configura en Spot, se ingresa un único valor de ENR con la opción de
menú el cual se utilizara para todas las frecuencias de medición.

[Warning: Draw object ignored][Warning: Draw object ignored][Warning: Draw object ignored][Warning: Draw object
ignored][Warning: Draw object ignored]A través del menú el usuario puede configurar el valor que empleará como
temperatura física que posee la fuente de ruido, indicada en el menú como Tcold \ o temperatura fría. Al presionar esta
opción, el usuario puede indicar si desea que el NFA cargue el valor de Tcold desde el sensor de temperatura de la
fuente de ruido inteligente (SNS Tcold = On) o si debe tomar el valor establecido por el usuario (User Tcold = On).
\ Cuando esta última opción esta activa, el usuario al presionar User Value y utilizando el teclado numérico puede
ingresar el valor de Tcold o puede cargar el valor de Tcold desde la SNS

[Warning: Draw object ignored][Warning: Draw object ignored][Warning: Draw object ignored]El menu ENR permite establecer
si el NFA empleará una fuente de ruido inteligente de la serie Agilent N4000 (SNS, serie N4000) o una fuente de ruido
normal, serie 346 de Agilent. \ El NFA tiene la capacidad de cargar los datos de ENR de forma automatica cuando detecta
la conexión de una fuente de ruido SNS, con pción \ Auto Load ENR establecida en On, 


\bigskip


\bigskip


\bigskip

[Warning: Draw object ignored][Warning: Draw object ignored][Warning: Draw object ignored][Warning: Draw object
ignored]Frequency / Points el NFA puede realizar mediciones sobre un rango de frecuencias, una lista de frecuencias o
en una frecuencia puntual. Por medio de la opción Freq Mode se puede establecer que el NFA realize la medicion en forma
de barrido sobre un rango de frecuencias , sobre una lista de frecuencias, o en una frecuencia puntual (Fixed) 

\subsection[Cuando se utiliza el modo de barrido, el rango de frecuencias se especifica ya sea por su frecuencia de
inicio y por su frecuencia final o bien por la frecuencia central del rango y su extensión. El numero de puntos \ que
posee el rango se introduce con ]{[Warning: Draw object ignored][Warning: Draw object ignored][Warning: Draw object
ignored][Warning: Draw object ignored][Warning: Draw object ignored]Cuando se utiliza el modo de barrido, el rango de
frecuencias se especifica ya sea por su frecuencia de inicio y por su frecuencia final o bien por la frecuencia central
del rango y su extensión. El numero de puntos \ que posee el rango se introduce con }
\subsection[Cuando se establece el NFA que realice las mediciones sobre una lista de frecuencias, al presionar la
opción, se muestra en pantalla el editor de lista de frecuencia sobre el cual el usuario ingresa los valores de
frecuencia por medio del teclado numérico.]{[Warning: Draw object ignored][Warning: Draw object ignored]Cuando se
establece el NFA que realice las mediciones sobre una lista de frecuencias, al presionar la opción, se muestra en
pantalla el editor de lista de frecuencia sobre el cual el usuario ingresa los valores de frecuencia por medio del
teclado numérico.}
\begin{figure}
\centering
\includegraphics{Sistemamedicinderuido-img002.jpg}
\end{figure}
\subsection[Cuando se desea medir en una frecuencia puntual, esta se ingresa por medio de la opción ]{[Warning: Draw
object ignored]Cuando se desea medir en una frecuencia puntual, esta se ingresa por medio de la opción }
[Warning: Draw object ignored][Warning: Draw object ignored][Warning: Draw object ignored]Averaging / Bandwidth El
promedio y el ancho de banda permite corregir los efectos del jitter del display en las mediciones. permite configurar
el promedio y el ancho de banda. Para cada frecuencia de medición, el NFA mide la potencia de ruido en un ancho de
banda centrado en torno a esta. El usuario puede elegir un valor para este ancho de banda de entre 6 opciones
disponibles. El dispositivo puede ademas tomar múltiples mediciones y luego tomar el promedio de estas como el valor
definitivo de la medición. Cuando se activa el promedio, el usuario puede especificar cuantas veces debe tomarse el
mismo.

Al reducir el ancho de banda de medición, se incrementan los efectos del jitter en el display, lo cual puede corregirse
incrementando el numero de promedios.

Si se emplea un ancho de banda menor de 100 kHz, la cantidad de promedios debería ser 40 veces mayor si se desea obtener
la exactitud que se obtendría con un ancho de banda de 4 MHz.

[Warning: Draw object ignored]Si el promedio esta activo, el usuario puede seleccionar el modo de promedio (Average
Mode) entre puntual (Point) \ o barrido (Sweep). Cuando se realiza el promedio puntual, el NFA calcula el promedio de
cada punto de frecuencia antes de avanzar al siguiente. Cuando el NFA ejecuta el promedio en barrido, el NFA realiza la
medición en un punto y avanza al siguiente. Cuando llega al ultimo punto, vuelve al primer punto toma una nueva
medición y la promedia con el valor del último barrido, acumulando promedios en el tiempo. Esta opción permite ver el
valor de la medicón en pantalla en tiempo real.

[Warning: Draw object ignored][Warning: Draw object ignored][Warning: Draw object ignored][Warning: Draw object
ignored][Warning: Draw object ignored][Warning: Draw object ignored][Warning: Draw object ignored] Scale: Permite
realizar ajustes relacionados a las escalas de los ejes en la presentación gráfica. La configuración de escala opera
sobre el gráfico actualmente activo, se indica en \ \ el cual en la figura es un gráfico de PHot. Al presionar
Autoscale la escala vertical de la gráfica activa se ajusta de modo automático para cubrir todo su rango de valores. Se
puede seleccionar la unidad de presentación entre dB y lineal (Linear). \ Para la gráfica actualmente activa se puede
ajustar los \ valores límites máximo y mínimo del eje vertical, así como también las unidades por división (Scale /
Div).


\bigskip


\bigskip


\bigskip


\bigskip


\bigskip


\bigskip


\bigskip


\bigskip


\bigskip


\bigskip


\bigskip


\bigskip


\bigskip


\bigskip

[Warning: Draw object ignored][Warning: Draw object ignored][Warning: Draw object ignored][Warning: Draw object
ignored][Warning: Draw object ignored][Warning: Draw object ignored][Warning: Draw object ignored]Format: Permite
seleccionar el formato de presentación de los datos de medición entre gráfico, tabla y valor puntual. Se \ puede
activar la presentación de doble trazo en una única gráfica (Combined = On) o cada traza en su respectiva gráfica
(Combined \ = Off), \ así como también activar la rejilla sobre los gráficos. 

[Warning: Draw object ignored][Warning: Draw object ignored]Pueden graficarse trazos almacenados en memoria del equipo,
con \ y


\bigskip


\bigskip


\bigskip


\bigskip


\bigskip

[Warning: Draw object ignored]Sweep:

\clearpage
\bigskip

[Warning: Draw object ignored][Warning: Draw object ignored][Warning: Draw object ignored][Warning: Draw object
ignored]File: El N8975 posee la capacidad de almacenar información en su memoria interna o en un disco floppy, en forma
de archivos. Al presionar la tecla File se presenta un menú con las funciones básicas sobre el sistema de archivos, Las
opciones de menú permiten \ cargar (Load) y guardar (Save) archivos. La opción de menú , File Manager, presenta el
administrador de archivos, por medio del cual se pueden copiar, renombrar y eliminar archivos, asi como también
formatear un diskette en la unidad floppy.

El N8975 puede almacenar o cargar archivos con tablas de ENR, estado del sistema, trazos limites, tablas de perdidas y
gráficos desde o hacia la pantalla. 

[Warning: Draw object ignored][Warning: Draw object ignored][Warning: Draw object ignored][Warning: Draw object
ignored][Warning: Draw object ignored]Loss Comp: permite realizar ajustes para compensar la atenuación y el ruido que
introducen los elementos presentes en el camino de señal, como cables y acopladores, ubicados antes y después del DUT.
En la pantalla de edición se pude introducir un valor fijo de atenuación en los campos respectivos. También pude
configurarse el equipo para aceptar o una tablas de valores de atenuación en función de la frecuencia en y luego
ingresar las tablas de perdidas antes del DUT y después del DUT para las perdidas de los elementos antes y después del
DUT. La pantalla cabia en este caso al modo de edición de tablas, donde se introduce un valor de frecuencia seguido de
su respectivo valor de perdidas.

[Warning: Draw object ignored]En \ se puede establecer la temperatura de ruido equivalente para los elementos antes o
después del DUT.


\bigskip

[Warning: Draw object ignored][Warning: Draw object ignored][Warning: Draw object ignored][Warning: Draw object
ignored][Warning: Draw object ignored][Warning: Draw object ignored][Warning: Draw object ignored]Result: permite
elegir el valor de la medición que se visualiza en el gráfico activo, como la figura de ruido, la ganancia, el factor
Y, la temperatura efectiva de ruido, la potencia de ruido “caliente” (fuente de ruido encendida) o la potencia de ruido
fría.

\clearpage
\bigskip

Teclas de acción

Al presionar se ejecuta un acción Calibrate, Full screen, Restart, Save Trace and Print.

Capacidades del N8975

Medición de potencia de ruido (PHOT, PCOLD~).

Medición de Temperatura efectiva.

Determinación del factor Y.

Determinación de la figura de ruido.

Medición de Ganancia.

Puede ejecutar medidas en dispositivos simples o en sistemas de conversión de frecuencias.

Mediciones básicas

Preparación para proceso básico de medición

Antes de realizar alguna medición de parámetros de ruido con el N8975, por lo general se deben ejecutar unos pasos
previos de configuración del equipo.

Para la medición de figura de ruido, el usuario debe ingresar ciertos datos al NFA antes de dar marcha con la medición.

\begin{itemize}
\item Ingresar los datos de la razón de ruido en exceso (ENR).
\item Establecer el rango de frecuencias o la frecuencia individual sobre las cueles se desea la medida.
\item Establecer el ancho de banda y configurar el promedio.
\item Calibrar el analizador.
\item Mostrar los resultados en pantalla.
\end{itemize}
Ingresar datos de ENR

[Warning: Draw object ignored]El NFA requiere los datos de ENR de las fuentes de ruido que se utilicen durante las
mediciones. El equipo emplea los valores de ENR en dos situaciones distintas: \ durante el proceso de auto-calibración
y durante la medición de figura de ruido. Cuando el equipo ejecuta la auto-calibración, éste emplea una tabla de
valores ENR de la fuente de ruido con la cual se lleva acaba este ajuste, por medio del cual el equipo elimina su
contribución de ruido en la medición. Durante la medición de figura de ruido, el equipo emplea los datos de ENR de la
fuente de ruido y mediciones directas de potencia de ruido para calcular y presentar en pantalla el valor F. 

El N8975 admite el ingreso de los datos ENR se ingresan al N8975 en forma de tabla o en forma de valor puntual (spot
value). \ Para el formato tabla, cada fila de esta consiste en un par de valores, frecuencia y ENR, esta tabla se
emplea para medición en múltiples frecuencias. Si el equipo necesita el valor de ENR en alguna frecuencia que no este
listado en esta tabla, simplemente calculará el valor que necesita por interpolación. \ Esto ocurre cuando la
configuración establecida por el usuario para la frecuencia máxima, frecuencia \ mínima \ y cantidad de puntos de
medición provocan que el rango de frecuencias sobre el cual medirá el NFA no concuerde con las frecuencias de la tabla
ENR. 

Debería medirse el DUT a las mismas frecuencias que establece la tabla de ENR de la fuente de ruido.

Cuando se ingresa un valor puntual de ENR, el equipo lo emplea para medición en una sola frecuencia \ es aplicado en
todo todo el rango de medición.

El equipo puede operar con \ dos tablas distintas para ENR: una tabla para ENR de calibración \ y una tabla para valores
de ENR de medición. La tabla ENR de calibración la emplea el equipo cuando ejecuta el proceso de auto-calibración. La
tabla ENR de medición la emplea el equipo \ durante la medición de figura de ruido. \ 

Puede configurarse el equipo para que utilice dos tablas de ENR distintas (calibración y medición) o para que utiliza
una única tabla de ENR, la tabla ENR de calibración, para la tarea de auto calibración y medición de figura de ruido.

La utilidad en emplear dos tablas de ENR esta en que se puede emplear una fuente de ruido para el proceso de
auto-calibración y otra fuente de ruido distinta para el proceso de medición. 

La fuentes de ruido normales, como las \ Agilent 346, el usuario debe ingresar \ de forma manual o por medio de un
diskette que le suministra el fabricante los datos de ENR, el fabricante suministra estos datos. La fuentes de ruido
inteligentes de Agilent (SNS) pueden cargar los datos de ENR de manera automática al NFA.

Los puntos de frecuencia a medir son determinados por entradas en la tabla de ENR?

Los datos de ENR pueden cargarse de cuatro formas distintas maneras:

\begin{itemize}
\item Se puede ingresar un único valor de THOT
\item Puede introducir datos de ENR por medio de un disco floppy, en donde previamente se haya almacenado la data de
Introduciéndolo en la ranura que dispone el NFA y utilizando el administrador de archivos. Se accede a las funciones e
control de archivos por medio del botón File (panel System).
\item Puede cargar los datos desde la memoria interna del NFA. 
\item Puede cargar los datos de forma remota, a través del bus GPIB. 
\item En caso de utilizar una fuente de ruido inteligente, como las Agilent serie N4000, el usuario puede elegir si
cargar de foma automática cuando se conecte una fuente de ruido.
\end{itemize}
Ingresar datos de temperatura (TCOLD)

En caso de que la temperatura en el recinto donde se efectúe la medición sea distinta a la temperatura por defecto
establecida en el equipo de 296.05K, se debe ingresar al equipo el valor de la temperatura (TCOLD). Cuando se emplean
las fuentes de ruido inteligentes SNS, serie Agilent N4000, este paso puede \ ya que estas fuentes incluyen un sensor
de temperatura interno, el NFA puede cargar automaticamente el valor de temperatura correcto al momento de efectuar
cada medición.

Establecer las frecuencias de medición

El usuario debe establecer el conjunto de frecuencias sobre las cuales se realizará la medida. El N8975 dispone de tres
opciones para la selección de frecuencias de medición: barrido (swep), lista (list) y fija (fixed).

Barrido (sweep): el rango de frecuencias de medición se obtiene de una frecuencia inicial, una frecuencia final y el
numero de medidas.

Lista (list): las frecuencias de medición se especifican ingresando en el N8975 una lista de valores de frecuencia.

Fija (fixed): el usuario establece que la medición se realizará en un único valor de frecuencia.

Establecer el ancho de banda y activación del promedio.

Se selecciona un valor para el \ ancho de banda, el cual determina \ en torno a cada valor de frecuencia sobre el cual
se integra la potencia de ruido. El usuario debe escogen un valor de ancho de una lista, las opciones para el ancho de
banda son \ 100 kHz, 200 kHz, 400 kHz, 1 MHz, 2Hz y 4MHz.

El usuario puede elegir si desea activar el promedio en las mediciones. Si se activa el proedio, puede establecer si es
un promedio puntual o promedio de barrido.

Al activar el promedio el jitter y se provee de mediciones más precisas cuanto más promedios se realice. Sin embargo, la
velocidad de medición se reduce.

Calibración

Cumplidos los pasos anteriores, se ejecuta la auto calibración del equipo N8975. La auto calibración le permite a este
equipo compensar la contribución de ruido introducida por el cableado y accesorios que se encuentren en el camino de
señal además del ruido generado por el propio N8975.



\begin{figure}
\centering
\begin{minipage}{8.063cm}
Conexión del sistema para ejecutar auto calibración.
\includegraphics{Sistemamedicinderuido-img003.png}\end{minipage}
\end{figure}
Se conecta el equipo con indica la figura. Los datos de ENR para la fuente de ruido empleada en la calibración deben
haberse introducido previamente en la tabla de calibración.

Mostrar resultados

El equipo puede presenta los resultados en forma de gráficos, tabla o valor textual. En cuanto a la presentación de
gráficos, el \ equipo puede presentar dos resultados en pantalla de manera simultanea o combinarlos en un mismo
gráfico. Puede salvar la gráfica activa en memoria.

Interruptor mecánico de 3GHz

El NFA N8975 posee un interruptor mecánico ajustado para conmutar del rango de frecuencia de 10 MHz a 3.0 GHz al rango
de 3.0GHz a 26.5GHz. Si el rango de frecuencia que seleccione el usuario cruza el punto de 3.0GHz, el switch mecánico
se activa. El interruptor mecánico tiene un un numero limitado de activaciones sobre el cual es confiable. La
conmutación sobre los 3.0 GHz debe limitarse siempre que sea posible

Antes de emplear el N2002A debe realizarse un test de verificación, para asegurar que los caminos de conmutación
funcionen y que el VSWR este dentro de los limites [1.19]

Bibliografía

[1] NFA NOISE FIGURE ANALYZER CONFIGURATION GUIDE-KEYSIGHT

[2] NFA SERIES NOISE FIGURE ANALYZERS DEMO GUIDE-AGILENT

[3] NOISE FIGURE ANALYZERS NFA SERIES QUICK REFERENCE GUIDE-AGILENT

\subsection[Banco de atenuadores y aisladores N2002A (Noise Source Test Set )]{Banco de atenuadores y aisladores N2002A
(Noise Source Test Set )}
El banco de atenuadores y aisladores N2002A (figura \ref{seq:refDrawing0}) es un dispositivo que, de acuerdo a la nota
de aplicación de Agilent Technologies [2], esta destinado a facilitar la calibración de fuentes de ruido de forma
rápida y precisa. Es un instrumento que se integra a un sistema para calibración de fuentes de ruido, como el mostrado
en la figura 1. 

En la medición de figura de ruido de alta precisión la incertidumbre de esta esta fuertemente relacionada a la
incertidumbre de la fuente de ruido empleada en la medición. Al conformar un banco de calibración de NS empleando el
N2002A, se pueden realizar mediciones de alta precisión en “casa”, lo que evita recurrir a laboratorios especializados
para efectuar esta tarea.

se debe emplear cuando se efectúan mediciones de ENR alta precisión en fuentes de ruido.

El N2002A se inserta en el camino de señal de RF o UW, entre la salida de la fuente de ruido bajo prueba y la entrada
del NFA. La función de este es de proveer aislamiento entre la fuente de ruido y el NFA con el fin de minimizar el
coeficiente de reflexión. Las reflexiones entre el DUT y la fuente de ruido causan incertidumbre en la potencia de
ruido que emerge de la fuente; la medición entonces no se refiere a la impedancia de 50 Ohms deseados, sino a la
impedancia actual de la fuente de ruido. Incorporando el N2002A dentro del sistema de calibración se minimiza la
interacción entre el DUT y el NFA, minimizando el coeficiente de reflexión y de esta forma la incertidumbre. Esto
permite una calibración más precisa de la fuente de ruido, asegurando precisión, \ repetibilidad y trazabilidad en las
mediciones ademas de reducir de manera significativa la incertidumbre [2].

En la figura \ref{seq:refDrawing1} se muestra una vista interna y en la figura \ref{seq:refDrawing2}. Se aprecia que
este equipo esta conformado por cuatro secciones de atenuadores y una sección de aislador. Los atenuadores o el
aislador son conectados o desconectados del camino de señal por medio dos conjuntos de interruptores, etiquetados como
A2 y A6 respectivamente en la figura 3. Las señales que controlan estos interruptores provienen del exterior del
equipo, son generadas por un dispositivo de la serie 11713 de Agilent / Keysight Technologies, unidad controladora de
atenuadores e interruptores. Cada sección de atenuador permite el paso de señal en un rango de frecuencia distinto,
como se indica en la figura 3. El N2002A cubre un rango de frecuencias idéntico al del N8975, va desde 10 MHz hasta
26.5 GHz. 

\begin{figure}
\centering
\begin{minipage}{16.992cm}
Figura {\refstepcounter{Drawing}\theDrawing\label{seq:refDrawing0}}: Banco de aisladores y atenuadores Agilent N2002
\includegraphics{Sistemamedicinderuido-img004.png}\end{minipage}
\end{figure}
El N2002A no posee fuente de poder interna ni tampoco realiza mediciones. No dispone de interfaz de usuario, este equipo
debe ser comandado por medio de un dispositivo de la serie 11713. EL 11713 brinda la interfaz de usuario necesaria, en
éste el usuario realiza la selección del rango de frecuencia sobre el cual N2002A debe permitir el paso.

\begin{figure}
\centering
\begin{minipage}{17.074cm}
Figura {\refstepcounter{Drawing}\theDrawing\label{seq:refDrawing1}}: Vista interna del N2002
\includegraphics{Sistemamedicinderuido-img005.png}\end{minipage}
\end{figure}
El N2002A cuenta unicamente con interfaces eléctricas dispuestas en el panel frontal y posterior del equipo (figura
\ref{seq:refDrawing3}). La interfaz para señal de RF / UW esta dispuesta en el panel frontal (figura
\ref{seq:refDrawing3}a) y posee dos conectores, un conector es la entrada de señal en el cual se conecta la fuente de
ruido (izquierda) y el otro conector es la salida filtrada o atenuada la cual se conecta a la entrada de señal del NFA
(derecha). En en el panel posterior se encuentra la interfaz para señales de control (figura \ref{seq:refDrawing3}b),
en esta se encuentran dos conectores para las señales generadas por un dispositivo de la serie 11713.


\bigskip


\bigskip



\begin{figure}
\centering
\begin{minipage}{16.817cm}
[Warning: Draw object ignored]Figura {\refstepcounter{Drawing}\theDrawing\label{seq:refDrawing2}}: Esquema de la
estructura interna del N2002A
\end{minipage}
\end{figure}
\begin{flushleft}
\tablefirsthead{}
\tablehead{}
\tabletail{}
\tablelasttail{}
\begin{supertabular}{|m{4.4570003cm}|m{4.88cm}|}
\hline
\multicolumn{2}{|m{9.537cm}|}{\centering Tabla}\\\hline
\centering A1 &
\centering\arraybslash Aislador 18 GHz – 26,5 GHz\\\hline
\centering A2 &
\centering\arraybslash Aislador 12 GHz – 18 GHz\\\hline
\centering A3 &
\centering\arraybslash Aislador 6 GHz – 12 GHz\\\hline
\centering A4 &
\centering\arraybslash Aislador 3 GHz – 6 GHz\\\hline
\centering A5 &
\centering\arraybslash Atenuador de 3 dB\\\hline
\end{supertabular}
\end{flushleft}


\begin{figure}
\centering
\begin{minipage}{18.119cm}
[Warning: Draw object ignored]Figura {\refstepcounter{Drawing}\theDrawing\label{seq:refDrawing3}}: Interfaces eléctricas
del N2002A
\end{minipage}
\end{figure}
Según recomendación de Agilent, se debe usar el N2002A en conjunto con el Agilent N8975A (NFA) como equipo básico
fundamental para calibración de fuente de ruido. 

N2002A es un equipo para calibrar fuentes de ruido “en casa” [3.3]. El objetivo es calibrar fuentes de ruido a estándar
trazables.

Permite calibrar de manera rápida, repetible con niveles mínimos de incertidumbre. Este equipo es necesario cuando se
realizan pruebas de ENR sobre una fuente de ruido. Asegura resultados de calibración precisos, incrementa la confianza
en la medición, permite el desarrollo de DUTs con especificaciones más exigentes. Entrega resultados trazables a
estándares nacionales.

El proceso de calibración consiste en comparar el desempeño de forma trazable de una fuente de ruido en relación al
desempeño de otra fuente de ruido, que se toma como estándar de calibración, o contra las especificaciones del
fabricante. Para ello se realizan \ dos pruebas de verificación de desempeño: 

\begin{figure}
\centering
\begin{minipage}{19.011cm}
[Warning: Draw object ignored]Figura {\refstepcounter{Drawing}\theDrawing\label{seq:refDrawing4}}: Instrumentación para
calibración de fuentes de ruido [3].
\end{minipage}
\end{figure}
\begin{itemize}
\item Medición de la razón de ruido excedente (ENR).
\item Medición del coeficiente de reflexión complejo (magnitud y fase). 
\end{itemize}
Si la fuente de ruido \ falla cualquiera de estas pruebas es indicador que requiere reparación o ajuste.

El sistema de calibración de Agilent permite verificas las fuentes de ruido Agilent de la serie 346 (346A, 346B, 346C) y
las fuentes de ruido inteligentes de la serie Agilent N4000A (N4000A, N4001A, N4002A). Este proceso de calibración
permite calibrar fuentes de ruido entre 10.0MHz y 26.5GHz.

En la figura \ref{seq:refDrawing4} se muestra la instrumentación propuesta por Agilent [3] para la medida de coeficiente
de reflexión (figura \ref{seq:refDrawing4}a) y para la medida del ENR (figura \ref{seq:refDrawing4}b).

\begin{center}
\tablefirsthead{}
\tablehead{}
\tabletail{}
\tablelasttail{}
\begin{supertabular}{|m{9.09cm}|m{7.9370003cm}|}
\hline
\multicolumn{2}{|m{17.227cm}|}{Equipo necesario para calibración de NS}\\\hline
Coeficiente de reflexión &
ENR\\\hline
\begin{itemize}
\item[] Analizador de Red que cubra el rango 10MHz a 26.5GHz (VNA) 8753ES o 8753ET , 8722Es o 8722ET
\end{itemize}
 &
Analizador de figura de ruido N8975A\\\hline
\begin{itemize}
\item[] N8975A NFA.
\end{itemize}
 &
\begin{itemize}
\item[] N2002A Conjunto para probar fuentes de ruido.
\end{itemize}
\\\hline
\begin{itemize}
\item[] Kit de calibración apropiado para VNA
\end{itemize}
 &
\begin{itemize}
\item[] 11713A.
\end{itemize}
\\\hline
\begin{itemize}
\item[] Agilent 11713A
\end{itemize}
 &
\begin{itemize}
\item[] Fuente de ruido de referencia estándar.
\end{itemize}
\\\hline
\begin{itemize}
\item[] Cable Viking para conexión 11713A
\end{itemize}
 &
~
\\\hline
\end{supertabular}
\end{center}

\bigskip

\subsection[Descripción de las pruebas de verificación [1.28{]}]{Descripción de las pruebas de verificación [1.28]}
\subsubsection[Medición de ENR [1.28{]}]{Medición de ENR [1.28]}
La prueba de ENR consiste en comparar los resultados de la prueba sobre una fuente de ruido bajo prueba (DUT) contra los
resultados de una prueba sobre una fuente de ruido referencia estándar. La referencia estándar es una fuente de ruido
calibrada con valores conocidos de ENR. Las medidas se llevan a cabo tanto en la fuente de ruido DUT asi como en la
fuente de ruido referencia estándar. Los valores de ENR del DUT se derivan de estos resultados. \ 

\ Los resultados de esta prueba permiten asegurar si la fuente de ruido cumple con las especificaciones de calibración
del fabricante,

La precisión de la medidas para el DUT es altamente dependiente de la exactitud de la calibración de la referencia
estándar. Se debe usar, según Agilent, una referencia estándar que haya sido calibrada por un laboratorio
especializado.

Las pruebas deben realizarse dentro de la temperatura ambiente de 296 ± 1K (23 ± 1) °C.

Las fuentes de ruido requieren calibración periódica del desempeño operacional. En condiciones de uso normal y
ambientales, se calibra la NS cada 12 meses [1.27].

\subsubsection[Resumen del procedimiento de medición de ENR [1.29{]} [1.39{]}.]{Resumen del procedimiento de medición de
ENR [1.29] [1.39].}
Se emplea la instrumentación indicada en la figura \ref{seq:refDrawing4}b. Se inicia el proceso al encender los equipos
y permitir que calienten por una hora. Se debe permitir que las fuentes de ruido se estabilicen a la temperatura
ambiente. No se deben usar las fuentes de ruido una hora antes de realizar las mediciones.

Se deben cargar los datos de ENR de la fuente de ruido referencia estándar en el analizador de figura de ruido. Estos
datos los proporciona el fabricante de la NS ya sea en formato digital o físico. SI se usa una fuente de ruido
inteligente (SNS), el NFA puede cargar los datos de ENR que se encuentran en la memoria de la SNS de forma automática,
si el NFA esta habilitado.

La secuencia de pasos para la medición emplea las tabla \ref{seq:refDrawing5}a para registro de resultados

\begin{itemize}
\item Los valores de ENR de la referencia estándar (ENR1) son conocidos, ingresar estos valores en la columna ENR1 de la
tabla 1.
\item Se conecta el equipo de prueba como indica la figura \ref{seq:refDrawing4}b.. Conectar la fuente de ruido de
referencia, asegurando que el conector RF de esta NS es del mismo tipo que el conector de la NS DUT]. Ajustar los
interruptores del 11713, para el canal de frecuencia requerido, por ejemplo interruptores 9 y 0 ON para medir entre
10MHz y 3.0GHz. 
\item Establecer el equipo para realizar la medida del primer punto de frecuencia. 
\item Medir el factor Y lineal de la NS referencia estándar.
\item Anotar este valor en la tabla , bajo la columna Y1.
\item Establecer el equipo para medir el siguiente punto de frecuencia, repetir el procedimiento hasta que todos los
puntos de medición estén completos.
\item Remover la referencia estándar de la entrada del N2002A y la NS DUT.
\item Establecer el equipo para medir el primer punto de frecuencia.
\item Medir el factor Y lineal en la NS DUT.
\item Anotar en las tablas de registro el resultado bajo la columna Y2.
\item Repetir el procedimiento para todos los puntos a medir.
\item Con los resultados obtenidos, introducirlos en las ecuaciones y calcular con ellas el ENR y los valores de
incertidumbre.
\end{itemize}
\subsubsection{}
Conectar cables Viking de la parte posterior del 11713A a la parte posterior del N2002A. Conectar Atten X del 11713A al
Attenuator X del N2002A. Conectar Atten Y del 11713A al Attenuator Y del N2002A.

Proceso de Calibración [1.26]

SI la fuente de ruido falla cualquiera de estas pruebas de desempeño, la NS requiere reparación.

Aparte de la calibración en puntos cardinales de frecuencia, se puede realizar la calibración en otros puntos. El máximo
de puntos frecuencia-ENR es de 81.

Tabla de VSWR típico en [1.19]

Agilent N2002A empleado cuando se requiere realizar pruebas de Razón de Ruido en Exceso (ENR) sobre una fuente de ruido
[1.14]. 

\subsection{Ecuaciones }
\subsubsection{ENR}
 $\mathit{ENR}_2=10\log (\frac{(Y_2-1)(T_0\frac{10^{\frac{\mathit{ENR}_1}{10}}}{Y_1-1})}{T_0})$ (1)

\subsubsection{Incertidumbre}
 $U_C\mathit{ENR}_2=\sqrt{(U_C\mathit{ENR}_1)^2+(U_C\mathit{Sys})^2}$ (2)

donde

TO = 290 K.

ENR1 = Valor de ENR de la fuente de ruido de referencia en cada punto de frecuencia.

ENR2 = Valor calculado de ENR de la fuente de ruido DUT en cada punto de frecuencia.

Y1 = Valor medido del factor Y de la fuente de ruido de referencia en cada punto de frecuencia.

Y2 = Valor medido del factor Y de la fuente de ruido DUT en cada punto de frecuencia.

UcENR1 = Valor de incertidumbre de ENR de la fuente de ruido de referencia en cada punto de frecuencia.

UcENR2 = Valor calculador para la incertidumbre de ENR de la fuente de ruido de referencia en cada punto de frecuencia.

UcSys = Incertidumbre total del sistema de medición en cada punto de frecuencia.


\bigskip


\bigskip



\begin{figure}
\centering
\begin{minipage}{15.983cm}
[Warning: Draw object ignored]Figura {\refstepcounter{Drawing}\theDrawing\label{seq:refDrawing5}}: Tablas para registro
de datos, relativos a la medición de ENR
\end{minipage}
\end{figure}
El N2002A es controlado por el 11713A. El 11713A es controlado por el software N2002A Noise Source Demonstration
Software \ (No encontrado. Ver en su lugar VEE PRO en KeySight).

Equipo de Prueba Recomendado


\bigskip

Para mediciones de ENR y el coeficiente de reflexión (magnitud \ y fase), equipo listado en las tablas 2-4 y 2-5 del
documento [1].

\subsection{}
\subsection{Medición de coeficiente de reflexión (magnitud y fase) [1.31]}
\subsection{Rango de las mediciones de ENR para fuentes de ruido Agilent}
La medición de ENR sobre la fuente de ruido permite garantizar que esta se encuentre dentro de las especificaciones, por
ejemplo dadas por Agilent en la tabla 1.

\subsection[]{}
\begin{figure}
\centering
\begin{minipage}{5.29cm}
Tabla 3 [1.30]:

\includegraphics{Sistemamedicinderuido-img006.png}Rango de ENR para fuentes de ruido
\end{minipage}
\end{figure}
Automatización del proceso de medición

El software que automatiza el proceso de medición esta escrito dentro de Agilent VEE Pro, el cual parece ser un entorno
de ejecución (run time), esta disponible como un archivo VEE Pro o un archivo VEE Pro run time (posiblemente sea un
script). 

\begin{itemize}
\item Software listado en [3.8]
\item Agilent VEE Pro
\item Agilent VEE Pro run time (provisto con el N2002A)
\end{itemize}
VEE Pro puede comunicarse a través de GPIB, LAN, USB, RS-232, VXI y LXI.

Para su uso requiere una tarjeta de interfaz GPIB (Agilent o National Instruments).

Sistema de calibración de fuente de ruido Agilent


\bigskip

El sistema de la figura 1 cuenta con los equipos [2.5]:

\begin{itemize}
\item NFA N8975, opera en un rango de 10MHz a 26.6GHz) (con opción 1D5 la cual es una referencia de frecuencia de alta
estabilidad). 
\item Cuenta con el N2002A conjunto para pruebas con fuente de ruido que puede incluir todos los cables y conectores
necesarios para ejecutar calibración de NS con conectores de 3.5mm y tipo N (opción 001). Incluye el 11713A. 
\item Incluye una fuente de ruido estándar de oro.
\end{itemize}
Características del sistema [2.5]

Entre otras, puede calibrar fuentes de ruido tipo SNS y de la serie 346 de Agilent.

Nota importante de [2.3]

El N2002A conjunto para pruebas con fuente de ruido debe ser usado en un sistema de calibración de fuente de ruido, este
equipo no posee fuente de poder y no realiza mediciones.

\subsection{}

\bigskip

\subsection{Equipo complementario al N2002A}
Según [2.4] este equipo cuenta con las fuentes de ruido inteligentes (SNS) Agilent de la serie N4000 y de la serie 346.


\bigskip


\bigskip


\bigskip

\subsection{Prueba de verificación [1.17]}
Por medio de un analizador vectorial de red se verifica el coeficiente sea el que indica la tabla para cada frecuencia.

\begin{flushleft}
\tablefirsthead{}
\tablehead{}
\tabletail{}
\tablelasttail{}
\begin{supertabular}{|m{3.972cm}|m{1.012cm}|m{1.012cm}|m{1.012cm}|m{1.012cm}|m{1.012cm}|m{1.012cm}|m{1.012cm}|m{1.012cm}|m{1.012cm}|m{1.012cm}|}
\hline
\centering Frecuencia (GHz) &
\centering 9 &
\centering 10 &
\centering 11 &
\centering 12 &
\centering 13 &
\centering 14 &
\centering 15 &
\centering 16 &
\centering 17 &
\centering\arraybslash 18\\\hline
\centering Limite de ROE típico &
\centering 1:1.15 &
\centering 1:1.15 &
\centering 1:1.15 &
\centering 1:1.15 &
\centering 1:1.15 &
\centering 1:1.15 &
\centering 1:1.15 &
\centering 1:1.15 &
\centering 1:1.15 &
\centering\arraybslash 1:1.15\\\hline
\centering Combinación botones 11713 &
\centering 3 {}- 7 &
\centering 3 {}- 7 &
\centering 3 {}- 7 &
\centering 3 {}- 7 &
\centering 2 {}- 6 &
\centering 2 {}- 6 &
\centering 2 {}- 6 &
\centering 2 {}- 6 &
\centering 4 {}- 8 &
\centering\arraybslash 4 {}- 8\\\hline
\end{supertabular}
\end{flushleft}

\bigskip

\begin{flushleft}
\tablefirsthead{}
\tablehead{}
\tabletail{}
\tablelasttail{}
\begin{supertabular}{|m{3.972cm}|m{1.012cm}|m{1.012cm}|m{1.012cm}|m{1.012cm}|m{1.012cm}|m{1.012cm}|m{1.012cm}|m{1.012cm}|m{1.012cm}|}
\hline
\centering Frecuencia (GHz) &
\centering 19 &
\centering 20 &
\centering 21 &
\centering 22 &
\centering 23 &
\centering 24 &
\centering 25 &
\centering 26 &
\centering\arraybslash 25.5\\\hline
\centering Limite de ROE típico &
\centering 1:1.18 &
\centering 1:1.18 &
\centering 1:1.18 &
\centering 1:1.18 &
\centering 1:1.18 &
\centering 1:1.18 &
\centering 1:1.18 &
\centering 1:1.18 &
\centering\arraybslash 1:1.18\\\hline
\centering Combinación botones 11713 &
\centering 4 {}- 8 &
\centering 4 {}- 8 &
\centering 4 {}- 8 &
\centering 4 {}- 8 &
\centering 4 {}- 8 &
\centering 4 {}- 8 &
\centering 4 {}- 8 &
\centering 4 {}- 8 &
\centering\arraybslash 4 {}- 8\\\hline
\end{supertabular}
\end{flushleft}
[Warning: Draw object ignored]

\subsection{}

\bigskip

Bibliografía

[1] N2002A NOISE SOURCE TEST SET USER'S GUIDE

[2] AGILENT N2002A NOISE SOURCE TEST SET 10MHZ TO 26.5GHZ.

[3] NOISE SOURCE CALIBRATION USING THE AGILENT N8975A NOISE FIGURE ANALYZER AND THE N2002A NOISE SOURCE TEST SET-AGILENT

[4] \ KEYSIGHT N2002A NOISE SOURCE TEST SET-KEYSIGHT


\bigskip

\subsection[Controlador de interruptores y atenuadores 11713 (Attenuator Switch Driver).]{Controlador de interruptores y
atenuadores 11713 (Attenuator Switch Driver).}
Los equipos de la serie 11713 están diseñados para generar las señales de control o conmutación para \ bancos de
atenuadores o interruptores electromecánicos para RF y UW, de acuerdo a la selección hecha por el usuario en su panel
frontal o en forma de comando enviado de forma remota a este dispositivo a través de un bus GPIB, USB o una red LAN.
Los atenuadores e interruptores coaxiales electromecánicos no disponen de interfaz de usuario, se debe emplear un
equipo de la serie 11713 para que el usuario pueda controlar a estos dispositivos.

El 11713 permite al usuario controlar un banco de interruptores o atenuadores interactuando con su interfaz física en su
panel frontal (modo local) y también permite control en modo remoto, el usuario puede enviar comandos a través de un
bus GPIB (modelo 11713A), un bus USB o una red LAN (modelos 111713B y 11713C).

Fabricado inicialmente por Agilent Technologies con el modelo 11713A (figura \ref{seq:refDrawing6}a) actualmente es
producido por Keysight Technologies, en dos versiones mejoradas pero que conservan toda la funcionalidad del equipo
original de Agilent, en los equipos 11713B (figura \ref{seq:refDrawing6}b) y 11713C (figura \ref{seq:refDrawing6}c).

Los equipos de la serie 11713 pueden controlar una amplia gama de modelos de atenuadores o interruptores, en la tabla
\ref{seq:refTable0} se muestran los modelos compatibles de Agilent. Los interruptores a controlar pueden ser de tipo
SPDT, bypass, matrix, transfer y multipuerto.

\begin{figure}
\centering
\begin{minipage}{17.552cm}
[Warning: Draw object ignored]Figura {\refstepcounter{Drawing}\theDrawing\label{seq:refDrawing6}}: Versiones para los
equipos de la serie 11713.
\end{minipage}
\end{figure}
\begin{center}
\tablefirsthead{}
\tablehead{}
\tabletail{}
\tablelasttail{}
\begin{supertabular}{|m{3.504cm}|m{13.527cm}|}
\hline
\centering Tipo de interruptor &
\centering\arraybslash Modelos Agilent\\\hline
\centering SPDT &
\centering\arraybslash 8761B, 8762A/B/C/F, 8765A/B/C/D/F, N1810TL, N1810UL\\\hline
\centering Bypass &
\centering\arraybslash 8763A/B/C, 8764A/B/C, N1811TL, N1812UL\\\hline
\centering Multipuerto &
\centering\arraybslash 87104A/B/C, 87204A/B/C, 87106A/B/C, 87206A/B/C, 8766K, 8767K, 8768K, 8769K, 8767M, 8768M,
8769M\\\hline
\centering Matrix &
\centering\arraybslash 87406B, 87606B, \\\hline
\centering Transfer &
\centering\arraybslash 87222C/D/E\\\hline
\end{supertabular}
\end{center}
Tabla {\refstepcounter{Table}\theTable\label{seq:refTable0}}: Modelos de interruptores Agilent compatibles con el 11713


\bigskip

\begin{center}
\tablefirsthead{}
\tablehead{}
\tabletail{}
\tablelasttail{}
\begin{supertabular}{|m{8.514cm}|m{8.514cm}|}
\hline
\multicolumn{2}{|m{17.227999cm}|}{~
}\\\hline
\centering Modelo de atenuador Agilent &
\centering\arraybslash Atenuación\\\hline
\centering 8494G,H (33320G,H) &
\centering\arraybslash 11 dB, paso 1 dB \\\hline
\centering 8495G,H,K (33321 G,H,K) &
\centering\arraybslash 70 dB, paso 10 dB \\\hline
\centering 8496G,H (33322G,H) &
\centering\arraybslash 110 dB, paso 10 dB \\\hline
\centering 8497K ( 33323K) &
\centering\arraybslash 90 dB, paso 10 dB \\\hline
\centering 84904K,L (33324K,L) &
\centering\arraybslash 11 dB, paso 1 dB \\\hline
\centering 84906K,L ( 33326K,L) &
\centering\arraybslash 90 dB, paso 10 dB \\\hline
\centering 84907K,L (33327K,L) &
\centering\arraybslash 70 dB, paso 10 dB \\\hline
\end{supertabular}
\end{center}
Los equipos de la serie pueden manejar un numero de atenuadores o interruptores, según el modelos. En \ general, el
modelo 11713C puede el doble de atenuadores e interruptores que el modelo 11713B.

Los equipos de la serie 11713 disponen de dos tipos de interfaces, una interfaz de usuario y una interfaz eléctrica. Los
equipos 11713B y 11713C agregan una tercera interfaz de comunicaciones. Estos equipos no presentan una interfaz para
señales de RF o UW, no manejan ni realizan mediciones sobre este tipo de señales.

La interfaz de usuario se encuentra en el panel frontal de estos equipos, en el 11713A consiste básicamente en tres
grupos de pulsadores. Los equipos 11713B y el 11713C también disponen de pulsadores en su panel frontal y además
agregan una pantalla LCD a la interfaz de usuario.


\bigskip


\bigskip

\begin{center}
\tablefirsthead{}
\tablehead{}
\tabletail{}
\tablelasttail{}
\begin{supertabular}{m{2.352cm}|m{3.598cm}|m{3.598cm}|m{3.592cm}|m{3.287cm}|}
\hline
\multicolumn{5}{|m{17.227cm}|}{\centering Interfaz de usuario en los equipos 11713}\\\hline
\multicolumn{1}{|m{2.352cm}|}{~
} &
~
 &
\centering 11713A &
\centering 11713B &
\centering\arraybslash 11713C\\\hline
\multicolumn{1}{|m{2.352cm}|}{\centering Botones} &
{\centering Control de atenuadores X\par}

{\centering (ATTENUATOR X)\par}

\centering (1 al 4) &
\centering 1 banco de 4 botones.  &
\centering 1 banco de 4 botones &
{\centering 2 bancos de 4 botones cada uno.\par}

~
\\\hline
 &
{\centering Control de atenuadores Y (ATTENUATOR Y) \par}

\centering (5 al 6) &
\centering 1 banco de 4 botones.  &
\centering 1 banco de 4 botones &
{\centering 2 bancos de 4 botones cada uno.\par}

~
\\\hhline{~----}
 &
{\centering Control de interruptores \par}

\centering (9 y 0) &
\centering 1 banco de 2 botones &
{\centering 1 banco de 2 botones\par}

~
 &
\centering\arraybslash 2 bancos de dos botones cada uno. \\\hhline{~----}
 &
\centering Teclas de flecha &
\centering No &
\centering Si  &
\centering\arraybslash Si\\\hhline{~----}
 &
\centering Preset, Config, Save/Recall &
\centering No &
\centering Si &
\centering\arraybslash Si\\\hhline{~----}
\multicolumn{1}{|m{2.352cm}|}{\centering Pantalla LCD} &
~
 &
\centering No &
\centering Si &
\centering\arraybslash Si\\\hline
\end{supertabular}
\end{center}

\bigskip

El 11713 presenta una interfaz eléctrica la cual entrega señales de control que permiten seleccionar un nivel de
atenuación en los atenuadores o abrir y cerrar un interruptor coaxial. Esta interfaz es accesible por medio del panel
trasero de los equipos de la serie 11713, en forma de conectores como se aprecia en la figura 8 un detalle del panel
posterior del 11713B. En este panel se encuentran los conectores con las señales de control para atenuadores e
interruptores coaxiales. 

\begin{center}
\tablefirsthead{}
\tablehead{}
\tabletail{}
\tablelasttail{}
\begin{supertabular}{|m{4.157cm}|m{4.157cm}|m{4.157cm}|m{4.157cm}|}
\hline
\multicolumn{4}{|m{17.227999cm}|}{\centering Interfaz eléctrica en los equipos 11713}\\\hline
~
 &
\centering 11713A &
\centering 11713B &
\centering\arraybslash 11713C\\\hline
{\centering Control de atenuadores\par}

\centering (conectores Viking de 12 pines) &
\centering 1 par de conectores  &
\centering \ par de conectores &
\centering\arraybslash 2 pares de conectores \\\hline
{\centering Control \ de interruptores coaxiales\par}

\centering (jacks banana) &
\centering 1 par (A y B) &
\centering 1 par (A y B) &
\centering\arraybslash 2 pares\\\hline
\centering Alimentación DC en los puertos  &
\centering +24 V DC &
\centering +24 V DC &
\centering\arraybslash +5, +15, +24 V DC, ajustable por usuario\\\hline
\centering Control TTL &
\centering No &
\centering No &
\centering\arraybslash Si\\\hline
\centering Máximo de atenuadores programables por pasos &
\centering 2 de 4 secciones &
\centering 2 de 4 secciones &
\centering\arraybslash 4\\\hline
\centering Máximo de interruptores coaxiales &
{\centering 2 en los jack banana\par}

\centering Hasta 10 SPDT con los conectores Viking &
{\centering 2 en los jack banana\par}

\centering Hasta 10 SPDT con los conectores Viking &
{\centering 4 en los jack banana\par}

\centering\arraybslash Hasta 20 SPDT con los conectores Viking\\\hline
\end{supertabular}
\end{center}

\bigskip

En un banco de atenuadores, como los \ de Agilent, la cantidad de atenuación que se introduce en el camino de señal es
determinada por apertura o cierre de un un conjunto de interruptores electromecánicos, que insertan o retiran
atenuadores en el camino de señal. Los interruptores coaxiales también emplean interruptores electromecánicos. Las
señales de comando que el 11713 envía a los interruptores electromecánicos es una señal de potencia, de tipo lógico y
referidas a tierra.

Las señales de control para los modelos de atenuadores Agilent se encuentran en los conectores Viking. \ Existe un par
de éstos en los modelos 11713 A y B, etiquetados como ATTEN X y ATTEN Y. El modelo 11713C dispone de dos pares de
conectores Viking. Las señales presentes en los conectores Viking también pueden emplearse para el control de
interruptores electromecánicos coaxiales.

La conexión entre un equipo de la serie 11713 con un atenuador o un interruptor coaxial se realiza por medio de cables
especiales que se insertan en los conectores que disponen estos equipos. La información sobre modelos de cable de
acuerdo al modelo de atenuador o interruptor se ofrece en las hojas de datos en forma de matrices de selección. Los
cables de conexión se eligen de acuerdo al numero de opción de los equipos 11713 y de acuerdo al modelo del equipo
atenuador o interruptor, ubicando estos datos en la matriz de selección. La matriz de selección remite a una figura en
donde se indica un esquema con instrucciones para realizar las conexiones entre éstos equipos. 

\begin{figure}
\centering
\begin{minipage}{17.157cm}
[Warning: Draw object ignored]Figura \stepcounter{Drawing}{\theDrawing}: Sección del panel posterior del 11713B
\end{minipage}
\end{figure}
Existe un modelo de cable apropiado para cada modelo de atenuador o interruptor coaxial, pero uno de sus extremos
siempre debe poseer un conector Viking hembra si se desean utilizar éstos con un equipo 11713. 

\ Un conector Viking en el equipo 11713, como se muestra en la figura \ \ref{seq:refDrawing8}, \ posee 12 pines. Los
pines 1 y 2 portan la tensión DC para alimentar al periférico. Esta tensión de alimentación DC en los equipos 11713A y
11713B es de valor fijo de +24 V DC, en el 11713C puede ser seleccionada por el usuario a un valor fijo de +5, +15, +24
V DC o ajustada a un valor entre 0 y +24 V DC. Los pines del 3 al 12 llevan las señales de conmutación, las cuales son
de tipo lógico. En la figura \ref{seq:refDrawing9} se muestra un esquema de driver interno en el 11713 para las señales
de control. De esta figura se deduce que las señales de control trabajan en pares, esto es, mientras un pin es llevado
a tierra el pin complementario es colocado en alta impedancia. Las señales de control en conjunto con la alimentación
DC común permite manejar parejas de interruptores electromecánicos \ en dos estados, abierto y cerrado, o interruptores
electromecánicos simples en los cuales una bobina abre y otra bobina cierra el circuito



\begin{figure}
\centering
\begin{minipage}{17.268cm}
[Warning: Draw object ignored]Figura {\refstepcounter{Drawing}\theDrawing\label{seq:refDrawing8}}: Disposición de pines
en un conector Viking
\end{minipage}
\end{figure}


\begin{figure}
\centering
\begin{minipage}{17.214cm}
[Warning: Draw object ignored]Figura {\refstepcounter{Drawing}\theDrawing\label{seq:refDrawing9}}

\ Diagrama interno generación de señales en conectores Viking
\end{minipage}
\end{figure}
Los conectores banana ubicados en el panel posterior en los equipos 11713 (figura 8) están destinados al comando de
interruptores electromecánicos coaxiales. Están dispuestos en parejas y etiquetados como A y B. En los \ modelos 11713A
y 11713B existen dos pares de éstos (S0 y S9) y en el modelo 11713C posee cuatro pares con las etiquetas S0 y S9 (Bank1
y Bank2). En la figura \ref{seq:refDrawing10} se muestra un diagrama del driver interno para estos puertos. Ambos jacks
bananas, A y B, generan una señal de tipo binaria y trabajan de forma complementaria, esto significa que cuando un jack
presenta la tensión de tierra (0 V) el jack complementario presenta una tensión DC. Cada grup de conectores S0 y S9
diponen de un jack banana común con un suministro de alimentación DC, para los interruptores que la requieran. Esta
tensión de alimentación en los equipos 11713A y 11713B es fija en +24 V DC. En el modelo 11713C esta tensión es puede
ser programada por el usuario a un valor fijo de +5, +15 y 25 VDC o ajustada a un valor entre 0 y +24V DC.


\bigskip



\begin{figure}
\centering
\begin{minipage}{17.404cm}
[Warning: Draw object ignored]Figura {\refstepcounter{Drawing}\theDrawing\label{seq:refDrawing10}} 

Diagrama interno generación de señales en jacks banana
\end{minipage}
\end{figure}
En el sistema para medición de ruido de la figura 1, un equipo de la serie 11713 se emplea con un doble propósito,
servir como interfaz de usuario y controlador del banco de atenuadores N2002. El estado de cada pareja de pines en un
conector Viking en el panel posterior de estos equipos esta relacionado en forma directa con el estado del un botón en
el panel frontal. En la tabla \ref{seq:refTable1} se indica esta relación. Los botones en el panel frontal ubicados en
la sección etiquetado como Attenuator X controlan el estado de los pines ubicados en el conector Viking del panel
posterior etiquetado como ATTEN X. Se cumple una relación idéntica para los botones en la sección etiquetada como
Attenuator Y del panel frontal y los conectores Viking etiquetados como ATTEN Y del panel posterior. Por ejemplo los
pines 5 y 6 en el conector Viking ATTEN X se corresponde al estado del botón 1 en la sección Attenuator X. De la misma
forma, los pines 5 y 6 del conector Viking ATTEN Y responden al estado del botón 5 de la sección Attenuator Y.

\begin{table}
\raggedright
\begin{tabular}{|m{2.34cm}|m{5.448cm}|}

\multicolumn{2}{m{7.988cm}}{{\centering Tabla {\refstepcounter{Table}\theTable\label{seq:refTable1}}\par}

\centering Relación botones panel frontal con los pines en puertos Viking}\\\hline
{\centering Pines\par}

\centering ATTEN X y ATTEN Y &
{\centering Controlado por botones \par}

\centering\arraybslash Attenuator X, Attenuator Y\\\hline
\centering 1 &
\centering\arraybslash Alimentación \\\hline
\centering 2 &
\centering\arraybslash Tierra (GND)\\\hline
\centering 5 &
{\centering Botón 1 para ATTEN X \par}

\centering\arraybslash Botón 5 para ATTEN Y\\\hline
\multicolumn{1}{|m{2.34cm}}{\centering 6} &
\\\hhline{-~}
\centering 7 &
{\centering Botón 2 para ATTEN X\par}

\centering\arraybslash Botón 6 para ATTEN Y\\\hline
\multicolumn{1}{|m{2.34cm}}{\centering 8} &
\\\hhline{-~}
\centering 9 &
{\centering Botón 3 para ATTEN X \par}

\centering\arraybslash Botón 7 para ATTEN Y\\\hline
\multicolumn{1}{|m{2.34cm}}{\centering 10} &
\\\hhline{-~}
\centering 11 &
{\centering Botón 4 para ATTEN X \par}

\centering\arraybslash Botón 8 para ATTEN Y\\\hline
\multicolumn{1}{|m{2.34cm}}{\centering 12} &
\\\hhline{-~}\end{tabular}

\end{table}

\bigskip

Los pines en los puertos Viking operan en pareja y de forma complementaria, cuando un pin se encuentra a tierra (GND) su
pareja correspondiente se encuentra en alta impedancia. En la tabla 3 se indica la relación que existe \ entre el
estado de los botones en el panel frontal y el estado de los pines en los conectores Viking. Por ejemplo, cuando el
botón 1 del panel Attenuator X se encuentra encendido, en el respectivo conector Viking ATTEN X, el pin 5 se encuentra
a tierra (GND) mientras que su pin complementario se encuentra en alta impedancia. Cuando el mismo botón se apaga, los
pines 5 y 6 intercambian de estado.

\begin{table}
\centering
\begin{tabular}{m{1.2179999cm}m{1.2179999cm}m{2.892cm}|m{2.217cm}|m{2.218cm}|}

\multicolumn{5}{m{10.563001cm}}{{\centering Tabla \stepcounter{Table}{\theTable}\par}

\centering Configuración botones y pines puerto Viking}\\\hline
\multicolumn{2}{|m{2.636cm}|}{{\centering Botones \par}

\centering ATTENUATORS} &
\centering Estado del botón &
\multicolumn{2}{m{4.635cm}|}{{\centering Estado de los pines puerto \par}

\centering ATTEN}\\\hline
\multicolumn{1}{|m{1.2179999cm}|}{\centering X} &
\centering Y &
 &
\centering Pin &
\centering\arraybslash Estado\\\hhline{--~--}
\multicolumn{1}{|m{1.2179999cm}|}{\centering 1} &
\multicolumn{1}{m{1.2179999cm}|}{\centering 5} &
\centering OFF &
\centering 5  &
\centering\arraybslash GND\\\hline
 &
 &
 &
\centering 6 &
\centering\arraybslash Hi Z\\\hhline{~~~--}
 &
 &
\centering ON &
\centering 5 &
\centering\arraybslash Hi Z\\\hhline{~~---}
 &
 &
 &
\centering 6 &
\centering\arraybslash GND\\\hhline{~~~--}
\multicolumn{1}{|m{1.2179999cm}|}{\centering 2} &
\multicolumn{1}{m{1.2179999cm}|}{\centering 6} &
\centering OFF &
\centering 7 &
\centering\arraybslash GND\\\hline
 &
 &
 &
\centering 8 &
\centering\arraybslash Hi Z\\\hhline{~~~--}
 &
 &
\centering ON &
\centering 7 &
\centering\arraybslash Hi Z\\\hhline{~~---}
 &
 &
 &
\centering 8 &
\centering\arraybslash GND\\\hhline{~~~--}
\multicolumn{1}{|m{1.2179999cm}|}{\centering 3} &
\multicolumn{1}{m{1.2179999cm}|}{\centering 7} &
\centering OFF &
\centering 9 &
\centering\arraybslash GND\\\hline
 &
 &
 &
\centering 10 &
\centering\arraybslash Hi Z\\\hhline{~~~--}
 &
 &
\centering ON &
\centering 9 &
\centering\arraybslash Hi Z\\\hhline{~~---}
 &
 &
 &
\centering 10 &
\centering\arraybslash GND\\\hhline{~~---}
\multicolumn{1}{|m{1.2179999cm}|}{\centering 4} &
\multicolumn{1}{m{1.2179999cm}|}{\centering 8} &
\centering OFF &
\centering 11 &
\centering\arraybslash GND\\\hline
 &
 &
 &
\centering 12 &
\centering\arraybslash Hi Z\\\hhline{~~~--}
 &
 &
\centering ON &
\centering 11 &
\centering\arraybslash Hi Z\\\hhline{~~---}
 &
 &
 &
\centering 12 &
\centering\arraybslash GND\\\hhline{~~~--}\end{tabular}

\end{table}

\bigskip

Los jack banana presentes en el panel posterior, etiquetados como A y B bajo las secciones S9 y S0, también operan por
parejas, su estado es complementario y es función del estado presente en los botones del panel frontal, etiquetados
como S9 y S0. La tabla \ref{seq:refTable3} resume la relación que existe entre los botones S9 y S0 con los niveles de
tensión presentes en los jack banana A y B de las secciones S9 y S0. Por ejemplo, de la tabla \ref{seq:refTable3} se
observa que cuando el botón S9 esta encendido (ON), en la sección S9 del panel posterior el estado el jack banana A
\ se encuentra puesto a tierra (0 V) mientras que el jack banana B presenta una tensión positiva DC (+24 V en el 11713
A). Si ahora el botón S9 se apaga (OFF), los jacks banana A y B de la sección S9 intercambian de estado.

\begin{table}
\centering
\begin{tabular}{m{2.0cm}|m{2.5019999cm}|m{1.6099999cm}|m{1.694cm}|m{1.583cm}|}

\multicolumn{5}{m{10.189cm}}{{\centering Tabla {\refstepcounter{Table}\theTable\label{seq:refTable3}}\par}

\centering Relación entre los botones S9 y S0 en el panel frontal y los jacks banana en el panel posterior}\\\hline
\multicolumn{1}{|m{2.0cm}|}{\centering Interruptor coaxial} &
\centering Estado botones S9 y S0 &
\multicolumn{3}{m{5.287cm}|}{\centering Tensión en el jack banana panel posterior}\\\hline
 &
 &
\centering Jack &
\centering A &
\centering\arraybslash B\\\hhline{~~---}
\multicolumn{1}{|m{2.0cm}|}{\centering S9} &
\centering OFF &
\centering S9  &
\centering +V DC &
\centering\arraybslash GND \\\hline
 &
\centering ON &
\centering S9 &
\centering GND  &
\centering\arraybslash +V DC\\\hhline{~----}
\multicolumn{1}{|m{2.0cm}|}{\centering S0} &
\centering OFF &
\centering S0  &
\centering + V DC &
\centering\arraybslash GND\\\hline
 &
\centering ON &
\centering S0  &
\centering GND  &
\centering\arraybslash +V DC\\\hhline{~----}\end{tabular}

\end{table}

\bigskip

En el sistema para medición de figura de ruido, el banco de atenuadores N2002 permite el paso de ciertos intervalos de
frecuencia de acuerdo al estado de los botones en el panel frontal de un equipo de la serie 11713. \ \ En la tabla
\ref{seq:refTable4} se muestran los rangos de frecuencia de las señales para las cuales el N2002 permite el paso en
función del estado de los botones en las secciones Attenuator X y Attenuator Y del panel frontal de un 11713. De esta
tabla, cuando los interruptores S9 y S0 están presionados, el N2002 admite el paso de señales cuyas frecuencias se
encuentre entre 10 MHz y 3 GHz. Cuando los botones 1 y 5 de las secciones respectivas Attenuator X y \ Attenuator Y del
panel frontal están presionados, el N2002 admite el paso de señales con frecuencias comprendidas entre 3 GHz y 6 GHz.

\begin{table}
\raggedright
\begin{tabular}{|m{3.587cm}|m{1.163cm}|m{1.163cm}|m{1.163cm}|m{1.163cm}|m{1.163cm}|m{1.163cm}|m{1.163cm}|m{1.163cm}|m{1.163cm}|m{1.163cm}|}

\multicolumn{11}{m{17.217001cm}}{{\centering Tabla {\refstepcounter{Table}\theTable\label{seq:refTable4}}\par}

\centering Relación entre los rangos de frecuencia de paso de banda en el N2002 y la combinación de botones en un equipo
11713}\\\hline
\centering Rango de frecuencia (MHz) &
\multicolumn{8}{m{10.704cm}|}{\centering Atenuadores} &
\multicolumn{2}{m{2.526cm}|}{\centering Interruptores}\\\hline
 &
\multicolumn{4}{m{5.2520003cm}|}{\centering Attenuator X} &
\multicolumn{4}{m{5.2520003cm}}{\centering Attenuator Y} &
 &
\\\hhline{~--------~~}
 &
\centering 1 &
\centering 2 &
\centering 3 &
\centering 4 &
\centering 5 &
\centering 6 &
\centering 7 &
\centering 8 &
\centering S9 &
\centering\arraybslash S0\\\hhline{~----------}
\centering 10 a {\textless} 3000 &
~
 &
~
 &
~
 &
~
 &
~
 &
~
 &
~
 &
~
 &
\centering X &
\centering\arraybslash X\\\hline
\centering 3000 a 6000 &
\centering X &
~
 &
~
 &
~
 &
\centering X &
~
 &
~
 &
~
 &
~
 &
~
\\\hline
\centering {\textless} 6000 a 12000 &
~
 &
~
 &
\centering X &
~
 &
~
 &
~
 &
\centering X &
~
 &
~
 &
~
\\\hline
\centering {\textgreater} 12000 a 18000 &
~
 &
\centering X &
~
 &
~
 &
~
 &
\centering X &
~
 &
~
 &
~
 &
~
\\\hline
\centering {\textgreater} 18000 a 26500 &
~
 &
~
 &
~
 &
\centering X &
~
 &
~
 &
~
 &
\centering X &
~
 &
~
\\\hline\end{tabular}

\end{table}

\bigskip

\begin{center}
\tablefirsthead{}
\tablehead{}
\tabletail{}
\tablelasttail{}
\begin{supertabular}{|m{5.0880003cm}|m{8.992001cm}|}
\multicolumn{2}{m{14.280001cm}}{{\centering Tabla \stepcounter{Table}{\theTable}\par}

\centering Especificaciones para interruptores internos}\\\hline
\centering Tempo de respuesta  &
\centering\arraybslash 100 $\mu $s máximo parejas de contactos 1 a 8\\\hline
 &
\centering\arraybslash 20 ms máximo para pareja de contactos 9 y 0.\\\hhline{~-}
\centering Tiempo de vida &
\centering\arraybslash Menos de 2 000 000 conmutaciones a 0.7 A para contactos 9 y 0\\\hline
\centering Máxima inductancia de carga &
\centering\arraybslash 500 mH\\\hline
\centering Máxima capacitancia de carga &
\centering\arraybslash Menor a 0.01uF para contactos 9 y 0\\\hline
\end{supertabular}
\end{center}
Interfaces de comunicaciones

\begin{center}
\tablefirsthead{}
\tablehead{}
\tabletail{}
\tablelasttail{}
\begin{supertabular}{|m{4.157cm}|m{4.157cm}|m{4.157cm}|m{4.157cm}|}
\hline
~
 &
\centering 11713A &
\centering 11713B &
\centering\arraybslash 11713C\\\hline
\centering GPIB &
\centering Si &
\centering Si &
\centering\arraybslash Si\\\hline
\centering USB &
\centering No &
\centering Si &
\centering\arraybslash Si\\\hline
\centering LAN &
\centering No &
\centering Si &
\centering\arraybslash Si\\\hline
\end{supertabular}
\end{center}

\bigskip



\begin{figure}
\centering
\begin{minipage}{17.325cm}
Figura \stepcounter{Drawing}{\theDrawing}

\includegraphics{Sistemamedicinderuido-img007.png}\ Puertos en interfaz de comunicaciones
\end{minipage}
\end{figure}
Control desde panel frontal o remoto, programabilidad de software, interfaz de usuario.

\begin{itemize}
\item Programación remota desde interfaces \ [1.1]:
\item GPIB de acuerdo a IEEE 488.2 y IEC65 (compatibilidad con SH0, AH1, T0, TE0, L2, LE0, SR0, RL1, PP0, DC0, DT0, C0).
\item 10/100 BaseT \ LAN.
\item USB 2.0.
\end{itemize}
Lenguaje de comandos:

SCPI standard interface commands, compatible hacia atrás con Keysight 11713A.

\subsection{Especificaciones eléctricas}
\subsection{}
\begin{flushleft}
\tablefirsthead{}
\tablehead{}
\tabletail{}
\tablelasttail{}
\begin{supertabular}{|m{4.795cm}|m{11.89cm}|}
\hline
\multicolumn{2}{|m{16.884998cm}|}{\centering Fuente de alimentación periféricos}\\\hline
\centering Voltaje &
{\centering +24V ± 5\%\par}

\centering\arraybslash +24 V ± 2,0 VDC\\\hline
 &
\centering\arraybslash +5V ± 5\% (Únicamente 11713C)\\\hhline{~-}
 &
\centering\arraybslash +15V ± 5\% (Únicamente 11713C)\\\hhline{~-}
\centering Corriente  &
{\centering 1.7A máxima corriente continua.\par}

{\centering 1.3 A pico máximo por 1 segundo (11713A)\par}

\centering\arraybslash 0.65 A máxima corriente continua (11713A)\\\hline
 &
\centering\arraybslash Parejas de contactos del 1 al 8, 9, 0 máxima corriente de 0.7A por contacto.\\\hhline{~-}
 &
~
\\\hhline{~-}
\multicolumn{2}{|m{16.884998cm}|}{\centering Fuente de alimentación AC}\\\hline
\centering Voltaje de linea &
{\centering 85 a 264 VAC, 47 a 63 Hz (selección automática en 11713B/C)\par}

{\centering 100 o 120 VAC +5\%, -10\% de 48 a 440 Hz (11713A)\par}

\centering\arraybslash 200 o 240 VAC, +5\%, -10\% de 48 a 66 Hz (11713A)\\\hline
 &
{\centering 100 VA máximo\par}

\centering\arraybslash 80 VA máximo (11713A)\\\hhline{~-}
\multicolumn{2}{|m{16.884998cm}|}{\centering Especificaciones para interruptores internos}\\\hline
\centering Tempo de respuesta  &
\centering\arraybslash 100 $\mu $s máximo parejas de contactos 1 a 8\\\hline
 &
\centering\arraybslash 20 ms máximo para pareja de contactos 9 y 0.\\\hhline{~-}
\centering Tiempo de vida &
\centering\arraybslash Menos de 2 000 000 conmutaciones para contactos 9 y 0\\\hline
\centering Máxima inductancia de carga &
\centering\arraybslash 500 mH\\\hline
\centering Máxima capacitancia de carga &
\centering\arraybslash Menor a 0.01uF para contactos 9 y 0\\\hline
\end{supertabular}
\end{flushleft}

\bigskip


\bigskip

Control remoto de equipo de prueba automatizado de pequeña escala (small scale automated test equipment – ATE). 

Controla hasta 20SPDT interruptores de forma concurrente, o 4 atenuadores programables y 4 interruptores SPDT Y 4
interruptores de microondas.

Con tri-fuente de voltaje integrada , ahorra espacio en rack (unidamente en el 11713C), de 5, 15 y 24V.

El 11713C tiene dos bancos individuales de salidas, con voltajes de comando independientes.

El 11713C posee un driver Fast TTL, con cualquiera de los conectores Viking o en los puertos S0 o S9.

Compatible hacia atrás con el 11713A.

Provee control remoto o desde el panel frontal para atenuadores programables o relés electromecánicos o de estado
solido. Posee UI intuitiva con una variedad de opciones de conmutación, programabilidad de software y características
para control remoto que permiten rápido diseño de pruebas de validación automatizadas.

Programación vía GPIB/USB con instrucciones simples de una linea.

El 11713B y C es un instrumento LXI clase C (LAN eXtensions for Instrumentation – LXI), puede ser controlados de forma
remota a través de una interface web, usada en entornos de alto volumen de producción. 

Posee drivers de instrumentación como IVI-COM que provee compatibilidad de programación con entornos de desarrollo de
apps y soporta estandar de la industria de PC como lo es Component Object Model (COM).

El estandar GPIB soporta automatización a través de programación por mediode scriptsd, \ lo que asegura compatibilidad
con el 11713A.

\subsection{Características 11713B}
Instrumento compatible con GPIB.

Capacidad de manejar de forma concurrente dos atenuadores programables de cuatro secciones, \ dos interruptores
coaxiales para microondas o hasta 10 interruptores SPDT.

De forma opcional posee conectividad USB y LAN.

\subsection[Características 11713C]{Características 11713C}
Instrumento compatible con GPIB, USB, LAN.

Capacidad de manejar hasta cuatro atenuadores programables y cuatro interruptores coaxiales para microondas y hasta 20
interruptores SPDT.

Capacidad de selección de voltaje para drivers de interruptores entre 5 V, 15 V y 24 V admás de voltaje definido por el
usuario. 

\subsection{Usos del 11713}
Según [1.14] \ el equipo puede utilizarse para controlar:

\begin{itemize}
\item Hasta dos atenuadores programables por pasos de 4 secciones.
\item Dos interruptores de microondas.
\item Hasta 10 interruptores SPDT.
\end{itemize}
Tabla Comparativa

Entre los modelos 11713 C y B [2.14].



\begin{figure}
\centering
\includegraphics{Sistemamedicinderuido-img008.png}
\end{figure}
Estos modelos operan con un conjunto de atenuadores, interruptores compatibles y accesorios fabricados por
Agilent-KeySight [2.15].

\subsection[Panel frontal del Keysight 11713B]{Panel frontal del Keysight 11713B}


\begin{figure}
\centering
\includegraphics{Sistemamedicinderuido-img009.png}
\end{figure}
\begin{center}
\tablefirsthead{}
\tablehead{}
\tabletail{}
\tablelasttail{}
\begin{supertabular}{|m{0.881cm}|m{3.9609997cm}|m{11.985001cm}|}
\hline
\centering 1 &
\centering LCD Screen &
~
\\\hline
\centering 2  &
\centering Botones &
\centering\arraybslash Botones sin marcas referidos al contenido por el texto de la pantalla\\\hline
\centering 3 &
\centering Menu/Enter &
\centering\arraybslash Presione este boton para activar o desactivar el parámetro con resalte\\\hline
\centering 4 &
\centering Preset &
\centering\arraybslash Presione este boton para establecer en preset el equipo\\\hline
\centering 5 &
\centering Config &
\centering\arraybslash Presione este boton para acceder al menú de configuración. Aquí se establece el tipo de
atenuador, el voltage de alimentación y la condición TTL.\\\hline
\centering 6 &
\centering Save/Recall &
\centering\arraybslash Presione este botón para almacenar la configuración actual o restaurar configuración previamente
guardada.\\\hline
\centering 7 &
\centering Botones de navegación &
\centering\arraybslash Los botones con flechas son utilizadors para navegar los parametros mostrados en la pantalla LCD
o para el cambio de parametros como la dirección GPIB.\\\hline
\centering 8 &
\centering Interruptores &
\centering\arraybslash En modo local, los botonees interruptores 9 y 0 cambian la posición del interruptor coaxial
conectado a los jck banana ubicados en el panel posterior S9 A/B y S0 A/B respectivamente.\\\hline
\centering 9 &
\centering Attenuattor Y &
\centering\arraybslash En el modo local, los botones 5, 6, 7 y 8 cambian la configuración del la atenuación en el
atenuador o cambian la posición de los interruptores coaxiales conectados en el conector ATTEN Y en el panel
posterior.\\\hline
\centering 10 &
\centering Attenuator X &
\centering\arraybslash En el modo local, los botones 1, 2, 3 y 4 cambian la configuración del la atenuación en el
atenuador o cambian la posición de los interruptores coaxiales conectados en el conector ATTEN X en el panel
posterior.\\\hline
\centering 11 &
\centering On / Standby &
\centering\arraybslash Presione esta tecla para conmutar entre encendido y standby. Cuando la alimentación es
suministrada, este LED se ilumina en rojo. Presionar este botón una vez, enciende el equipo y coloca este LED en
verde.\\\hline
\centering 12 &
\centering Local &
\centering\arraybslash Presione esta tecla para controlar el equipo desde el panel frontal, cuando esta operando via
interfaz remota.\\\hline
\end{supertabular}
\end{center}

\bigskip

\subsection{Panel posterior del KeySIght 11713B.}


\begin{figure}
\centering
\includegraphics{Sistemamedicinderuido-img010.png}
\end{figure}

\bigskip

\begin{center}
\tablefirsthead{}
\tablehead{}
\tabletail{}
\tablelasttail{}
\begin{supertabular}{|m{0.42300004cm}|m{3.6959999cm}|m{12.87cm}|}
\hline
\centering 1 &
\centering ATTEN X &
\centering\arraybslash Conector Viking para conexión de un atenuador o interruptores.\\\hline
\centering 2 &
\centering ATTEN Y &
\centering\arraybslash Conector Viking para conexión de un atenuador o interruptores.\\\hline
\centering 3 &
\centering S9 A / B &
\centering\arraybslash Conectores Jack banana para conexión de interruptor coaxial.\\\hline
\centering 4 &
\centering 24 VDC COM &
\centering\arraybslash Conector jack banana que suministra +24VDC para alimentación de los interruptores coaxiales
conectados en S9 o S0\\\hline
\centering 5 &
\centering S0 A / B &
\centering\arraybslash Conector jack banana para conexión de interruptor coaxial\\\hline
\centering 6 &
\centering Receptáculo &
\centering\arraybslash Conecta el primario del transformador al voltaje de línea \\\hline
\centering 7 &
\centering Símbolo de alerta &
\centering\arraybslash Símbolo de alerta para el usuario\\\hline
\centering 8 &
\centering Conector GPIB &
\centering\arraybslash Conector interfaz para un dispositivo fuente a un dispositivo escucha, usado para operación
remota.\\\hline
\centering 9 &
\centering Conector LAN &
\centering\arraybslash Interfaz conector para un cable LAN (solo si dispone de la opción LXI)\\\hline
\centering 10 &
\centering Conector USB &
\centering\arraybslash Conector tipo mini{}-B de 5 pines para un cable USB.\\\hline
\centering 11 &
\centering Marcas del instrumento &
~
\\\hline
\end{supertabular}
\end{center}

\bigskip

\subsection{}
Importante

El estado de los botones LED indican el pin o cable en el panel trasero que se coloca a tierra, por ejemplo si el botón
3 de ATTENUATOR X esta iluminado, el pin 10 del conector ATTEN X esta a tierra y el pin 9 floata en alta impedancia.


\bigskip

Importante

Corriente de carga máxima en los pines es de 1.7A

Información eléctrica sobre los puertos del panel trasero


\bigskip

Esquema interno para pines de salida

Se refiere a los botones en el panel frontal marcados como Attenuator X (botones del 1 a 5) y los botones marcados como
Attenuator Y (botones del 5 al 6) \ Cada botón controla una pareja de interruptores basados en transistor en el
interior del equipo, los colectores de estos transistores se conectan a los puertos en el panel trasero marcados como
ATTEN X y ATTEN Y. Usado en el control de atenuadores


\bigskip

Los contactos del puerto trasero ATTEN X y ATTEN Y pueden también manejar hasta 10 relevadores externos, con la
precaución de utilizar diodos de protección en paralelo con la bobina de los mismos, para absorver las sobre tensiones
de conmutación.


\bigskip

Los botones del panel frontal marcados con los números nueve y cero se encargan de controlar la conmutación de los
interruptores ubicados en los banana jack del panel trasero marcados como S9 (A-B) y S0 (A-B). Cuando los botones S9 o
S0 se iluminan, se aplica la tensión del driver (24 V en el 11713A configurable en 11713B/C) al banana jack B y 0V al
jack banana A del conector respectivo en el panel trasero. Cuando los botones S9 o S0 están apagados, la tensión del
driver se aplica al banana jack A y 0 V se aplican al banana jack B de los conectores respectivos del panel trasero.


\bigskip


\bigskip


\bigskip


\bigskip


\bigskip


\bigskip


\bigskip


\bigskip


\bigskip


\bigskip


\bigskip


\bigskip


\bigskip

Los puertos ATTEN X y ATTEN Y también permiten manejar relevadores electromecánicos de acuerdo a la figura 4. La
corriente continua total de carga en loas puerto ATTEN X/ ATENY y S0/S9 debe ser menor de 650 mA.

\begin{figure}
\centering
\begin{minipage}{17.491cm}
Figura : 

\includegraphics{Sistemamedicinderuido-img011.png}Conexión típica para un atenuador programable de 4 secciones.
\end{minipage}
\end{figure}
 

\begin{figure}
\centering
\begin{minipage}{8.841cm}
Figura 4

\includegraphics{Sistemamedicinderuido-img012.png}Esquema interno de un driver para pin de salida [1.36]
\end{minipage}
\end{figure}
\subsection[\ Puerto GPIB ]{\ Puerto GPIB }
Los niveles lógicos del bus son TTL compatibles, presentan lógica negativa: el valor verdadero (1) es representado en el
rango de tensión de 0.0 a +0.4 VDC y el valor falso (0) es representado con las tensiones entre +2,5 a +5.0 VDC.


\bigskip



\begin{figure}
\centering
\begin{minipage}{10.028cm}
Figura : Receptáculo GPIB en el panel trasero [2]
\includegraphics{Sistemamedicinderuido-img013.png}\end{minipage}
\end{figure}

\bigskip


\bigskip

\subsection{Conexión de atenuadores e interruptores}
Puertos ATTEN X y ATTEN Y

En el panel trasero, los puertos marcados con ATENN X y ATTEN Y poseen conectores macho tipo Viking de 12 pines. Cada
botón en el panel frontal comanda las señales aplicadas a una pareja de pines.

\subsection{}
\subsection{}

\bigskip


\bigskip


\bigskip


\bigskip

\subsection{}


\begin{figure}
\centering
\begin{minipage}{11.456cm}
Figura

\includegraphics{Sistemamedicinderuido-img014.png}Conexión de puerto S0 a interruptor coaxial [2]
\end{minipage}
\end{figure}
\subsection{}

\bigskip


\bigskip


\bigskip


\bigskip

Bibliografía

[1] 11713BC ATTENUATOR SWITCH DRIVERS CONFIGURATION GUIDE-KEYSIGHT

[2] 11713A ATTENUATOR SWITCH DRIVER OPERATING AND SERVICE MANUAL-AGILENT


\bigskip


\bigskip


\bigskip

\subsection{}

\bigskip

\subsection{Puente LAN/GPIB para Windows E5810 }
El E5810 actúa como un puente entre una red LAN y equipos que soporten conexión GPIB y/o RS-232. \ Permite \ \ realizar
operaciones I/O para obtener mediciones de datos de manera local o remota desde la instrumentación GPIB o RS-232.

Conexiones de red

EL E5810 puede conectarse a una red de la dos siguientes formas: 

\begin{itemize}
\item Conexión a una red Empresarial (corporativa).
\item Conexión a una red Local (LAN aislada).
\item Conexión directa a una PC.
\end{itemize}
El E5810 puede servir de puente para 14 instrumentos GPIB y para un (1) instrumento RS-232, vía red Ethernet
\ 10BASE-T/100BASE-TX. El E5810 puede detectar la configuración de red y ajustarse de forma automática a la velocidad
apropiada. Este equipo posee un conector RJ-45 en su parte posterior. 

\subsection{Software }
El E5810 incluye Agilent IO libraries Suite, la cual incluye Agilent Virtual Instrument Software Architecture (VISA),
VISA COM, Standard Control Library (SICL) y varias utilidades I/O. Este software provee compatibilidad con diferentes
fabricantes de software y hardware. Provee una capa de software para operaciones I/O, permite utilizar varios lenguajes
como Visual Basic, Visual C++ y Agilent VEE.

El E5810 soporta Dynamic Host Configuration Protocol (DCHP el cual le permite obtener su dirección de IP de forma
automática. Por defecto el equipo puede usar DHCP, se puede deshabilitar DHCP y asignar una dirección estática IP al
E5810.

El E5810 soporta todas operaciones I/O provistas por VISA, VISA COM, SICL y Agilent VEE. 

Como se muestra en la figura 1, el PC posee las aplicaciones cliente VISA LAN, TCP/IP LAN, necesarias para acceder al
E5810. El E5810 posee un servidor LAN además de firmware TCP/IP LAN que le permite actuar como un servidor LAN.

El software cliente realiza conexión con el servidor remoto dentro del E5810, establecida la conexión, el cliente envía
peticiones I/O al servidor E5810 a través de la red al servidor E5810, el software instalado en la PC cliente VISA LAN
emplea la suite de protocolos TCP/IP LAN para ello. El servidor del E5810 ejecuta estas peticiones de I/O en el
instrumento GPIB o RS-232 apropiado.

El E5810 pude servir a múltiples clientes conectados en un momento dado. La cantidad máxima de clientes concurrentes
depende del uso de memoria de E5810, la cual esta determinada por el numero de clientes y el numero de sesiones que
corren estos clientes. Existe un máximo de 16 clientes accediendo de forma concurrente.

Múltiples instrumentos GPIB se pueden conectar al bus GPIB, pero solo una operación de I/O puede ocurrir en el bus GPIB
en un momento dado. Solo se ejecuta una petición de un cliente a la vez, las demás deben esperar hasta que el cliente
actual termine. El primer cliente en acceder es el primer cliente servido (cola FIFO).

En caso de que el cliente requiera realizar una secuencia de operaciones I/O que no deban ser interrumpidas, el cliente
debe obtener un lock sobre la interfaz GPIB del dispositivo. Completada la operación debe liberar el lock.


\bigskip



\begin{figure}
\centering
\begin{minipage}{11.501cm}
Figura 1

\includegraphics{Sistemamedicinderuido-img015.png}Network stack [1.25]
\end{minipage}
\end{figure}
\subsection[Conexiones típicas]{Conexiones típicas}
En una conexión Empresarial (corporativa), el E5810 se conecta a la red a través de un router o switch. En esta
configuración, el E5810 es visible en la red Empresarial. En una típica red empresarial consistiría de un Router
(Gateway), un servidor corporativo DHCP. El router envía paquetes de información cada host de la red basado en la
dirección IP de cada host. La subred esta definida por la mascara de subred. Esta mascara permite identificar a cual
subred una dirección IP pertenece. [1.28].

Para que el E5810 pueda operar en una red empresarial requiere de una dirección IP, una mascara de subred (subnet mask)
\ y un puente por defecto (default gateway). 

Si la red soporta DHCP, el E5810 obtiene estos valores del servidor DHCP. SI no soporta DHCP, la dirección IP debe
configurarse de forma manual.



\begin{figure}
\centering
\includegraphics{Sistemamedicinderuido-img016.png}
\end{figure}

\bigskip



\begin{figure}
\centering
\begin{minipage}{8.841cm}
Configuración de red típica [1.27]
\includegraphics{Sistemamedicinderuido-img017.png}\end{minipage}
\end{figure}
Para una conexión a red local se emplea un hub o un switch. Puede existir un servidor DHCP. No es visible en la red
Empresarial. 

Para una conexión directa desde un PC al E5810, es necesario un cable de crossover, el cual se conecta. desde el puerto
LAN del 5810 a el puerto LAN de la tarjeta en la PC. Tampoco es visible en la red empresarial.

\begin{figure}
\centering
\includegraphics{Sistemamedicinderuido-img018.png}
\end{figure}
\begin{figure}
\centering
\begin{minipage}{11.88cm}
[2.3] Esquema de conexión típica
\includegraphics{Sistemamedicinderuido-img019.png}\end{minipage}
\end{figure}

\bigskip

Panel frontal

Valores de configuración por defecto



\begin{figure}
\centering
\includegraphics{Sistemamedicinderuido-img020.png}
\caption[Valores por defecto [1.19{]}]{Valores por defecto [1.19]}

\end{figure}

\bigskip


\bigskip


\bigskip


\bigskip


\bigskip


\bigskip


\bigskip


\bigskip


\bigskip


\bigskip

\subsection{Comunicaciones con el E5810}
El acceso remoto, a través de una red, al E5810 se puede realizar de dos formas:

\begin{itemize}
\item Por medio del acceso web, a través de un explorador web.
\item Acceso empleando lenguaje de programación soportado por Agilent IO Libraries Suite.
\end{itemize}
Acceso Web

Empleando un explorador web (como Internet Explorer o Firefox), se accede al E5810 simplemente tecleando en la barra de
direcciones la dirección IP o el nombre de host asignado al E5810. Al presionar enter, se mostrará la pantalla de
bienvenida.

A través de esta pantalla se puede configurar los parámetros LAN/GPIB (Configure your LAN/GPIB Gateway) y establecer
comunicación con dispositivos GPIB o RS-232.

Acceso empleando un lenguaje de programación.

Se puede programar los instrumentos conectados al E5810 por medio de un lenguaje de programación soportado como C o
Visual Basic empleando VISA, VISA COM o SICL. Para ello se deben instalar en la PC las librerías I/O adecuadas,
conocidas como Agilent IO Libraries

\begin{itemize}
\item[] 
\begin{figure}
\centering
\begin{minipage}{11.03cm}
Figura

\includegraphics{Sistemamedicinderuido-img021.png}Pantalla de bienvenida mostrada en un explorador web (acceso web)
[1.30]
\end{minipage}
\end{figure}
\end{itemize}
Bibliografía

[1] AGILENT E5810A LAN GPIB GATEWAY FOR WINDOWS-AGILENT USERS GUIDE

[2] AGILENT E5810A LAN GPIB GATEWAY FOR WINDOWS-KEYSIGHT GETTING STARTED

\subsection{Conectores}
El rango de frecuencia de cualquier conector esta limitado por el primer modo de propagación de la guía de onda circular
en la estructura coaxial. Disminuir el diámetro del conductor externo incrementa la más alta frecuencia usable.
Rellenar el espacio de aire con dieléctrico disminuye la frecuencia más alta usable.

Si las dimensiones mecánicas de los conectores no esta apareada, si el enchapado/plating no es el adecuado, o si la
separación de contactos en la unión es excesiva, el coeficiente de reflexión y las perdidas resistivas se incrementan.

\subsection{Conector tipo-N}
Conectores de propósito general, sexuados, relativamente económicos. Conectores robustos de 7 mm se desempeña bien en
entornos de operación extremos y en aplicaciones que requieran conexiones repetidas.

El conector tipo N (Navy) de 50 Ohm fuen diseñado en la decada de 1940 para uso militar, operando inicialmente a 4 GHz.
Posteriormente en la decada de 1960 lo avances tecnologicos incrementaro la frecuencia a 12 GHz y luego, en modo libre,
hasta 18 GHz.

Los conectores de Agilent tipo N operan hasta 18GHz. Son compatibles con el estándar MIL-C-39012. \ Algunos productos
emplean un conector de 75 Ohm que usan un conector tipo N con un diámetro menor del conductor central.


\bigskip

\subsection{Conector 3.5mm}
Conectores de precisión de 3.5 mm, sexuados con dieléctrico de aire. El aire provee aislamiento dieléctrico entre el
conductor central y externo. Una arandela plástica dentro del cuerpo del conector soporta al conductor central.

Los conectores de 3.5 mm hacen juego con conectores SMA. Los conectores de 3.5 mm son lo suficientemente duraderos pra
uso en repetidas conexiones.

El conector de 3.5 mm fue desarrollado inicialmente en Hewlet Packard, ahora Agilent Technologies. Su diseño es
altamente robusto, compatibles con las dimensiones físicas de los conectores SMA. Conector de modo libre hasta 34 GHz. 

Los conectores de precisión de 3.5 mm emplean dieléctrico de aire. El conductor central del macho o hembra es soportado
por arandelas plásticas.


\bigskip

\begin{center}
\tablefirsthead{}
\tablehead{}
\tabletail{}
\tablelasttail{}
\begin{supertabular}{|m{2.952cm}|m{4.323cm}|}
\hline
\centering Tipo de conector &
\centering\arraybslash Torque lb-pulgada (N-cm)\\\hline
\centering Precisión 7 mm &
\centering\arraybslash 12 (136)\\\hline
\centering Precisión 3.5 mm &
\centering\arraybslash 8 (90)\\\hline
\centering SMA * &
\centering\arraybslash 5 (56)\\\hline
\centering Precisión 2.4 mm &
\centering\arraybslash 8 (90)\\\hline
\centering Precisión 1.85 mm &
\centering\arraybslash 8 (90)\\\hline
\centering Tipo-N &
\centering\arraybslash 12 (136)\\\hline
\end{supertabular}
\end{center}
* Usar el valor de torque SMA para conexión de SMA macho con 3.5 mm hembra. Usar el valor de torque 3.5 mm para conexión
3.5 mm macho con SMA hembra.

Tabla \stepcounter{Table}{\theTable}: Valores recomendados de torque para conectores [1]

Adaptadores para puerto de prueba de 3.5 mm a tipo-N (50 Ohm test port adapter) [2]

Parte del kit 11878A de Keysight, empleados para pruebas en dispositivos con conectores de 3.5 mm en equipos de medición
con conectores tipo N. Cada adaptado presenta un conector tipo N en un extremo y un conector de 3.5 mm de precisión en
el extremo opuesto.

Los conectores SMA hacen juego con los conectores de precisión de 3.5 mm. Los conectores SMA no son conectores de
precisión, un conector SMA desgastado o fuera de tolerancia puede dañar a un conector de 3.5 mm incluso destruirlo en
la primera conexión [2]. Se debe inspeccionar con cuidado un conector.

Se deben seguir ciertas precauciones al unir conectores SMA con conectores de precisión de 3.5 mm,

\begin{itemize}
\item Introducirlos de manera recta
\item Asegurar que el pin de contacto macho esta alineado de manera precisa con el pin hembra.
\item No sobre ajustar los conectores.
\item Nunca rotar los conectores por su parte central (al girar el dispositivo).
\item Únicamente girar la rosca externa del conector macho.
\item Usar torque de 5 pulgadas-lb (50 N-cm) para la conexión
\end{itemize}

\bigskip

\begin{center}
\tablefirsthead{}
\tablehead{}
\tabletail{}
\tablelasttail{}
\begin{supertabular}{|m{2.677cm}|m{3.793cm}|}
\hline
\centering Numero de parte &
~
\\\hline
\centering 1250-1744 &
\centering\arraybslash 3.5 mm (m) a tipo-N (m)\\\hline
\centering 1250-1745 &
\centering\arraybslash 3.5 mm (f) a tipo-N (f)\\\hline
\centering 1250-1750 &
\centering\arraybslash 3,5 mm (m) a tipo-N (f)\\\hline
\end{supertabular}
\end{center}
Tabla \stepcounter{Table}{\theTable}: Adaptadores de 3.5 mm a tipo N [2]


\bigskip

\begin{center}
\tablefirsthead{}
\tablehead{}
\tabletail{}
\tablelasttail{}
\begin{supertabular}{|m{2.756cm}|m{5.1660004cm}|}
\hline
\centering Conector &
\centering\arraybslash Rango de frecuencia útil (GHz)\\\hline
\centering Presicion 7 mm &
\centering\arraybslash DC a 20\\\hline
\centering Tipo N &
\centering\arraybslash DC a 18 GHz\\\hline
\centering PSC-N &
\centering\arraybslash DC a 18 GHz\\\hline
\centering SMA &
\centering\arraybslash DC a 23\\\hline
\centering Precision 3.5 mm &
\centering\arraybslash DC a 34\\\hline
\centering PSC-3.5 mm &
\centering\arraybslash DC a 34\\\hline
\centering Precision 2.4 mm &
\centering\arraybslash DC a 50\\\hline
\centering PSC-2.4 mm &
\centering\arraybslash DC a 50\\\hline
\end{supertabular}
\end{center}
Tabla \stepcounter{Table}{\theTable}: Rango de frecuencia útil de diversos tipos de conectores [3]


\bigskip

\subsection{Adaptadores coaxiales de precisión de 3.5 mm 83059}
Conectores coaxiales de precisión \ grado instrumentación de 3.5 mm, ofrecen un sobresaliente desempeño hasta 26.5 GHz.
Con SWR mejor que 1.05. Se emplean como adaptadores entre dispositivos y equipos de medición de alto costo. 



\begin{figure}
\centering
\begin{minipage}{15.656cm}
[Warning: Draw object ignored]Figura 1

Conectores de precisión coaxiales de 3.5 mm
\end{minipage}
\end{figure}

\bigskip

\begin{center}
\tablefirsthead{}
\tablehead{}
\tabletail{}
\tablelasttail{}
\begin{supertabular}{|m{1.2659999cm}|m{2.873cm}|m{2.9329998cm}|m{3.136cm}|m{4.4570003cm}|}
\hline
\centering Modelo &
\centering Tipo de conector &
\centering Frecuencia (GHz) &
\centering Perdida de retorno típica &
\centering\arraybslash Perdida de inserción típica\\\hline
\centering 83059A &
\centering 3.5 mm (m-m) &
\centering DC -26.5 &
\centering {}-32 dB  &
\centering\arraybslash 0.074 dB\\\hline
\centering 83059B &
\centering 3,5 mm (f-f) &
\centering DC -26.5 &
\centering {}-32 dB &
\centering\arraybslash 0.074 dB\\\hline
\end{supertabular}
\end{center}

\bigskip


\bigskip


\bigskip

Bibliografía

[1] AGILENT RF AND MICROWAVE TEST ACCESORIES-AGILENT

[2] 11878A 50 OHM 3.5 MM ADAPTER KIT OPERATING AND SERVICE MANUAL-KEYSIGHT

[3] CONNECTOR CARE FOR \ RF \& MICROWAVE COAXIAL CONNECTORS-HEWLETT PACKARD

[4] 83059 PRECISION 3.5 MM COAXIAL ADAPTERS DC 26.5GHZ TECHNICAL OVERVIEW-KEYSIGHT


\bigskip

\section[Referencias bibliográficas]{Referencias bibliográficas}
[1] AGILENT N2002A NOISE SOURCE TEST SET 10MHZ TO 26.5GHZ

[2] NOISE SOURCE CALIBRATION USING THE AGILENT N8975A NOISE FIGURE ANALYZER AND THE N2002A NOISE SOURCE TEST SET-AGILENT


\bigskip
\end{document}
