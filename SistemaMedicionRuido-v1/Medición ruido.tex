\documentclass{article}
\usepackage[left=3cm,right=3cm,top=3.5cm,bottom=2cm,
bindingoffset=5mm]{geometry}
\usepackage[utf8]{inputenc}
\usepackage[T1]{fontenc}
% Fuente escalable
\usepackage{lmodern}
\usepackage[english,spanish]{babel}
\usepackage{amsmath}
\usepackage{amssymb,amsfonts,textcomp}
\usepackage{array}
\usepackage{supertabular}
\usepackage{booktabs}
\usepackage{threeparttable}
\usepackage{hhline}
\usepackage[pdftex]{graphicx}
\usepackage{microtype}
% Bibliografia
%\usepackage{bibtex}
\usepackage{biblatex}
\bibliographystyle{amsplain}
\bibliography{Bibliografia.bib}
% Formato de unidades
\usepackage{siunitx}
% Graficos
\usepackage{tikz}
% Esquematicos
\usepackage[siunitx, american, cuteinductors, smartlabels]{circuitikz}
% Tablas con ancho establecido por usuario
\usepackage{tabularx}
% Agrega comandos extra al comando tabular
% \toprule, \midrule, \bottomrule
\usepackage{booktabs}
% Permite unir varias filas en tablas
\usepackage{multirow}
% Encabezados personalizados
\usepackage{fancyhdr}
\usepackage{graphicx}

% Para graficos XY
\usepackage{pgfplots}

\makeatletter
\newcommand\arraybslash{\let\\\@arraycr}
\makeatother
\setlength\tabcolsep{1mm}
\renewcommand\arraystretch{1.3}
\newcounter{Table}
\renewcommand\theTable{\arabic{Table}}
\newcounter{Drawing}
\renewcommand\theDrawing{\arabic{Drawing}}
\title{}
\author{}
\date{2017-03-28}

% Cabeceras
\pagestyle{fancy}
% Borra cabecera y pie actuales
\fancyhf{}
% Cintillo cabecera
\chead{
	\includegraphics[width=150mm]{Imagenes/Cabecera.png}
}

\begin{document}
	
	\section{Ruido eléctrico}
		El ruido eléctrico se manifiesta por pequeñas fluctuaciones de corriente y de voltaje en los dispositivos electrónicos. Son múltiples las orígenes o causas del ruido eléctrico, pero todas ellas se pueden agrupar en dos grandes categorías:
		
		\begin{itemize} 
			\item Ruido extrínseco
			\item Ruido intrínseco
		\end{itemize}	
	
		El \emph{ruido extrínseco} es producto de la recepción de señales indeseables generadas en el exterior del dispositivo, y se manifiesta como una perturbación en la repuesta de corriente o voltaje del mismo. Este ruido puede se generado por el hombre cuando tiene su origen en la radiación de artefactos eléctricos o electrónicos. También es producto de fenómenos naturales como lo son relámpagos, tormentas eléctricas, ruido intergaláctico o disturbios atmosféricos en general.
		
		El \emph{ruido intrínseco} es generado el interior del dispositivo, no es producto de la detección de señales externas. Existen diversos tipos de ruido intrínseco mencionados en la literatura, como lo son el \emph{ruido de metralla}, el \emph{ruido de centelleo}, el \emph{ruido ráfaga} y el \emph{ruido térmico}. Este tipo de ruido se debe básicamente al hecho de que la carga eléctrica no es continua, sino que es llevada en cantidades discretas iguales a la carga del electrón.
			
		\subsection{Ruido térmico}
		
		El \emph{ruido térmico} encuentra su origen en el movimiento y colisiones aleatorias de los portadores de carga dentro de los materiales conductores o semiconductores, es uno de los tipos de ruido de mayor importancia en sistemas RF y de microondas. Al ser producto del movimiento térmico de los electrones, guarda relación con la temperatura absoluta T: el ruido térmico es directamente proporcional a T, conforme T se acerque a cero, el ruido térmico también se acercará a cero. Es generado por cualquier elemento de circuito que presente perdidas de energía, como lo son resistores o líneas de transmisión. 
		
		Existen otras fuentes de ruido, externas a un dispositivo, generadas por la absorción atmosférica y la radiación cósmica de fondo, que puede considerarse de origen térmico ya que es producto del movimiento aleatorio de partículas cargadas.
		
		\subsubsection{Voltaje y potencia de ruido}
		
		Todo dispositivo o elemento de circuito que presente perdidas genera ruido térmico. En el caso de un elemento de circuito básico, el resistor, el movimiento aleatorio de los electrones con una energía cinética proporcional a la temperatura se manifiesta como ligeras fluctuaciones de voltaje a través de las terminales del resistor, como se ilustra en la figura \ref{F:RESISTOR_RUIDO}.		
		
		\begin{figure}
			\centering
			
			\begin{circuitikz}
				\draw 
					(0, 0) to[R, l_=$R$] (0, 2)
					to [short, -o] (2, 2)
					to [open, v^=$v_{n}(t)$] (2, 0)
					to [short, o-] (0, 0);
			\end{circuitikz}
			\caption{Fuente de ruido térmico}
			\label{F:RESISTOR_RUIDO}
			
			\begin{tikzpicture}[>=stealth]
				\begin{axis}[
					xmin=-4,xmax=4,
					ymin=-2,ymax=2,
					axis x line=middle,
					axis y line=middle,
					axis line style=<->,
					xlabel={$x$},
					ylabel={$y$},
					]
					\addplot[no marks,blue,<->] expression[domain=-pi:pi,samples=100]{sin(deg(2*x))+1/2} 
					node[pos=0.65,anchor=south west]{$y=\sin(2x)+\frac{1}{2}$}; 
				\end{axis}
			\end{tikzpicture}
			
		\end{figure}	
		
		El valor instantáneo de $v_{n}(t)$ de no puede ser precisado debido a su naturaleza estocástico
		tica, pero puede ser modelado por sus valores estadísticos. El valor medio de ésta tensión es cero y su valor cuadrático medio, en un ancho de banda estrecho $B$, viene dado por la aproximación de \emph{Rayleight-Jeans}. Este puede ser modelado con una fuente de voltaje en serie o una fuente de corriente en paralelo con los siguientes valores cuadráticos medios,  ecuaciones \eqref{E:VOLTAGE_RUIDO} y \eqref{E:CORRIENTE_RUIDO},	
		
		\begin{equation}
		\label{E:VOLTAGE_RUIDO}	
		\bar{v^2_{N}} = 4{\kappa}TBR{\Delta}f		
		\end{equation}
		
		\begin{equation}
		\label{E:CORRIENTE_RUIDO}
		\bar{i^2_{N}} = 4{\kappa}T{\Delta}f
		\text{$\quad{\kappa}=1.380\dot10^{-23} \, {\SI{}\joule}{\SI{}\kelvin}$} 	
		\end{equation}
		
		\noindent donde $R$ es la resistencia en \SI{}{\ohm}, ${\Delta}f$ es el ancho de banda en \SI{}{\hertz} sobre el cual el ruido es medido. El ruido térmico en un resistor es \emph{independiente} de su composición.
		
		\subsection{Densidad espectral}
		
		La potencia de ruido por unidad de ancho de banda $S$ que un resistor en equilibrio térmico a una una temperatura T entrega a través de sus terminales viene dada por. La desidad de potencia de ruido para el modelo Thevenin del resistor viene dada por
		
		\begin{equation}
		\label{E:DENSIDAD_POTENCIA_VOLTAGE}
		S_{v}(f) = 2{\kappa}TR
		\end{equation}
		
		\noindent y la densidad de potencia de ruido para el modelo Norton del resistor resulta	
		
		\begin{equation}
		\label{E:DENSIDAD_POTENCIA_CORRIENTE}
		S_{i}(f) = 2{\kappa}TG
		\end{equation}	
		
		\noindent donde $\kappa$ es la constante de Boltzmann, G es la conductancia del resistor en Siemens ($\SI{}\siemens$) y $R$ su resistencia en Ohms ($\SI{}\ohm$).	
		
		La potencia total de ruido $P$ en $Watts$ en un ancho de banda bilateral, desde -B hasta +B $\SI{}\hertz$ esta dada por
		
		\begin{equation}
		\label{E:POTENCIA_TOTAL_1}
		P = 2S_{v}(f)B
		\end{equation}
		
		\noindent sustituyendo \eqref{E:DENSIDAD_POTENCIA_VOLTAGE} en \eqref{E:POTENCIA_TOTAL_1} se obtiene
		
		\begin{equation}
		\label{E:POTENCIA_TOTAL_2}
		P = 4{\kappa}TRB
		\end{equation}	
		
		\subsection{Máxima potencia disponible}
		
		
		Tensión en la carga 
		\begin{equation}
		\label{E:VOLTAJE_MAXIMO}
		v_{R} =    \frac{R}{2} \text{\quad\SI{}\volt \, rms}	
		\end{equation}
		La potencia en la carga resulta la máxima potencia disponible
		\begin{equation}
		\label{E:VOLTAJE_MAXIMO}
		P_{n} = \frac{v^2_{R}}{4R} = {\kappa}TB \text{\quad\SI{}\watt}	
		\end{equation}	
		
		
		\section{Tempertura de ruido equivalente}	
		
		\begin{figure}
		\centering		
			\begin{circuitikz}
				\draw
					(0, 0) rectangle ++(2, 2)
					(1, 1) node[below] {Red ruidosa}	
					(2, 1.8) to[short, -o] (3, 1.8)
					to[short,  i=$N_{O}$] (4, 1.8)
					to[R, l^=$R$] (4, 0.2)
					to[short] (3, 0.2)
					to[short, o-] (2, 0.2);
			\end{circuitikz}
		\caption{Fuente de ruido termico}
		\label{F:FUENTE_RUIDO}
		\end{figure}
		
		\begin{figure}
		\centering
			\begin{circuitikz}	
				\draw[dotted]
					(0, 0) rectangle ++(2, 2);	
				\draw
					(1, 0.2) to[R, l_=$R$] (1, 1.8)
					to[short, -o] (3, 1.8)
					to[short,  i=$N_{O}$] (4, 1.8)
					to[R, l^=$R(T_E)$] (4, 0.2)
					to[short, -o] (3, 0.2)
					to[short] (1, 0.2);
			\end{circuitikz}	
		\caption{Red ruidosa remplazada por resistor a temperatura equivalente}
		\label{F:RED_SIN_RUIDO}
		\end{figure}
		
		\begin{equation}
		T_{E} = \frac{N_{0}}{{\kappa}B}	\text{\quad \SI{}\kelvin}
		\label{E:TEMPERATURA_EQUIVALENTE_RUIDO}
		\end{equation}
		
		
		%	\begin{circuitikz} [american voltages, baseline=(current bounding box.center)]
		%		\ctikzset {label/align = straight}
		%		\draw (0, 0)
		%		to[V=$V_{Th}$]  (0, 2)
		%		to[R=$R_{Th}$] (2.5, 2)
		%		to[short, i=$I$, -o] (4, 2)
		%		to[short] (4.5, 2)
		%		(0,0) to[short, -o] (4, 0)
		%		to[short] (4.5, 0)
		%		node[draw, minimum width=2cm, minimum height=2.4cm, anchor=south west] at (4.5, -0.2){Load};
		%	\end{circuitikz}
		%
		%	\begin{circuitikz} [american voltages, baseline=(current bounding box.center)]
		%		\draw
		%		(0, 0) to [V=$V_{n}$] (0, 3)
		%		to[R=$R_{n}$] (3, 3)
		%		to[short, i_<=$I$, -o] (6, 3)
		%		to[short] (9, 3)
		%		(0, 0) to[short, -o] (6, 0)
		%		to[short] (9, 0)
		%		node[draw, minimum width=3cm, minimum height=4cm, anchor=south west] at (9, -0.4){Source};
		%	\end{circuitikz}
		%
		%	\begin{circuitikz}
		%		\draw
		%		(0, 0) node[mixer] (m) {}
		%		(m.1) to[short, -o] ++(-1, 0)
		%		(m.2) to[short, -o] ++(0, -1)
		%		(m.3) to[short, -o] ++(1, 0);
		%		(m.1) node[inputarrow] {}
		%		(m.2) node[inputarrow, rotate=90] {};
		%	\end{circuitikz}
		%
		%
		%
		%	\begin{tikzpicture}		
		%		\coordinate (SolarCellP) at (0.5, 0);
		%		\coordinate (ChopperP) at (4, 0);
		%		\coordinate (InverterP) at (8, 0);
		%		
		%		\begin{scope}[every node/.style={draw,ultra thick, fill=AlleeRed!30, rectangle,text width=1.5cm, minimum height=2cm, text badly centered}]
		%		
		%			\node[name=pv, pattern=grid, pattern color=AlleeRed!30]	at (SolarCellP) {\bfseries Solar Panel};	
		%			\node[name=chopper] at (ChopperP){DC-DC};
		%			\node[name=inverter] at (InverterP){DC-AC};
		%
		%		\end{scope}
		%
		%		\draw
		%			% PV to chopper
		%			(pv.30) -- (chopper.150)
		%			(pv.330) -- (chopper.210)
		%			% DC link and cap
		%			(chopper.30) -- (inverter.150)
		%			(chopper.330) -- (inverter.210)
		%			($(chopper.330)!0.4!(inverter.210)$) to[battery, l=\SI{60}{\volt}, mirror] ($(chopper.30)!0.4!(inverter.150)$)
		%   			%AC side
		%			(inverter.30)
		%			to[short] ++(1, 0)
		%			to[L] ++(1.1, 0)
		%			to[R] ++(1.1, 0)
		%			to[short] ++(1, 0) coordinate(N1)
		%			to[L, bipoles/length=0.8cm] ++(0, -1)
		%			|- (inverter.330);
		%
		%	\end{tikzpicture}
		%
		%	\begin{tikzpicture}
		%		\draw
		%			 plot coordinates{(0,0) (6,6)}
		%			[->] (0, -3.5) node[below]{$M$} -- (0, 2) node[above, scale = 1.3]{$\varepsilon$};
		%			\path (0, 0) node[draw, shape=rectangle](v0) {$v_0$};
		%	\end{tikzpicture}
		%
		%	\begin{tikzpicture}[scale=2]
		%		\tikzstyle{every node} = [draw, shape=rectangle];
		%		\path(0:0cm)			node (v0) {$v_0$};
		%		\path(0:1cm)			node (v1) {$v_1$};
		%		\path(72:1cm)		node (v2) {$v_2$};
		%		\path(2*72:1cm)		node (v3) {$v_3$};
		%		\path(3*72:1cm)		node (v4) {$v_4$};
		%		\path(4*72:1cm)		node (v5) {$v_4$};
		%
		%		\draw [->]  (v0) -- (v1)
		%			[<-] (v0) -- (v2)
		%			[->] (v0) -- (v3)
		%			[<-] (v0) -- (v4)
		%			[->] (v0) -- (v5);
		%			
		%	\end{tikzpicture}
		%
		%		\begin{tikzpicture}
		%			\draw (0,0) 	rectangle ++(2, 2) (1, 1) node[below] {$sin(x)$}
		%				(1, 3)	rectangle ++(3, 2) node[below] {$cos(x)$}
		%					rectangle (4, 3) node[below] {$tan(x)$};		
		%		\end{tikzpicture}	
		%
		
		\section{Potencia de ruido disponible desde el DUT}
		
		\begin{figure}
		\centering
		\begin{circuitikz}
			\ctikzset{bipoles/amp/width=1.2}
		\draw 
				(0, 0) to[sV, l=$2{\kappa}T_{S}B\SI{}{\watt}$] (0, 2)				
				(0, 2) to[generic, l=$Z_{0}$, -o] (2, 2)
				to[amp, t=DUT, -o] (6, 2) 
				to[short] (7, 2)
				to[generic, l=$Z_{0}$] (7, 0)
				to[short] (6, 0)
				to[short, o-o] (2, 0)
				to[short] (0, 0)
				node[ground] {};
		
			\draw 
				(2.5, 1.8) node[below] {${\kappa}T_{S}B$}	
				(6.5, 1.5) node[below] {$N_{O}$};
		\end{circuitikz}
		\caption{Respuesta al ruido DUT}\label{F:RESPUESTA_RUIDO_DUT} 
		\end{figure}
		
		\begin{equation}
		N_O = {\kappa}BGT_S + N_A
		\label{E:RUIDO_SALIDA_DUT}
		\end{equation}
		
		\begin{figure}
		\centering
		\begin{circuitikz}
			\draw [->] (-0.5, 0) node[anchor=east] {$0$} -- (4, 0) node[anchor=west]{$T_{S}(\SI{}{\kelvin})$};
			\draw	[->] (0, -0.5) node[below]{$0$} -- (0, 4) node[above]{$N_{O}(\SI{}{\watt})$};
			\draw [thick] (0, 0.5) -- (3.5, 3.2);
			\draw  [dotted]  
				(1.5, 1.7) -- (2, 1.7)
				(2, 1.7) -- (2, 2.1);
			\draw 
				(2, 1.7) node[right]  {Pendiente ${\kappa}T_{S}BG$}
				(0, 0.5) node[left] {$N_A$};
			\draw [dotted]
				(1.5, 0) -- (1.5, 1.7)
				(2, 0) -- (2, 1.7);
			\draw 
				(1.5, -0.1) node[below]{$T_{C}$}
				(2, -0.1) node[below] {$T_{H}$};
			\draw	[dotted]
				(0, 1.7) -- (1.5, 1.7)
				(0, 2.1) -- (2, 2.1);
			\draw 
				(-0.1, 1.7) node[left] {$N_{C}$}
				(-0.1, 2.1) node[left] {$N_{H}$};			
		\end{circuitikz}
		\caption{Potencia disponible del DUT}
		\end{figure}
		
		
		\begin{figure}
		\centering
		\begin{circuitikz}
			\draw 
				(0, 0) to[R, l=${\kappa}T_{S}$] (0, 4)
				(0, 4) to[short, -o] (1, 4)
				(1, 4) to[amp, t=$A_{1}$] (4, 4)
				(4, 4) to[amp, t=$A_{2}$] (7, 4)
				(7, 4) to[amp, t=$A_{3}$, -o] (10, 4)
				(10, 4) to[short] (12, 4)
				(12, 4) to[R, l=$R_{L}$] (12, 0)
				(12, 0) to[short] (11, 0)
				(11, 0) to[short, o-o] (1, 0)
				(1, 0) to[short] (0, 0)
				node[ground] {};
		
			\draw
				(2.5, 4.5) node[above] {$N_{A1}, G_{1}, B$}
				(5.5, 4.5) node[above] {$N_{A2}, G_{2}, B$}
				(8.5, 4.5) node[above] {$N_{A3}, G_{3}, B$};
		
			\draw
				(1, 2.8) node[below] {${\kappa}T_{O}B$}
		
				(4, 2.8) node[below] {${\kappa}T_{O}BG_{1}$}
				(4, 2.3) node[below] {$N_{A1}$}
			
				(7, 2.8) node[below] {${\kappa}T_{O}BG_{1}G_{2}$}
				(7, 2.3) node[below] {$BG_{2}N_{A1}$}
				(7, 1.8) node[below] {$N_{A2}$}
		
				(10, 2.8) node[below] {${\kappa}T_{O}BG_1{1}G_{2}G_{3}$}
				(10, 2.3) node[below] {$BG_{3}G_{2}G_{1}N_{A1}$}
				(10, 1.8) node[below] {$BG_{3}G_{2}N_{A2}$}
				(10, 1.3) node[below] {$N_{A3}$};
		
		\end{circuitikz}
		\caption{Dispositivos en cascada}\label{F:AMPLIFICADORES_CASCADA} 
		\end{figure}
		
		
		%	\begin{circuitikz}
		%		\draw	(0, 0) to[amp, l=\SI{10}{dB}] ++(2.5, 0);
		%		\draw	(2.5, 0) to[lowpass, l=opt.] ++(2.5, 0);			
		%	\end{circuitikz}
		
		
		%	\begin{figure}
		%		\centering
		%		\begin{circuitikz}
		%			\ctikzset{bipoles/amp/width=0.9}
		%			\draw (0, 0) to[amp, t=LNA, l_=$F{=}0.9\,$dB, o-o] ++(3, 0);
		%		\end{circuitikz}
		%		\caption{Un LNA.} \label{F:LNA}
		%	\end{figure}
		
		\begin{figure}
		\centering
		\begin{circuitikz}
			\ctikzset{bipoles/amp/width=1.5}
			\draw
				(0, 0) to[amp, t=$DUT_{k}$, o-o] (4, 0)
				to[amp, t=$DUT_{k+1}$, o-o] (8, 0);
				
		\end{circuitikz}
		\end{figure}
		
		\begin{figure}
		\centering
		\begin{circuitikz}[american voltages]
			\draw
				(0, 0) to[open] (0, 4)
				to[short, o-] (1, 4)
				to[generic, l_=$jX_{11, k}$] (1, 2)
				to[R, l_=$R_{11,k}$, v^>=$e_{k}$] (1, 0)
				node[ground]{}
		
				(2, 0) node[ground]{}
				to[sV_<=$a'_{k}e_{k}$] (2, 4)
				to[R, l_=$R_{22, k}$] (4, 4)
				to[generic, l_=$jX{22, k}$] (6, 4)
				to[short, -o] (7, 4)
				to[short] (8, 4)
				to[generic, l_=$jX_{11, k + 1}$] (8, 2)
				to[R, l_=$R_{11, k + 1}$, v^>=$e_{k + 1}$] (8, 0)
				node[ground] {}		
				
				(9, 0) node[ground]{}
				to[sV_<=$a'_{k+1}e_{k+1}$] (9, 4)
				to[R, l_=$R_{22, k + 1}$] (11, 4)
				to[generic, l_=$jX_{22, k + 1}$, -o] (13, 4);				
			
		\end{circuitikz}
		\caption{Dispositivo activo, lineal, unidireccional y su modelo equivalente}
		\label{F:MODELO_EQUIVALENTE_DUT}
		
		\end{figure}	

\end{document}