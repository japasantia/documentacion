% Para documento texto corto
\documentclass[paper=letter,oneside,fontsize=12pt, parskip=full]{article}
%\documentclass[paper=letter,oneside,fontsize=11pt, parskip=full]{scrartcl}
%\documentclass{amsart}
%\documentclass[paper=letter,oneside,fontsize=12pt]{scrartcl}

% Establece dimensiones de los margenes
% \usepackage[inner=1.5cm,outer=3cm,top=2cm,bottom=4cm,
% bindingoffset=5mm]{geometry}
\usepackage[left=3cm,right=3cm,top=3cm,bottom=3cm,
bindingoffset=0cm, footskip=0.5cm, headheight=2cm]{geometry}

% Carga babel, idioma ingles
\usepackage[english, spanish]{babel}

% Elimina sangrias y aumenta espacio entre parrafos
\usepackage{parskip}

% Permite ingresar caracteres acentuados y especiales 
% sin necesidad de emplear comando
% utf8 codificacion de entrada Unicode (mas simbolos que ASCII)
\usepackage[utf8]{inputenc}

% T1 encoding for European, English, American text
\usepackage[T1]{fontenc}
% Fuente escalable
\usepackage{lmodern}

% Agrega comandos extra al comando tabular
% \toprule, \midrule, \bottomrule
\usepackage{booktabs}
% Unir filas
\usepackage{multirow}
% Tablas con ancho establecido por usuario
\usepackage{tabularx}
% Para posicionamiento preciso de tablas dentro del texto

\usepackage{float}

% Establece espacio entre lineas
\usepackage[onehalfspacing]{setspace}

% Tabla de tres secciones
\usepackage[flushleft]{threeparttable}

%Para resizebox
\usepackage{graphicx}

%Uso de colores
\usepackage{xcolor}
% Permite controlar colores de tablas
\usepackage{xcolor, colortbl}


% Definicion colores tabla cronograma
\definecolor{colorta}{rgb}{0.3569,0.608,0.8353}
\definecolor{colortb}{rgb}{0.4392,0.678,0.2784}
\definecolor{colortc}{rgb}{1.0000,0.361,0.0000}
\definecolor{colortd}{rgb}{0.1804,0.455,0.7098}
\definecolor{colorte}{rgb}{0.9294,0.490,0.1922}
\definecolor{colortf}{rgb}{0.2667,0.3290,0.4157}
\definecolor{colortg}{rgb}{0.3290,0.2667,0.4157}
\definecolor{colortj}{rgb}{0.2667,0.4157,0.329}

\newcommand{\TA}{\cellcolor{colorta}}
\newcommand{\TB}{\cellcolor{colortb}}
\newcommand{\TC}{\cellcolor{colortc}}
\newcommand{\TD}{\cellcolor{colortd}}
\newcommand{\TE}{\cellcolor{colorte}}
\newcommand{\TF}{\cellcolor{colortf}}
\newcommand{\TG}{\cellcolor{colortg}}
\newcommand{\TJ}{\cellcolor{colortj}}

\begin{document}


	\begin{flushright}		

		Caracas, 09 de septiembre de 2015. \\
	
	\end{flushright}

	
	\begin{onehalfspace}
		\large
		Prof. Ebert Brea. \\
		Jefe del Departamento de Electrónica, Computación y Control. \\
		Escuela de Ingeniería Eléctrica. \\
		Universidad Central de Venezuela.	
	\end{onehalfspace}

	Presente.- 
	
	Tengo el agrado de dirigirme a Usted, en la oportunidad de hacerle llegar un cordial saludo y a la vez solicitarle la extensión del lapso de ejecución de mi Trabajo Especial de Grado y el replanteamiento del alcance de los objetivos del TEG.
	 
	Soy José Aquilino Arias Bustamante, C.I. V-14.666.744, y desde el 6 de marzo del presente me encuentro realizando el TEG dentro de la fundación CENDIT. El lapso asignado de 28 semanas, aprobado por Consejo de Escuela para la ejecución del TEG, se agotó sin que haya sido posible culminar el proyecto. Diversas han sido las causas que han impedido el alcance de los objetivos en el tiempo estipulado, algunas se citan más adelante. 
		
	Por este motivo le solicito la extensión del tiempo de ejecución de TEG por 18 semanas adicionales, que equivalen al tiempo que resta para que concluya el semestre 20017-1.
	
	Seguidamente le presento la reformulación de objetivos del proyecto TEG, los cuales inicialmente eran los siguientes:	
		
	\emph{Título del proyecto}
	
	\begin{center}
		DISEÑO DE UN EQUIPO ELECTRÓNICO CONTROLADOR DE
		INTERRUPTORES Y ATENUADORES EMPLEADO EN LA
		MEDICIÓN DE LA FIGURA DE RUIDO EN DISPOSITIVOS DE
		RADIO FRECUENCIA
	\end{center}	
	
	\emph{Objetivo general}
	
	Diseñar un equipo electrónico que permita emular las características funcionales de un
	controlador electrónico de interruptores y atenuadores.
	
	\emph{Objetivos específicos}
	
	\begin{enumerate}
		\item Elaborar un informe técnico a partir de un estudio del funcionamiento de los
		dispositivos presentes en el sistema de medición de figura de ruido de la fundación
		CENDIT.
		
		\item Diseñar un equipo electrónico que permita replicar las características funcionales de
		una unidad de control para atenuadores e interruptores de la serie 11713 de KeySight.
		
		\item Implementar el diseño como dispositivo físico.
		
		\item Integrar el diseño físico en el banco de medición de figura de ruido presente en el
		laboratorio del CENDIT.
		
		\item Generar un manual de usuario para el dispositivo diseñado.
	\end{enumerate}	

	\emph{Cambios en el alcance}
	
	\emph{Cambios en el diseño de hardware}

	En el diseño original para el dispositvo electrónico, se reducen sus  interfaces de comunicaciones, de tres que se habían propuesto inicialmente (GPIB, USB, LAN) a dos interfaces (USB y GPIB). De esta forma, se diseñará un dispositivo con todo el soporte requerido para las siguientes interfaces
	
	\begin{enumerate}
		\item USB (Class Device Communication, full speed).		
		\item LAN (Ethernet 100 Mbps BASE-T)
	\end{enumerate}

	La interfaz eléctrica del dispositivo,  que consiste en las señales eléctricas de control que el dispositivo para comandar la unidad de atenuadores y aisladores estarán constituidas por dos grupos de 16 señales cada uno. 
	
	El panel frontal consistirá de un teclado matricial, de no más de 16 teclas. 
	
	\emph{Cambios en el diseño de software}
	
	La aplicación a desarrollar, conocida como Software para Gestion de la Medición de Figura de Ruido (SGMFR) consistirá, a grandes rasgos, de lo siguiente.
	
	\begin{enumerate}
		\item Instalador para la aplicación
		\item Soporte para establecer comunicación de datos con los dispositivos del sistema de medición de figura de ruido.
		\item Interfaz de usuario gráfica.
		\item Asistencia al usuario en las etapas de medición: configuración, ejecución y generación de reportes.
		\item Generación de reportes de resultados de medición, en formato pdf.
	\end{enumerate}

	\emph{Cambios en el diseño de firmware}
	
	El firmware que se desarrollará para el microcontrolador central brindará soporte a las comunicaciones por medio de las interfaces USB y LAN/GPIB. Se encargará de gestionar la interacción del usuario con el panel frontal. Se encargara de controlar el valor de la tensión de alimentación en los puertos de señal Viking, la cual es programable por el usuario.		
	
	
	\emph{Nuevo cronograma de trabajo}

	\begin{center}
		\begin{threeparttable}[!h]	
			\arrayrulecolor{gray}
			\setlength{\extrarowheight}{4pt}		
		
				\begin{tabular}{|c|c|l|l|l|l|l|l|l|l|l|l|l|l|}
					\hline 		
					\textbf{Item} &	
					\begin{tabular}{c}
						{\raggedright \textbf{Semanas}} \\
						\textbf{Tareas}
					\end{tabular}
					 & 1 & 2 & 3 & 4 & 5 & 6 & 7 & 8 & 9 & 10 & 11 & 12 \\
					\hline
					1 & Expansor de puertos Viking & \TA & \TA & \TA & \TA & & & & & & & & \\
					\hline
					2 & Tarjeta interfaz de usuario  & & \TB & & & & & & & & & & \\
					\hline
					3 & Firmware del dispositivo & & & \TC & \TC & \TC & \TC & \TC & \TC & & & & \\ 
					\hline
					4 & Tarjeta madre & & & & & & & \TD & \TD & \TD & \TD & & \\
					\hline
					5 & Tarjeta de alimentación DC  &  & \TE & \TE & \TE & \TE & & & & & & & \\
					\hline
					6 & Aplicación SGMFR  & \TG & \TG & \TG & \TG & \TG & \TG & \TG & \TG & & & & \\
					\hline
					7 & Libro TEG   &  &  &  & \TJ & \TJ & \TJ & \TJ & \TJ & \TJ & \TJ & \TJ & \\
					\hline
				\end{tabular}	
				\begin{tablenotes}
					\small
					\item Fecha de inicio: 6 de Marzo de 2017. 
					\item Jornada de 8 horas diarias, lunes a viernes 8:00 AM a 12:00 M y 1:30 PM a 4:30 PM.
				\end{tablenotes}
			\caption{Cronograma de actividades inicial}			
			\label{Tab:CronogramaActividadesInicial}
		\end{threeparttable}
	\end{center}

	\begin{table}[h!]
		\resizebox{\linewidth}{!}{%
		\begin{tabular}{cllllc}
			\toprule
			\bfseries Item & \bfseries  Tarea & \bfseries  Entrada & \bfseries  Proceso & \bfseries  Salida & \bfseries  Semanas \\
			\midrule
			1 & Expansor puertos Viking & 
			\begin{tabular}{l}
				Materiales desarrollo PCB \\
				Circuitos integrados \\
				Aplicaciones EDA (KiCad, Eagle, Proteus)
			\end{tabular} &			
			\begin{tabular}{l}
				Diseño esquemáticos \\
				Revisión esquemáticos \\
				Selección de componentes \\
				Simulación \\
				Elaboración PCB \\
				Soldadura de componentes \\
				Depuración tarjeta individual \\
				Integración con tarjeta madre
			\end{tabular} &
			\begin{tabular}{l}
				Tarjeta en PCB				
			\end{tabular} &
			4 \\
			
			\midrule			
			1 & Expansor puertos Viking & 
			\begin{tabular}{l}
			Materiales desarrollo PCB \\
			Circuitos integrados \\
			Aplicaciones EDA (KiCad, Eagle, Proteus)
			\end{tabular} &			
			\begin{tabular}{l}
			Diseño esquemáticos \\
			Revisión esquemáticos \\
			Selección de componentes \\
			Simulación \\
			Elaboración PCB \\
			Soldadura de componentes \\
			Depuración tarjeta individual \\
			Integración con tarjeta madre
			\end{tabular} &
			\begin{tabular}{l}
				Tarjeta en PCB				
			\end{tabular} &
			4 \\	
			
			\midrule			
			2 & Tarjeta interfaz usuario & 
			\begin{tabular}{l}
				Materiales desarrollo PCB \\
				Elementos mecánicos (botones, cables, tornillos) \\
				Aplicaciones EDA (KiCad, Eagle) \\
			\end{tabular} &			
			\begin{tabular}{l}
				Diseño esquemáticos \\
				Revisión esquemáticos \\
				Selección de componentes \\
				Elaboración PCB \\
				Soldadura de componentes \\
				Pruebas de tarjeta individual \\
				Integración con tarjeta madre \\
			\end{tabular} &
			\begin{tabular}{l}
				Tarjeta de interfaz de usuario \\
				Teclado matricial.		
			\end{tabular} &
			1 \\			
			
			\midrule			
			3 & Firmware del dispositivo & 
			\begin{tabular}{l}
				Computador PC \\
				Tarjeta madre prototipo \\
				Aplicaciones IDE (MPLAB-X) \\
				Aplicaciones simulación electrónica (Proteus) \\
			\end{tabular} &			
			\begin{tabular}{l}
				Identificación de componentes \\
				Modelado de componentes \\
				Codificación \\
				Carga de firmware \\
				Pruebas aisladas \\
				Pruebas con periféricos individuales \\
				Integración \\
				Pruebas finales \\
			\end{tabular} &
			\begin{tabular}{l}
				Firmware controlador del dispositivo Cendit11713
			\end{tabular} &
			6 \\			

			\midrule			
			4 & Tarjeta madre & 
			\begin{tabular}{l}
				Materiales desarrollo PCB \\
				Circuitos integrados \\
				Elementos pasivos (resistores, capacitores) \\
				Elementos mecánicos (conectores de puertos, retenedores, tornillos) \\
			Carcasa metálica \\
			Aplicaciones EDA (KiCad, Eagle) 
			\end{tabular} &			
			\begin{tabular}{l}
				Diseño de esquemáticos \\
				Revisión de esquemáticos \\
				Selección de componentes \\
				Simulación \\
				Elaboración PCB \\
				Soldadura de componentes \\
				Carga de firmware \\
				Pruebas de tarjeta individual \\
				Integración con periféricos
			\end{tabular} &
			\begin{tabular}{l}
				Tarjeta madre en PCB		
			\end{tabular} &
			4 \\			
			
			\midrule			
			5 & Tarjeta alimentación DC
			 & 
			\begin{tabular}{l}
				Materiales desarrollo PCB \\
				Circuitos integrados \\
				Elementos pasivos (resistores, capacitores) \\
				Elementos mecánicos (conectores de puertos, retenedores, tornillos) \\
				Cables \\
				Aplicaciones EDA (KiCad, Eagle, Spice) 
			\end{tabular} &			
			\begin{tabular}{l}
				Diseño de esquemáticos \\
				Revisión de esquemáticos \\
				Selección de componentes \\
				Simulación \\
				Elaboración PCB \\
				Soldadura de componentes \\
				Pruebas de tarjeta individual 
			\end{tabular} &
			\begin{tabular}{l}
				Tarjeta en PCB	
			\end{tabular} &
			5 \\			
				
			\midrule				
			6 & Aplicación SGMFR & 
			\begin{tabular}{l}
				Computador PC \\
				Acceso a internet \\
				Bibliografía \\
				Aplicaciones IDE (JIDEA) \\
				JDK (librerías Java) \\
				Librerías comunicaciones con instrumentos 
			\end{tabular} &			
			\begin{tabular}{l}
				Identificación de componentes \\
				Modelado de componentes \\
				Codificación \\
				Pruebas \\
				Integración 
			\end{tabular} &
			\begin{tabular}{l}
				Aplicación funcional e instalador
			\end{tabular} &
			8 \\		
			
			\midrule			
			7 & Documentación
			&
			\begin{tabular}{l}
				Computador PC \\
				Acceso a internet \\
				Bibliografía \\
				Distribución de \LaTeX \\
				Editor de \LaTeX \\
			\end{tabular} &			
			\begin{tabular}{l}
				Recopilación de documentos \\
				Lectura \\
				Toma de notas \\
				Organización de notas \\
				Escritura de libro TEG \\
				Escritura de informes
			\end{tabular} &
			\begin{tabular}{l}
				Libro de TEG \\
				Informe técnico \\
				Instrucciones de trabajo 
			\end{tabular} &
			8 \\							
			\bottomrule
		\end{tabular}%
	}
	\caption{Descripción de actividades del cuadro \ref{Tab:CronogramaActividadesInicial}}
	\end{table}

	Sin más a que hacer referencia y agradecido por su atención, se despide de Ud.	
	
	\begin{flushright}
		Atentamente, 		
		
		\vspace{2cm}
		
		\begin{singlespace}
			\large
			Jose Arias \\
			{
				\small
				C.I. 14.666.744 \\
				correo@josearias.com.ve \\			
			}	
		\end{singlespace}
	
	\end{flushright}


\end{document}