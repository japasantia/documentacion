\subsection{Conectores}
El rango de frecuencia de cualquier conector esta limitado por el primer modo de propagación de la guía de onda circular en la estructura coaxial. Disminuir el diámetro del conductor externo incrementa la más alta frecuencia usable. Rellenar el espacio de aire con dieléctrico disminuye la frecuencia más alta usable.

Si las dimensiones mecánicas de los conectores no esta apareada, si el enchapado/plating no es el adecuado, o si la separación de contactos en la unión es excesiva, el coeficiente de reflexión y las perdidas resistivas se incrementan.

\subsubsection{Conector tipo N}
Conectores de propósito general, sexuados, relativamente económicos. Conectores robustos de 7 mm se desempeña bien en entornos de operación extremos y en aplicaciones que requieran conexiones repetidas.

El conector tipo N (Navy) de 50 Ohm fuen diseñado en la decada de 1940 para uso militar, operando inicialmente a 4 GHz. Posteriormente en la decada de 1960 lo avances tecnologicos incrementaro la frecuencia a 12 GHz y luego, en modo libre, hasta 18 GHz.

\subsubsection{Conector 3.5mm}

Conectores de precisión de 3.5 mm, sexuados con dieléctrico de aire. El aire provee aislamiento dieléctrico entre el conductor central y externo. Una arandela plástica dentro del cuerpo del conector soporta al conductor central.
Los conectores de 3.5 mm hacen juego con conectores SMA. Los conectores de 3.5 mm son lo suficientemente duraderos pra uso en repetidas conexiones.
El conector de 3.5 mm fue desarrollado inicialmente en Hewlet Packard, ahora Agilent Technologies. Su diseño es altamente robusto, compatibles con las dimensiones físicas de los conectores SMA. Conector de modo libre hasta 34 GHz. 
Los conectores de precisión de 3.5 mm emplean dieléctrico de aire. El conductor central del macho o hembra es soportado por arandelas plásticas.

\subsection{Caracteristicas mecanicas}

\begin{table}[h!]
	\centering
	\begin{tabular}{cc}
		\toprule
		{\bfseries Tipo de conector} 		& {\bfseries Torque \si{\pounds}-inches (\si{\newton}-\si{\meter})} 	\\
		\midrule
		Precisión 7 \si{\milli\meter} 		& 	12 (136) 		\\
		Precisión 3.5\si{\milli\meter}		& 	8 (90)			\\
		SMA*								& 	5 (56)			\\
		Precisión 2.5 \si{\milli\meter}		& 	8 (90)			\\
		Precisión 1.85 \si{\milli\meter}	& 	8 (90)			\\
		Tipo-N								&	12 (136)		\\		
		\bottomrule
	\end{tabular}
\end{table}

* Usar el valor de torque SMA para conexión de SMA macho con 3.5 mm hembra. Usar el valor de torque 3.5 mm para conexión 3.5 mm macho con SMA hembra.
Tabla 7: Valores recomendados de torque para conectores [1]
Adaptadores para puerto de prueba de 3.5 mm a tipo-N (50 Ohm test port adapter) [2]
Parte del kit 11878A de Keysight, empleados para pruebas en dispositivos con conectores de 3.5 mm en equipos de medición con conectores tipo N. Cada adaptado presenta un conector tipo N en un extremo y un conector de 3.5 mm de precisión en el extremo opuesto.
Los conectores SMA hacen juego con los conectores de precisión de 3.5 mm. Los conectores SMA no son conectores de precisión, un conector SMA desgastado o fuera de tolerancia puede dañar a un conector de 3.5 mm incluso destruirlo en la primera conexión [2]. Se debe inspeccionar con cuidado un conector.

\begin{itemize}
	\item Introducirlos de manera recta.
	\item Asegurar que el pin de contacto macho esta alineado de manera precisa con el pin hembra.
	\item No sobre ajustar los conectores.
	\item Nunca rotar los conectores por su parte central (al girar el dispositivo).
	\item Únicamente girar la rosca externa del conector macho.
	\item Usar torque de 5 lb-pulgada (50 \si{\newton}-\si{\meter}) para la conexión.

\end{itemize}

	\section{Conectores}

El rango de frecuencia de cualquier conector esta limitado por el primer modo de propagación de la guía de onda circular
en la estructura coaxial. Disminuir el diámetro del conductor externo incrementa la más alta frecuencia usable.
Rellenar el espacio de aire con dieléctrico disminuye la frecuencia más alta usable.

Si las dimensiones mecánicas de los conectores no esta apareada, si el enchapado/plating no es el adecuado, o si la
separación de contactos en la unión es excesiva, el coeficiente de reflexión y las perdidas resistivas se incrementan.

\subsection{Conector tipo-N}

Conectores de propósito general, sexuados, relativamente económicos. Conectores robustos de 7 mm se desempeña bien en
entornos de operación extremos y en aplicaciones que requieran conexiones repetidas.

El conector tipo N (Navy) de 50 Ohm fuen diseñado en la decada de 1940 para uso militar, operando inicialmente a 4 GHz.
Posteriormente en la decada de 1960 lo avances tecnologicos incrementaro la frecuencia a 12 GHz y luego, en modo libre,
hasta 18 GHz.

Los conectores de Agilent tipo N operan hasta 18GHz. Son compatibles con el estándar MIL-C-39012. \ Algunos productos
emplean un conector de 75 Ohm que usan un conector tipo N con un diámetro menor del conductor central.		

\subsection{Conector 3.5mm}

Conectores de precisión de 3.5 mm, sexuados con dieléctrico de aire. El aire provee aislamiento dieléctrico entre el
conductor central y externo. Una arandela plástica dentro del cuerpo del conector soporta al conductor central.

Los conectores de 3.5 mm hacen juego con conectores SMA. Los conectores de 3.5 mm son lo suficientemente duraderos pra
uso en repetidas conexiones.

El conector de 3.5 mm fue desarrollado inicialmente en Hewlet Packard, ahora Agilent Technologies. Su diseño es
altamente robusto, compatibles con las dimensiones físicas de los conectores SMA. Conector de modo libre hasta 34 GHz. 

Los conectores de precisión de 3.5 mm emplean dieléctrico de aire. El conductor central del macho o hembra es soportado
por arandelas plásticas.

\begin{center}
	\tablefirsthead{}
	\tablehead{}
	\tabletail{}
	\tablelasttail{}
	\begin{supertabular}{|m{2.952cm}|m{4.323cm}|}
		\hline
		\centering Tipo de conector &
		\centering Torque lb-pulgada (N-cm)\\\hline
		\centering Precisión 7 mm &
		\centering 12 (136)\\\hline
		\centering Precisión 3.5 mm &
		\centering 8 (90)\\\hline
		\centering SMA * &
		\centering 5 (56)\\\hline
		\centering Precisión 2.4 mm &
		\centering 8 (90)\\\hline
		\centering Precisión 1.85 mm &
		\centering 8 (90)\\\hline
		\centering Tipo-N &
		\centering 12 (136)\\\hline
	\end{supertabular}
\end{center}

* Usar el valor de torque SMA para conexión de SMA macho con 3.5 mm hembra. Usar el valor de torque 3.5 mm para conexión
3.5 mm macho con SMA hembra.

Tabla Valores recomendados de torque para conectores [1]

Adaptadores para puerto de prueba de 3.5 mm a tipo-N (50 Ohm test port adapter) [2]

Parte del kit 11878A de Keysight, empleados para pruebas en dispositivos con conectores de 3.5 mm en equipos de medición
con conectores tipo N. Cada adaptado presenta un conector tipo N en un extremo y un conector de 3.5 mm de precisión en
el extremo opuesto.

Los conectores SMA hacen juego con los conectores de precisión de 3.5 mm. Los conectores SMA no son conectores de
precisión, un conector SMA desgastado o fuera de tolerancia puede dañar a un conector de 3.5 mm incluso destruirlo en
la primera conexión [2]. Se debe inspeccionar con cuidado un conector.

Se deben seguir ciertas precauciones al unir conectores SMA con conectores de precisión de 3.5 mm,

\begin{itemize}
	\item Introducirlos de manera recta
	\item Asegurar que el pin de contacto macho esta alineado de manera precisa con el pin hembra.
	\item No sobre ajustar los conectores.
	\item Nunca rotar los conectores por su parte central (al girar el dispositivo).
	\item Únicamente girar la rosca externa del conector macho.
	\item Usar torque de 5 pulgadas-lb (50 N-cm) para la conexión
\end{itemize}

\begin{center}
	\tablefirsthead{}
	\tablehead{}
	\tabletail{}
	\tablelasttail{}
	\begin{supertabular}{|m{2.677cm}|m{3.793cm}|}
		\hline
		\centering Numero de parte &
		~
		\\\hline
		\centering 1250-1744 &
		\centering 3.5 mm (m) a tipo-N (m)\\\hline
		\centering 1250-1745 &
		\centering 3.5 mm (f) a tipo-N (f)\\\hline
		\centering 1250-1750 &
		\centering 3,5 mm (m) a tipo-N (f)\\\hline
	\end{supertabular}
\end{center}
Tabla \stepcounter{Table}{\theTable}: Adaptadores de 3.5 mm a tipo N [2]	

\begin{center}
	\tablefirsthead{}
	\tablehead{}
	\tabletail{}
	\tablelasttail{}
	\begin{supertabular}{|m{2.756cm}|m{5.1660004cm}|}
		\hline
		\centering Conector &
		\centering Rango de frecuencia útil (GHz)\\\hline
		\centering Presicion 7 mm &
		\centering DC a 20\\\hline
		\centering Tipo N &
		\centering DC a 18 GHz\\\hline
		\centering PSC-N &
		\centering DC a 18 GHz\\\hline
		\centering SMA &
		\centering DC a 23\\\hline
		\centering Precision 3.5 mm &
		\centering DC a 34\\\hline
		\centering PSC-3.5 mm &
		\centering DC a 34\\\hline
		\centering Precision 2.4 mm &
		\centering DC a 50\\\hline
		\centering PSC-2.4 mm &
		\centering DC a 50\\\hline
	\end{supertabular}
\end{center}

Tabla \stepcounter{Table}{\theTable}: Rango de frecuencia útil de diversos tipos de conectores [3]		
\subsection{Adaptadores coaxiales de precisión de 3.5 mm 83059}

Conectores coaxiales de precisión \ grado instrumentación de 3.5 mm, ofrecen un sobresaliente desempeño hasta 26.5 GHz.

Con SWR mejor que 1.05. Se emplean como adaptadores entre dispositivos y equipos de medición de alto costo. 	

\begin{figure}
	\centering
	\begin{minipage}{15.656cm}				
		\includegraphics{Imagenes/EsquemaConectors35mm.png}
		\caption{Conectores de precisión coaxiales de 3.5 mm}
		\label{Fig:ConectoresPrecision35mm}
	\end{minipage}
\end{figure}

\begin{center}
	\tablefirsthead{}
	\tablehead{}
	\tabletail{}
	\tablelasttail{}
	\begin{supertabular}{|m{1.2659999cm}|m{2.873cm}|m{2.9329998cm}|m{3.136cm}|m{4.4570003cm}|}
		\hline
		\centering Modelo &
		\centering Tipo de conector &
		\centering Frecuencia (GHz) &
		\centering Perdida de retorno típica &
		\centering Perdida de inserción típica\\\hline
		\centering 83059A &
		\centering 3.5 mm (m-m) &
		\centering DC -26.5 &
		\centering {}-32 dB  &
		\centering 0.074 dB \\ \hline
		\centering 83059B &
		\centering 3,5 mm (f-f) &
		\centering DC -26.5 &
		\centering {}-32 dB &
		\centering 0.074 dB\\\hline
	\end{supertabular}
\end{center}


