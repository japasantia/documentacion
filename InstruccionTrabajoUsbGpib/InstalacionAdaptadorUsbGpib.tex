% Para documento texto corto
\documentclass[paper=letter,oneside,fontsize=12pt]{article}
%\documentclass[paper=letter,oneside,fontsize=12pt, 
% parskip=full]{scrartcl}

% Establecer dimensiones de los margenes
%\ usepackage[inner=1.5cm,outer=3cm,top=2cm,bottom=4cm,
%	bindingoffset=5mm]{geometry}
\usepackage[left=3cm,right=3cm,top=3cm,bottom=3cm,
bindingoffset=0cm, footskip=0cm, headheight=2cm]{geometry}

% Permite ingresar caracteres acentuados y especiales 
% sin necesidad de emplear comando
% utf8 codificacion de entrada Unicode (mas simbolos que ASCII)
\usepackage[utf8]{inputenc}

% T1 encoding for European, English, American text
\usepackage[T1]{fontenc}
% Fuente escalable
\usepackage{lmodern}

% Carga babel, idioma ingles
\usepackage[english,spanish]{babel}

% Mejor jsutificacion, tipografia alta calidad.
\usepackage{microtype}
% Para unir columnas y filas en tablas
\usepackage{array}

% Agrega comandos extra al comando tabular
% \toprule, \midrule, \bottomrule
\usepackage{booktabs}
% Tablas con ancho establecido por usuario
\usepackage{tabularx}
% Para posicionamiento preciso de tablas dentro del texto
\usepackage{float}

% Encabezados personalizados
\usepackage{fancyhdr}
\usepackage{graphicx}

% Permite obtener el numero de la ultima pagina
\usepackage{lastpage}

% Paquetes para figuras
% Paquete caption para titulos figuras
% Paquete subcaption para subfiguras
\usepackage{caption}
\usepackage{subcaption}

% Espaciado inteligente
\usepackage{xspace} 

% Formato direccione URL
% \usepackage{hyperref}

% Cabeceras
\pagestyle{fancy}
% Borra cabecera y pie actuales
\fancyhead{}
% Cintillo cabecera
%\chead{
%	\includegraphics[width=150mm]{Imagenes/Cabecera.png}
%}
\fancyhead[L]{\includegraphics[width=0.3\textwidth]{Imagenes/cabecera.pdf}}
\fancyfoot[C]{ 
	\begin{center}
		\begin{tabularx}{\textwidth}{|X|m{10cm}|X|}
			\hline		
			\begin{center}
				\includegraphics[height=0.8cm]{Imagenes/pie-izq.pdf}
			\end{center} &	
			\begin{center}
				 Caracter Confidencial 
			\end{center} &
			\begin{center}
				\includegraphics[height=0.8cm]{Imagenes/pie-der.pdf} 			 
				\thepage~/~\pageref{LastPage} 
			 \end{center} \\	
			\hline 
		\end{tabularx}	 
	\end{center}
}

% Numeracion de paginas
% numeros arabigos
\pagenumbering{arabic}

\begin{document}
	
	%\begin{titlepage}		
	
	\begin{center} 
		
		\vspace{10cm}
		
		\begin{Large} 
			\textsc{Dirección de Servicios de Certificación}
			\\[5pt]
			\textsc{Laboratorio de Ensayos de Compatibilidad~Electromagnética~Radiada}
		\end{Large}
		
		\vspace{10cm}
		
		
		\begin{LARGE}			
			\textsc{Instalación del adaptador USB/GPIB~Agilent~82357B en linux}
		\end{LARGE}			
		
		\begin{large}			
			\textsc{Construcción de la librería \emph{linux-gpib} y carga de firmware}
		\end{large}	

		
		\begin{table}[!b]
			\begin{tabularx}{\linewidth}{|X|X|X|X|}	
				\hline				
				\multicolumn{2}{|l|}{\textbf{CÓDIGO}: FO-IT-002} & \multicolumn{2}{l|}{\textbf{N DOC:}} \\
				\hline
				Originado por:	& 	Elaborado por: & 
				Revisado por: 	& 	Aprobado por: \\
				\hline
				Br. Arias B., Jose A. & Br. Arias B., Jose A. & - & - \\
				\hline
				\textbf{Fecha: 20/06/2017 } & 
				\textbf{Fecha: 20/06/2016} & 
				\textbf{Fecha: } &
				\textbf{Fecha: } \\				
				\hline
			\end{tabularx}	
		\end{table}	
		
	\end{center}
	
	%\end{titlepage}
	
	\clearpage
	
	\tableofcontents
	
	\section{Objetivos}
		\begin{itemize}
			\item  Describir el proceso de instalación adaptador USB/GPIB Agilent 82357B.
			\item Instalar y construir, a partir del código fuente, la librería c de soporte (linux-gpib).
			\item Obtener y cargar el firmware para el adaptador.
		\end{itemize}
		
	\section{Alcance}
		Describe el proceso de instalación adaptador USB/GPIB Agilent 82357B, la instalación y construcción de la librería c de soporte (linux-gpib) a partir del código fuente y la obtención y carga del firmware para el adaptador.
	
		
	\section{Documentos de referencia}
	\subsection{Enlace para código fuente linux-gpib}
	A la fecha de este documento, se encuentra el código fuente para la libreria linux-gpib en la versión 4.0.4 
	
	\texttt{https://sourceforge.net/projects/linux-gpib/files/linux-gpib\%20for\%203.x.x\%20and\%202.6.x\%20kernels/}
	
	\texttt{http://linux-gpib.sourceforge.net/}
	
	\texttt{http://linux-gpib.sourceforge.net/doc\_html/x263.html\#AGILENT-82357A}
	
	Descripción del proceso para obtención y carga del firmware para el adaptador USB/GPIB 82357B
	
	\texttt{https://gist.github.com/turingbirds/6eb05c9267a6437183a9567700e8581a}
	
	Instalación de los  archivos de cabecera para el núcleo (kernel) apropiado al numero de versión de la distribución de linux que se utilice
	
	\texttt{https://www.cyberciti.biz/faq/howto-install-kernel-headers-package/}
	
	\texttt{http://git.net/ml/linux.hardware.gpib.general/2008-02/msg00001.html}
	
	\section{Términos y definiciones}
	
	\section{Personal autorizado}
		Personal técnico Cendit con interés en el uso del dispositivo
		
	\section{Personal requerido}
		Todo aquel personal técnico Cendit con interés en el uso del dispositivo
		
	\section{Materiales}
		
		
	\section{Herramientas y equipos}
		
		\begin{itemize}
			\item Adaptador USB/GPIB Agilent 82357B
			\item Computador personal con puerto USB.
		\end{itemize}
	
	\section{Equipos de protección personal}
		No se requieren equipos protección personal
		
	\section{Precauciones de seguridad}
		Para ejecutar esta actividad no se preveen precauciones de seguridad
		
	\section{Descripción de la actividad}	
		
		Instalar archivos de cabecera para el núcleo de linux, de acuerdo a su versión.
		
		Descargar código fuente de linux-gpib, seleccionar version acorde a versión de kernel de linux en uso
		
		Construir este código fuente
		
		Descargar el firmware para el adaptador USB/GPIB
		
		Instalar la utilidad fxload
		
		Usar fxload para cargar el firmware.	
		
	
	\section{Anexos}	

	
\end{document}